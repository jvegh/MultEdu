%%
%% The program-listings related macros for the MultEdu system
%%
% % It uses some definitions from MEMacros, so it should be after that file
% %
\ifx\MayFloat\undefined
\else
	\floatstyle{plaintop}
	\newfloat{program}{thp}{lpp}[chapter]
	\ifx\LectureLanguage\undefined
	\floatname{program}{Listing}
	\else
	  \ifthenelse{\equal{\LectureLanguage}{magyar}}
	   {\floatname{program}{Programlista}
	   \MEDebugMessage{MEListings: Changed name to Programlista by \LectureLanguage}
	   }
	   {	   \MEDebugMessage{MEListings:  name not changed to Programlista by \LectureLanguage}
	   }%false
	\fi
\fi

\ifx\eBook\undefined
	\def\lstsize{\scriptsize}
\else
	\def\lstsize{\tiny}
\fi

%% Set parameters for the appearance of the listings

\newcommand{\MESetStandardListingFormat}
{
	\lstset{
	    literate=% allow Hungarian umlauts
         {á}{{\'}a}1
         {Á}{{\'}Á}1
         {é}{{\'}e}1
         {É}{{\'}E}1
         {í}{{\'i}}1
         {Í}{{\'I}}1
         {ó}{{\'}o}1
         {Ó}{{\'}O}1
         {ú}{{\'u}}1
         {Ú}{{\'}u}1
         {ö}{{\"o}}1
         {Ö}{{\"}Ö}1
         {ü}{{\"}u}1
         {Ü}{{\"}U}1
	    {ő}{{\H{}o}}1
	    {ö}{{\H{}O}}1
	    {ű}{{\H{}U}}1
	    {Ű}{{\H{}u}}1
	    	}	    
	\lstset{
		%         numbers=left,               % Place for line numbers
		numberstyle=\tiny,          % Style of line numbers
		numbersep=5pt,              % Distance of line numbers from source
		tabsize=2,                 	 % Size of Tabs
		inputencoding=utf8/latin2,	% input encoding, allow accented chars
		extendedchars=true,         %
		escapechar=\@,
		breaklines=true,        % sets automatic line breaking
		columns=fullflexible,  % eliminates the spacing:
		breakatwhitespace=true,    % sets if automatic breaks should only happen at whitespace
        keepspaces=true,
        escapeinside={\%*}{*)},          % if you want to add a comment within your code
		frame=tb, 
		framerule=.5pt, 
		rulecolor= \color{SeparatorColor},
		backgroundcolor=\color{ivory},
%		basicstyle=\ttfamily\color{black}\lstsize\bfseries, 
		basicstyle=\ttfamily\color{black}\normalsize\bfseries, 
		keywordstyle=\bfseries\color{darkmagenta},
		identifierstyle=\bfseries\color{NavyBlue},
		commentstyle=\itshape\bfseries\color{ForestGreen},
		stringstyle=\itshape\bfseries\color{burntorange}, % Color for strings
		lineskip=0pt,aboveskip=4pt,belowskip=2pt,
		framesep=4pt,rulesep=2pt, %framerule=.25pt,
		showspaces=false,           % 
		showtabs=false,             % 
		framexleftmargin=0pt,
		framexrightmargin=0pt,
		showstringspaces=false      % Show empty spaces? 	
	}	%Take optional arguments
}%\MESetStandardListingFormat

\newcommand\MESetListingFormat[2][]	% Optional is the extra arguments
{
    \MESetStandardListingFormat
    \lstset
    	{
%		#1,
		language={#2},		% Select the language	
        numberstyle=\tiny,          % Style of line numbers
        numbersep=5pt,              % Distance of line numbers from source
        tabsize=2,                 	 % Size of Tabs
        inputencoding=utf8/latin2,	% input encoding, allow Hungarian umlauts
        extendedchars=true,         %
        escapechar=\@,
		breaklines=true,        % sets automatic line breaking
		breakatwhitespace=true,    % sets if automatic breaks should only happen at whitespace
		escapeinside={\%*}{*)},          % if you want to add a comment within your code
		frame=tb, 
		framerule=.5pt, 
		rulecolor= \color{SeparatorColor},
		backgroundcolor=\color{ivory},
        basicstyle=\ttfamily\color{black}\lstsize , 
        keywordstyle=\bfseries\color{magenta},
        identifierstyle=\bfseries\color{NavyBlue},
 		commentstyle=\itshape\bfseries\color{ForestGreen},
        stringstyle=\itshape\bfseries\color{burntorange}, % Color for strings
        lineskip=0pt,aboveskip=4pt,belowskip=2pt,
        framesep=4pt,rulesep=2pt, %framerule=.25pt,
        showspaces=false,           % 
        showtabs=false,             % 
        xleftmargin=-1pt,
        framexbottommargin=4pt,
        framextopmargin=4pt,
        gobble=5,
		framexleftmargin=0pt,
		framexrightmargin=0pt,
         showstringspaces=false      % Show empty spaces? 	
	}	%Take optional arguments
	\lstset{#1}
}%MESetListingFormat

\newlength{\FigWidth}\newlength{\FigHeight}
\makeatletter
%% Insert a source file in the text, with optional decorations
%%%Usage \MESourceFile[keys]{source file}{caption}{label}{ScaleFactor}
\newcommand\MESourceFile[5][]{
	\setkeys{MEMacros}{wide=false,language={[ANSI]C},options={}, decorations={},#1}% 
	% Define the environment: a 'Program' if it might float, a simple caption if not
	\MESetStandardListingFormat
	\ifx\MayFloat\undefined % A kind of slides
		\def\startsource{ \begin{center}
			              \setlength{\FigWidth}{#5\textwidth}
		                  {\color{HeadingColor}\bfseries\scriptsize #3}
			              \par\vskip-\baselineskip
		                }
		\def\stopsource{\end{center}}
	\else %% Some printable form, either A4 book,  WEB book or eBook
		\ifKV@MEMacros@wide %% It is a wide floating version
		   \def\startsource{ \begin{center}
			                \setlength{\FigWidth}{#5\textwidth} 
			                \begin{program*}[!hbt] 
			               }
			\def\stopsource{\end{program*}\end{center}}			
		\else % It is the narrow (one-column} version)
			\if@twocolumn
			  \setlength{\FigWidth}{#5\columnwidth}
			\else
			  \setlength{\FigWidth}{.7\textwidth}
			  \setlength{\FigWidth}{#5\FigWidth}
			 \fi
			\def\startsource{\begin{center}\begin{program}[!hbt]}
			\def\stopsource{\end{program}\end{center}}
		\fi
	\fi
	%% Start printing the figure here:
	\setlength{\FigHeight}{.8\textheight} 
	\noindent\startsource %{\caption{#3}} \hskip-1em
%	\maxsizebox{#5\FigWidth}{#5\FigHeight}
	{
		\mbox{}\phantomsection
		\noindent\begingroup\protected@edef\x{\endgroup\noexpand
			\lstinputlisting[language={\ME@language}, \ME@options, label=#4, name=#4,
			\ifx\LecturePrintable\undefined\else	caption ={#3}\fi
			]{#2}}
		\x
		\ME@decorations % Decorating comments
	}					
	\stopsource
}
\makeatother

%\newlength{\mywidth} % Used in defining listing width
% http://mirror.unl.edu/ctan/graphics/pgf/contrib/tikzmark/tikzmark.pdf
% % Prepare for some decorations on the listing files
\makeatletter
\newif\iflst@linemark
	
\lst@AddToHook{EveryLine}{%
 \begingroup
 \advance\c@lstnumber by 1\relax
 \pgfmark{line-\lst@name-\the\c@lstnumber-start}%
 \endgroup
}
	
\lst@AddToHook{EOL}{\pgfmark{line-\lst@name-\the\c@lstnumber-end}%
	\global\lst@linemarktrue
}
	
\lst@AddToHook{OutputBox}{%
 \iflst@linemark
 \pgfmark{line-\lst@name-\the\c@lstnumber-first}%
 \global\lst@linemarkfalse
 \fi
}
	
\def\tkzlst@fnum#1\relax#2\@STOP{%
 \def\@test{#2}%
 \ifx\@test\@empty
 \def\tkzlst@start{0}%
 \else
 \@tempcnta=#1\relax
 \advance\@tempcnta by -1\relax
 \def\tkzlst@start{\the\@tempcnta}%
 \fi
}
	
\lst@AddToHook{Init}{%
 \expandafter\tkzlst@fnum\lst@firstnumber\relax\@STOP
 \pgfmark{line-\lst@name-\tkzlst@start-start}%
}

% % Put a balloon around some lines in a source
% Usage: \MESourcelinesHighlight{BallonName}{SourceName}{FirstLine}{LastLine}
\newcommand\MESourcelinesHighlight[4]{%
  \pgfmathtruncatemacro\pgf@temp{%
   #3-1
  }%
  \iftikzmark{line-#2-\pgf@temp-start}{%
   \iftikzmark{line-#2-#3-first}{%
     \xdef\b@lines{({pic cs:line-#2-\pgf@temp-start} -| {pic cs:line-#2-#3-first})}%
   }{%
     \iftikzmark{line-#2-#3-start}{%
       \xdef\b@lines{({pic cs:line-#2-\pgf@temp-start} -| {pic cs:line-#2-#3-start})}%
     }{%
       \xdef\b@lines{(pic cs:line-#2-\pgf@temp-start)}%
     }%
   }%
  }{%
   \xdef\b@lines{}%
  }%
  \foreach \k in {#3,...,#4} {%
   \iftikzmark{line-#2-\k-first}{%
     \xdef\b@lines{\b@lines (pic cs:line-#2-\k-first) }
   }{}
   \iftikzmark{line-#2-\k-end}{%
     \xdef\b@lines{\b@lines (pic cs:line-#2-\k-end) }
   }{}
  }%
  \ifx\b@lines\pgfutil@empty
  \else
  \edef\pgf@temp{\noexpand\tikz[remember picture,overlay]\noexpand\node[fit={\b@lines}, color=ForestGreen,yshift=-2pt,
     draw, fill=green!30, opacity=0.4,  inner sep=0pt, rounded corners=2pt] (#1) {};
  }%
		\pgf@temp
  \fi
		}
\makeatother

% Prepare a comment relating to a balloon as rounded rectangle 
% Usage \MESourceBalloonComment[keys]{Balloon}{ShiftPosition}{Comment}{CommentShape}
% Keys: width[=3cm]
% 		color[=deeppeach]
\makeatletter
% ========= KEY DEFINITIONS =========
\newcommand\MESourceBalloonComment[5][]
{	% First draw the text box
	\setkeys{MEMacros}{width=3cm,#1}% 
	\setkeys{MEMacros}{color=deeppeach,#1}% 
   \begingroup%
  \tikz[remember picture,overlay]
  \node[rectangle, draw,  rounded corners,drop shadow,  align=left,  fill=\ME@color, text width=\ME@width, font=\lstsize] %\bfseries]
 at  ($(#2.east)+(\ME@width,0) +(#3)$) (#5) {\begin{minipage}{\ME@width}\lstsize#4\end{minipage}} ;
   % Now draw the connecting arrow
   \tikz[remember picture,overlay] 
   \draw[->,thick,color=burntorange] 
    (#5.west) --  ($ (#5.west) + (-.2cm,0) $)  |- (#2.east)  ;
	\endgroup
}
\makeatother

\makeatletter
% % Put numbered balls after the line 'Lineno'  in source 'Source'
% Usage: \MESourcelineListBalls[keys]{ListingLabel}{List of lines}
% Possible keys: color	% Color of the balls
%                        number	% Starting seq number
\newcounter{qan}\newcounter{qano}
\newcommand\MESourcelineListBalls[3][]{%
	\setkeys{MEMacros}{color=orange,#1}% 
	\setkeys{MEMacros}{number=1,#1}% 
\setcounter{qan}{\ME@number-1}
\setcounter{qano}{0}
   \begingroup%
    \foreach \x in {#3}
    {  \addtocounter{qan}{1}
    	\addtocounter{qano}{1}
      \only<\arabic{qano}>%
  {\tikz[remember picture,overlay]
    {\expandafter\node[circle, inner sep=2pt, draw,fill=\ME@color,ball color=\ME@color, shading=ball, font=\scriptsize\bfseries, drop shadow]
   at  ([xshift=+10pt,yshift=+2pt]{pic cs:line-#2-\x-end}) {\lstsize\arabic{qan}};\expandafter}}
   }
	\endgroup
}
\makeatother

%% Put a numbered ball in the text, to reference balls on a listing
%% Usage: \MEBall{ListingLabel}{Number}
\newcommand{\MEBall}[2]{%
{\tiny#1}~\hskip-4pt\raisebox{-.08cm}{
\tikz \node (1ex,1ex)
 [circle,draw,ball color=green, shading=ball,  font=\bfseries, scale=0.55] {#2};}%
~\hskip-4pt}% MEBall


\makeatletter
% Prepare a source comment relating to a source line
% Usage \MESourcelineComment[keys]{Sourcelabel}{LineNo}{ShiftPosition}{Comment}{CommentShape}
% Keys: width[=3cm]
% ========= KEY DEFINITIONS =========
%	\newlength{\commentboxlength}
\newcommand\MESourcelineComment[6][]
{
	\setkeys{MEMacros}{width=3cm,#1}% 
	\setkeys{MEMacros}{color=deeppeach,#1}% 
%	\setlength{\commentboxlength}{\ME@width}
   \begingroup%
		  \tikz[remember picture,overlay]
		  \node[rectangle, draw,  rounded corners,drop shadow,  align=left,
				fill=\ME@color, text width=\ME@width, font=\lstsize]
		   at  ($(#4) +(\ME@width,0) +({pic cs:line-#2-#3-end})$) (#6) {\begin{minipage}{\ME@width}\lstsize #5 \end{minipage}} ;
%		\tikz[remember picture,overlay] \draw[->,thick,color=burntorange]  (#6.south) |- ({pic cs:line-#2-#3-end});
		\tikz[remember picture,overlay] \draw[->,thick,color=burntorange]  (#6.west)  --  ($ (#6.west) + (-.2cm,0) $) |- ({pic cs:line-#2-#3-end});
	\endgroup
}
\makeatother


\makeatletter
% Prepare a figure comment relating to a source text
% Usage \MESourcelineFigure[keys]{SourceName}{LineNo}{ShiftPosition}{Graphics file}
% Keys: width[=3cm]
% ========= KEY DEFINITIONS =========
\define@key{MESourcelineFigure}{width}{\def\pb@width{#1}}
\newcommand\MESourcelineFigure[5][]
{
	\setkeys{MESourcelineFigure}{width=3cm,#1}% 
	\begingroup%
	\tikz[remember picture,overlay]
	\node[ rectangle,  draw,  rounded corners, drop shadow]
	at  ($(#4)+({pic cs:line-#2-#3-end})$)  {\includegraphics[width=\pb@width]{#5}} ;
	\endgroup
}
\makeatother

\newlength{\mywidth} % Used in defining listing width
%% Prepare a figure from comparing the source files provided; 
%%Usage \MESourceFileCompare[keys]{source file1}{source file2}{caption}{label}
\makeatletter
% ========= KEY DEFAULTS =========
\newcommand\MESourceFileCompare[5][]{
\setkeys{MEMacros}{language={[ANSI]C},options={},#1}% 
	  \begingroup%
		% Set up the listing format, pass language and extra options
	\begingroup\edef\x{\endgroup\noexpand\MESetListingFormat[\ME@options]{\ME@language}}\x
			\if@twocolumn%% %	\makeatletter%
					 \setlength\mywidth{\columnwidth}
				\else% \@twocolumnfalse
					 \setlength\mywidth{.46\textwidth}
				\fi
		% % Define the environment: a 'Program' if it might float, a simple caption if not
		\begin{center}
		\ifx\MayFloat\undefined
			{\color{HeadingColor}\bfseries\scriptsize #4}
		\else
	  	    		{
	  	    		\begin{program*}[h!btp]	\caption{#4}\par\vskip-\baselineskip
	  	    		 }
		\fi\mbox{}\phantomsection
		\vskip-1\baselineskip
		\begin{tabular}{p{.48\mywidth}p{.48\mywidth}}
		\lstinputlisting[label=#5, name=#5, linewidth=\mywidth ] {#2}&
		\lstinputlisting[label=#5, name=#5,linewidth=\mywidth ] {#3}\\
		\end{tabular}
	\ifx\MayFloat\undefined	% nothing needed
	\else
		{	
		\end{program*}\vskip-\baselineskip}
	\fi
	\end{center}
  \endgroup%
}

%%Usage \MESourceFileWithResult[keys]{source file}{result file}{caption}{label}
\newcommand\MESourceFileWithResult[5][]{
	\setkeys{MEMacros}{language={[ANSI]C},options={},#1}% 
	\begingroup%
	% Set up the listing format, pass language and extra options
	\begingroup\edef\x{\endgroup\noexpand\MESetListingFormat[\ME@options]{\ME@language}}\x
	\if@twocolumn%
	\setlength\mywidth{\columnwidth}
	\else% \@twocolumnfalse
	\setlength\mywidth{.46\textwidth}
	\ifx\eBook\undefined \lstset{basicstyle=\tiny} \fi
	\fi
	% % Define the environment: a 'Program' if it might float, a simple caption if not
	\ifx\MayFloat\undefined
	{\color{HeadingColor}\bfseries\scriptsize #4}
	\else
	{
		\begin{program*}[h!btp]	\caption{#4}\par\ifx\WEBBook\undefined\vskip-\baselineskip\else\fi
		}
		\vglue-1.7\baselineskip
		\fi\mbox{}\phantomsection
		\vskip-.6\baselineskip\noindent
		\begin{tabular}{p{.48\mywidth}p{.48\mywidth}}
		%		\ifx\MayFloat\undefined\else\caption{#3}\fi %\vglue-.7\baselineskip
		\noindent\begingroup\protected@edef\x{\endgroup\noexpand
			\lstinputlisting[language={\ME@language}, \ME@options, label=#5, name=#5, linewidth=\mywidth,
			]{#2}}
		\x
		\ME@decorations % Decorating comments
			&
			\lstset{language={bash},backgroundcolor=\color{gray!20}}				
			\lstinputlisting[ linewidth=\mywidth,
				 ]
				 {#3}\\
		\end{tabular}
		\par\vskip-.6\baselineskip
		\ifx\MayFloat\undefined	% nothing needed
		\else
		{	
			\caption{#4}\label{#5}
		\end{program*}}
		\fi
		\endgroup%
	}
\makeatother

% % Define new "language" for diff viewers
\lstdefinelanguage{diff}{
  morecomment=[f][\color{blue}]{@@},     % group identifier
  morecomment=[f][\color{red}]-,         % deleted lines 
  morecomment=[f][\color{green}]+,       % added lines
  morecomment=[f][\color{magenta}]{---}, % Diff header lines (must appear after +,-)
  morecomment=[f][\color{magenta}]{+++},
}
%% Define the Do It Yourself calculator language
%%
\lstdefinelanguage{[DIY]Assembler}
{morekeywords={NOP, HALT, SETIM, CLRIM, ADD, ADDC, SUB, SUBC, AND, OR, XOR, CMPA,
							SHL, SHR, ROLC, RORC, INCA, DECA, INCX, DECX, LDA, STA,
							BLDX, BLDSP, BSTSP, BLDIV, PUSHA, POPA, PUSHSR, POPSR, JMP, JSR,
							JZ, JNZ, JN, JNN, LC, JNC, JO, JNO, RTS, RTI},
%directives={.ORG,.END,.BYTE},
sensitive=false,
morecomment=[l]{\#},
}

%% Define the ARM processor mnemonics
%%
\lstdefinelanguage{[ARM]Assembler}
{morekeywords={
	ADC, ADD, AND, ADR,
							B, BEQ, BNE, BCS, BHS, BCC, BLO, BMI, BPL, BVS, BVC,
			BIC, BL, BX, CMN, CMP, EOR, LDM, LDR, 
								MLA, MOV, MRS, MSR, MOV, MVN, ORR, RSB, RSC, SBC,
								STM, STR, SUB, SUBS,
								SWI, SWP, TEQ, TST},
%directives={AREA,CODE,READONLY,ENTRY,END},
sensitive=false,
morecomment=[l]{;},
}

% % Define x86-64 mnemonics
\lstdefinelanguage
   [x64]{Assembler}     % add a "x64" dialect of Assembler
   [x86masm]{Assembler} % based on the "x86masm" dialect
   % with these extra keywords:
   {morekeywords={CDQE,CQO,CMPSQ,CMPXCHG16B,JRCXZ,LODSQ,MOVSXD, %
                  POPFQ,PUSHFQ,SCASQ,STOSQ,IRETQ,RDTSCP,SWAPGS, %
                  rax,rdx,rcx,rbx,rsi,rdi,rsp,rbp, %
                  r8,r8d,r8w,r8b,r9,r9d,r9w,r9b}} % etc.


% % Define x86-64 mnemonics
\lstdefinelanguage
   [y86]{Assembler}     % add a "Y86" dialect of Assembler
   [x86masm]{Assembler} % based on the "x86masm" dialect
   % with these extra keywords:
   {morekeywords={ halt, nop, movl, rrmovl, irmovl, rmmovl, mrmovl, rrmovl, jump,%
                  xorl, andl, je, addl, jne, subl, pushl, popl, jXX,%
                  cmovXX, cmovl, cmovle, cmove, cmovne, cmovge, cmovg, return,%
                   rax,rbx,rcx,rdx,xmm0,rbp, load,mulss,%
                   OPl, ZF, SF, OF,%
                   eno, ecc, esv,
                   QCreate, QTCreate, QFCreate, QTerm, QCall, QAlloc,
                   QWait, QPWait, QIWait, QInt
                  },
      morecomment=[l]{\#},
      sensitive=false,
%     directives={.pos, .align, .long,}
     } % etc.                 

