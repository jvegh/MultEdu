%http://tex.stackexchange.com/questions/35825/pretty-table-of-contents/35835#35835
\ifx\DisableFancyTOC\undefined
% colors to be used
\usepackage{refcount}
\colorlet{myred}{HeadingColor}
%\definecolor{myred}{RGB}{127,0,0}
\definecolor{myyellow}{RGB}{169,121,69}
  % % OK, can make a fancy TOC
  % a command to circle the part numbers
  \newcommand\CCircle[1]{\tikz[overlay,remember picture] 
    \node[draw=myyellow,circle, text width=18pt,line width=1pt,align=center] {#1};}
  
  % redefinition of the name of the ToC
  \renewcommand\printtoctitle[1]{\HUGE\sffamily\bfseries\color{myred}#1}
  
  \makeatletter
  % redefinitions for part entries
  \renewcommand\cftpartfont{\Huge\sffamily\bfseries}
  \renewcommand\partnumberline[1]{%
      \hbox to \textwidth{\hss\CCircle{\textcolor{myyellow}{#1}}\hss}%
    \vskip3.5ex\color{myred}}
  
  \renewcommand*{\l@part}[2]{%
    \ifnum \c@tocdepth >-2\relax
      \cftpartbreak
      \begingroup
        {\interlinepenalty\@M
         \leavevmode
         \settowidth{\@tempdima}{\cftpartfont\cftpartname}%
         \addtolength{\@tempdima}{\cftpartnumwidth}%
         \let\@cftbsnum \cftpartpresnum
         \let\@cftasnum \cftpartaftersnum
         \let\@cftasnumb \cftpartaftersnumb
         \advance\memRTLleftskip\@tempdima \null\nobreak\hskip -\memRTLleftskip
         \centering{\cftpartfont#1}\par%
         }
        \nobreak
          \global\@nobreaktrue
          \everypar{\global\@nobreakfalse\everypar{}}%
      \endgroup
    \fi}
 \ifx\eBook\undefined 
  % redefinitions for chapter entries
  \renewcommand\chapternumberline[1]{\mbox{\Large\@chapapp~#1}\par\noindent}
  \renewcommand\cftchapterfont{\Large\sffamily}
  \cftsetindents{chapter}{0pt}{0pt}
  \renewcommand\cftchapterpagefont{\HUGE\sffamily\bfseries\color{SeparatorColor}}
 \else
  % redefinitions for chapter entries
  \renewcommand\chapternumberline[1]{\mbox{\small\@chapapp~#1}\par\noindent}
  \renewcommand\cftchapterfont{\Large\sffamily}
  \cftsetindents{chapter}{0pt}{0pt}
  \renewcommand\cftchapterpagefont{\Large\sffamily\bfseries\color{SeparatorColor}}
 \fi 
  \newcommand*{\l@mychap}[3]{%
    \def\@chapapp{#3}\vskip1ex%
    \par\noindent\begin{minipage}{\textwidth}%
    \parbox{4.7em}{%
      \hfill{\cftchapterpagefont#2}%
    }\hspace*{1.5em}%
    \parbox{\dimexpr\textwidth-4.7em-15pt\relax}{%
      \cftchapterfont#1%
    }%
    \end{minipage}\par\vspace{2ex}%
  }
  
  \renewcommand*{\l@chapter}[2]{%
    \l@mychap{#1}{#2}{}%
%    \l@mychap{#1}{#2}{\chaptername}%
  }
  
  \renewcommand*{\l@appendix}[2]{%
    \l@mychap{#1}{#2}{\appendixname}%
  }
  
  % redefinitions for section entries
  \renewcommand\cftsectionfont{\sffamily}
  \renewcommand\cftsectionpagefont{\sffamily\bfseries\itshape\color{SeparatorColor}}
  \renewcommand{\cftsectionleader}{\nobreak}
  \renewcommand{\cftsectionafterpnum}{\cftparfillskip}
  \cftsetindents{section}{7.5em}{3em}
  \renewcommand\cftsectionformatpnum[1]{%
    \hskip1em\hbox to 4em{{\cftsectionpagefont #1\hfill}}}
  
  % redefinitions for subsection entries
  \renewcommand\cftsubsectionfont{\sffamily}
  \renewcommand\cftsubsectionpagefont{\sffamily\itshape\color{myred}}
  \renewcommand\cftsubsectionleader{\nobreak}
  \renewcommand{\cftsubsectionafterpnum}{\cftparfillskip}
  \renewcommand\cftsubsectiondotsep{\cftnodots}
  \cftsetindents{subsection}{10.5em}{3em}
  \renewcommand\cftsubsectionformatpnum[1]{%
    \hskip1em\hbox to 4em{{\cftsubsectionpagefont #1\hfill}}}
  \makeatother
  
  \settocdepth{subsection}
  \setsecnumdepth{subsection}
  
  % length to be used when drawing a line from the top of the text area
  \newlength\Myhead
  \setlength\Myhead{\dimexpr\headheight+\headsep+1in+\voffset+5ex\relax}
  
  % length to be used when drawing a line to the bottom of the text area
  \newlength\Myfoot
  \setlength\Myfoot{\dimexpr\paperheight-\Myhead-\textheight-\footskip+5ex\relax}
  
  % auxiliary counter to place labels
  \newcounter{chapmark}
  
  % Adds a mark and a label at the beginning of each chapter entry in the ToC and draws a line 
  % from the start of the chapter to the bottom of the text area if the mark 
  % for the chapter ending lies in a different page than the one from the end of the chapter.
  % (the value of tjose pages is calculated using the label)
  % Must be used right before each \chapter command
%  \newcommand\StartMark{%
%%    \addtocontents{toc}{\protect\label{st\thechapmark}%
%%      \protect\begin{tikzpicture}[overlay,remember picture,baseline]   
%%      \protect\node [anchor=base] (s\thechapmark) {};%
%%      \ifnum\getpagerefnumber{st\thechapmark}=\getpagerefnumber{en\thechapmark} \else
%%      \protect\draw[myyellow,line width=3pt] let \protect\p3= (s\thechapmark),%
%%        \protect\p4 = (current page.south) in %
%%        ($ (4em,\protect\y3) + (0,-1ex) $) -- ($ (4em,\protect\y4) + (0,\protect\the\Myfoot)$);\fi
%%      \protect\end{tikzpicture}\par}%
%  }
  
  % Adds a mark and a label at the end of each chapter entry in the ToC and draws a line from 
  % the top of the text area to the ending of the chapter if the mark 
  % for the chapter ending lies in a different page than the one from the start of the chapter
  % if both marks are in the same page, simple draws a line connecting the marks
  % (the value of tjose pages is calculated using the label)
  % Must be used right after the last sectional unit (that will go to the ToC) belonging to 
  % a chapter
%  \newcommand\EndMark{
%%  \addtocontents{toc}{\protect\label{en\thechapmark}%
%%    \protect\begin{tikzpicture}[overlay,remember picture,baseline]   
%%    \protect\node [anchor=base] (e\thechapmark) {};
%%    \ifnum\getpagerefnumber{st\thechapmark}=\getpagerefnumber{en\thechapmark} 
%%    \protect\draw[myyellow,line width=3pt] let \protect\p1= (s\thechapmark), \protect\p2=(e\thechapmark) in ($ (4em,\protect\y1) + (0,-1ex) $) -- ($ (4em,\protect\y2) + (0,2ex) $);
%%    \else%
%%    \protect\draw[myyellow,line width=3pt] let \protect\p1= (e\thechapmark), \protect\p2=(current page.north) in ($(4em,\protect\y2) + (0,-\protect\the\Myhead)$) -- ($ (4em,\protect\y1) + (0,1.5ex) $);
%%  \fi
%%  \protect\end{tikzpicture}\par}\stepcounter{chapmark}%
%  }
  
\else % \DisableFancyTOC\undefined
%  \newcommand\StartMark{}
%  \newcommand\EndMark{}
  % % No, make just plain TOC
\fi %\DisableFancyTOC\undefined