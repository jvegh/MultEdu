% % Part of Multedu: set up printed PDF formats 
%% This is the common part, included in the format specific files
%%% Define macro to print out a message if requested
\newcommand\MEDebugMessage[1]{\ifx\EnableDebug\undefined \else  \par\message{====MultEdu====:#1}\fi}
\providecommand\titlepage{}	% % Must define title page format somewhere
%
\usepackage{beamerarticle}	% % The source text uses 'beamer' macros
\usepackage{stfloats}
\usepackage[most]{tcolorbox}	% rounded color boxes
\usepackage{wallpaper}
\usepackage[ISBN=978-80-85955-35-4]{ean13isbn}
\usepackage{url}
\tcbuselibrary{breakable}

%% Set up options for the session, global, user or none
% % The user can have his own options in his own /src directory,
% % otherwise default ones are used
%\IfFileExists{src/Options.tex}
%	{%%  The user options for the MultEdu system
% % The default options for the document. Can be ustomized in file src/Options.tex
%
% Uncomment some '\def's below to make them effective
\def\EnableDebug{YES}	% % Print debug messages while running
%\def\DisableGlossary{YES}	% % Do not prepare a glossary
%\def\DisableTableOfContents{YES} % % Do not print table of contents
%\def\DisableAbstract{YES}	%% Do not prepare abstract for the book
%\def\DisableAcronyms{YES}	%% Do not prepare list of acronyms
%\def\DisableCopyright{YES}	%% Do not insert copyright info
%\def\DisableIndex{YES} % % Do not prepare an index
%\def\DisableListings{YES}}	% % Disable making program listings
%\def\DisableLogo{YES} % % Disable using a logo
%\def\DisableBibliography{YES}	% Disable printing bibliography
%\def\EnableAnswers{YES}	% % Enable Q/A macros
%



%	\MEDebugMessage{Using user's own options}
%	}	% % User options
%	{\IfFileExists{../../common/MEOptions.tex}
%		{\input{../../common/MEOptions.tex}
%	     \MEDebugMessage{Using system wide options}
%		}	% % Default options
%		{
%	      \MEDebugMessage{Using NO options}
%		} % % No options at all
%	} % Options
%
%% Set up document-wide settings like title, subtitle, publisher, etc.
% % The user should have his own definitions in his own /src directory,
% % otherwise de default ones are used
\IfFileExists{src/Heading.tex}
	{	
%% Define the author, title and publisher
%% Please use accents, no encoding and fonts defined
  \ifthenelse{\equal{\LectureLanguage}{english}}
  { % In English
	\def\LectureAuthor{J\'anos V\'egh}
	\def\LectureTitle{How to use package MultEdu}
	\def\LectureSubtitle{(How to prepare interesting and attractive teaching material)}
	 %Optional
%	\def\LecturePublisher{Miskolc University Faculty of \dots and Informatics}
	\def\LectureRevision{V\Version\ \at2016.08.19}
  }% true
  {% NOT english
  }
  \ifthenelse{\equal{\LectureLanguage}{magyar}}
  {	% in Hungarian
	\def\LectureAuthor{V\'egh J\'anos}
	\def\LectureTitle{Hogyan haszn\'aljuk\\ a MultEdu csomagot} 
	\def\LectureSubtitle{(Hogyan k\'esz\'\i{}ts\"unk \'erdekes\\\ \'es vonz\'o tananyagot)} %opcionális
%	\def\LecturePublisher{Miskolci Egyetem \dots és Informatikai Kara}
	\def\LectureRevision{V\Version\ \at2016.08.19}
  }% true
  {% NOT magyar
  }
\def\LectureEmail{Janos.Vegh\at unideb.hu}
	
	
%% Define the bibliography file
%% By default, no file assumed. Allows for language-dependent bibliographies
\ifx\LectureLanguage\undefined
\def\LectureBibliography{src/Bibliography}
\else

\IfFileExists{src/Bibliographyhu}
{\def\LectureBibliography{src/Bibliography,src/Bibliographyhu}}
{\def\LectureBibliography{src/Bibliography}}
\fi

	
}	% % User heading
	{ %% No heading is to be used
	      \MEDebugMessage{NO heading information found}
	} % Options


\setsecnumdepth{subsection}
\settocdepth{subsection}
%%%%%%%%%%%%%%%%%%%%%%
%% Colors used by MultEdu
%%%%%%%%%%%%%%%%%%%%%%

\definecolor{ForestGreen}{rgb}{0.0, 0.4, 0.0}
\definecolor{bronze}{rgb}{0.8, 0.5, 0.2}
\definecolor{uclagold}{rgb}{1.0, 0.7, 0.0}
%%% Maybe for listings
%\definecolor{wheat}{rgb}{0.96, 0.87, 0.7}
%\definecolor{vanilla}{rgb}{0.95, 0.9, 0.67}
\definecolor{timberwolf}{rgb}{0.86, 0.84, 0.82}
\definecolor{platinum}{rgb}{0.9, 0.89, 0.89}
\definecolor{pearl}{rgb}{0.94, 0.92, 0.84}
%\definecolor{navajowhite}{rgb}{1.0, 0.87, 0.68}
\definecolor{moccasin}{rgb}{0.98, 0.92, 0.84}
\definecolor{cream}{rgb}{1.0, 0.99, 0.82}
\definecolor{burntorange}{rgb}{0.8, 0.33, 0.0}
\definecolor{bronze}{rgb}{0.8, 0.5, 0.2}
\definecolor{yellowgreen}{HTML}{ADFF2F}
\definecolor{coral}{rgb}{1.0, 0.5, 0.31}
\definecolor{deeppeach}{rgb}{1.0, 0.8, 0.64}
%
%%Maybe for boxes
\definecolor{desertsand}{rgb}{0.93, 0.79, 0.69}
%\definecolor{deepchampagne}{rgb}{0.98, 0.84, 0.65}
\definecolor{celadon}{rgb}{0.67, 0.88, 0.69}
\definecolor{almond}{rgb}{0.94, 0.87, 0.8}
\definecolor{NavyBlue}{rgb}{0.0, 0.0, 0.5}
\definecolor{RoyalBlue}{HTML}{4169E1}
\definecolor{firebrick}{rgb}{0.7, 0.13, 0.13}
%
%
%%Maybe for headings
\definecolor{uclagold}{rgb}{1.0, 0.7, 0.0}
\definecolor{darkmagenta}{rgb}{0.55, 0.0, 0.55}
\definecolor{ultramarine}{rgb}{0.07, 0.04, 0.56}
\definecolor{champagne}{rgb}{0.97, 0.91, 0.81}
\definecolor{mediumchampagne}{rgb}{0.95, 0.9, 0.67}
\definecolor{bubbles}{rgb}{0.91, 1.0, 1.0}
\definecolor{ivory}{rgb}{1.0, 1.0, 0.94}
\definecolor{goldenrod}{rgb}{0.85, 0.65, 0.13}
\definecolor{ufogreen}{rgb}{0.24, 0.82, 0.44}

\ifx\LectureForWEB\undefined
    \colorlet{HighlightColor}{bronze}
    \definecolor{webgreen}{rgb}{0,.5,0}
    \definecolor{webbrown}{rgb}{.6,0,0}
    \definecolor{webyellow}{rgb}{0.98,0.92,0.73}
    \definecolor{webgray}{rgb}{.753,.753,.753}
    \definecolor{webblue}{rgb}{0,0,.8}
    \definecolor{webred}{rgb}{0.8, 0, 0}   
\else
 	\colorlet{HighlightColor}{uclagold}
\fi


%% Allow for redefining used colors
\ifx\HeadingColor\undefined
% \colorlet{HeadingColor}{NavyBlue}
% \colorlet{HeadingColor}{RoyalBlue}
  \colorlet{HeadingColor}{ForestGreen}
\fi
\ifx\ListingColor\undefined
  \colorlet{ListingColor}{ivory}
\fi
\ifx\BookColor\undefined
  \colorlet{BookColor}{NavyBlue}
\fi
	\colorlet{SeparatorColor}{burntorange}

%%%
%%% Colors for tables
%%%
%%\colorlet{tableheadcolor}{HeadingColor!25} % Table header colour = 25% gray
\colorlet{tableheadcolor}{HeadingColor!25} % Table header colour = 25% gray
\newcommand{\headcol}{\rowcolor{tableheadcolor}} %
\colorlet{tablerowcolor}{HeadingColor!10} % Table row separator colour = 10% gray
\newcommand{\rowcol}{\rowcolor{tablerowcolor}} %

	\def\TitlePageColor{ForestGreen}
	\usepackage{pgfplotstable}


\pgfplotscreateplotcyclelist{my black white}{%
solid, every mark/.append style={solid, fill=gray}, mark=*\\%
densely dashdotted,every mark/.append style={solid, fill=gray},mark=diamond*\\%
densely dotted, every mark/.append style={solid, fill=gray}, mark=triangle*\\%
loosely dashed, every mark/.append style={solid, fill=gray},mark=*\\%
dotted, every mark/.append style={solid, fill=gray}, mark=square*\\%
densely dotted, every mark/.append style={solid, fill=gray}, mark=otimes*\\%
dashed, every mark/.append style={solid, fill=gray},mark=diamond*\\%
densely dashed, every mark/.append style={solid, fill=gray},mark=square*\\%
dashdotted, every mark/.append style={solid, fill=gray},mark=otimes*\\%
dasdotdotted, every mark/.append style={solid},mark=star\\%
}

\pgfplotscreateplotcyclelist{my color list}{%
solid, color=webblue, every mark/.append style={solid, fill=webblue}, mark=*\\%
densely dashdotted, color=webgreen, every mark/.append style={solid, fill=webred},mark=diamond*\\%
densely dotted, color=webbrown, every mark/.append style={solid, fill=webgreen}, mark=triangle*\\%
loosely dashed, color=webred, every mark/.append style={solid, fill=webbrown},mark=*\\%
dotted, color=webblue, every mark/.append style={solid, fill=webyellow}, mark=square*\\%
densely dotted, color=webgreen, every mark/.append style={solid, fill=gray}, mark=otimes*\\%
dashed, color=webbrown, every mark/.append style={solid, fill=gray},mark=diamond*\\%
densely dashed, every mark/.append style={solid, fill=gray},mark=square*\\%
dashdotted, every mark/.append style={solid, fill=gray},mark=otimes*\\%
dasdotdotted, every mark/.append style={solid},mark=star\\%
}

\MEDebugMessage{MEColors loaded}		% Set up used colors
%%% Now select the language

\MEDebugMessage{MELanguages: started to load}
\usepackage[T1]{fontenc}
\usepackage[utf8]{inputenc}
%\usepackage[latin2]{inputenc}
\usepackage{t1enc}
\usepackage{ifthen}

% % The default is the English text
% % \LectureLanguage can change to another language
% % \UseSecondLanguage allows to use two languages simultanously
%
\ifx\LectureLanguage\undefined
% % Using English only
  	\hyphenpenalty=2000
  	\tolerance=20000
	\def\MEAbstract{Abstract}
  \MEDebugMessage{MELanguages: Configured to use English only}
\else
  % % Some other language must be used
  %% Check if using Hungarian
  \ifthenelse{\equal{\LectureLanguage}{magyar}}
  { % % implement Hungarian documenting
    \ifx\UseSecondLanguage\undefined
	  \usepackage[english,magyar]{babel}
    \else
	  \usepackage[magyar]{babel}
    \fi
	\selectlanguage{magyar}
	\frenchspacing
  	\hyphenpenalty=2000
  	\tolerance=20000
	\let\Aref\relax	%% Needed for workaround \Aref bug
	\def\MEAbstract{Kivonat}
  	\MEDebugMessage{MELanguages: Hungarian language selected}
  % % http://tex.stackexchange.com/questions/195491/ifthenelse-equal-string-comparison-fails
  }% true
  {\MEDebugMessage{MELanguages: NOT Hungarian}% false
  }
  \ifthenelse{\equal{\LectureLanguage}{german}}
  {
  \MEDebugMessage{MELanguages: German}
  
  % % Just an example how to implement further languages
  }% true
  {  \MEDebugMessage{MELanguages: NOT German}
  
  }% false
\fi %\LectureLanguage

\providecommand*{\at}{\symbol{64}}	% To avoid the special handling for the @ symbol 
\providecommand*{\bs}{\symbol{92}}	%% back slash
\providecommand*{\bq}{\symbol{96}}	%% back quote
\providecommand*{\qs}{\symbol{34}}	%% quote symbol
\providecommand*{\xor}{\symbol{94}}	%% Up arrow  symbol
\providecommand*{\bbar}{\symbol{124}}	%% Up arrow  symbol
\MEDebugMessage{MELanguages: successfully loaded}

	% Set up selected language
%
% This file contains the general-purpose packages needed for the MultEdu system
%
\usepackage{morewrites}	% To eliminate No room for a new \write
\usepackage{graphicx}
\usepackage{xkeyval}	% Allow to use key-value pairs in arguments of macros
\usepackage{xstring}  % Strings in LaTeX
\usepackage{etoolbox}		% Utility package, mainly for packet writers
\usepackage{color,calc}
\usepackage{xifthen}	% Provides \isempty
\usepackage{enumitem}	% Set packed items?
\setitemize{noitemsep,topsep=0pt,parsep=0pt,partopsep=0pt}
	
\usepackage{adjustbox}
\usepackage{wasysym}	% for smiley
\usepackage{listingsutf8}	% Use program listings
\usepackage{colortbl}			% Color tables
%	\usepackage{longtable}			% Break tables at page boundaries
\usepackage{calc}
\usepackage{answers}
\usepackage{savesym} 	% http://comments.gmane.org/gmane.emacs.orgmode/72621
\usepackage{booktabs}		% Book quality tables
\savesymbol{iint}
\savesymbol{iiint}
\ifx\EnableTimer\undefined
\else
	\ifx\LectureTime\undefined\def\LectureTime{15}\fi
	\usepackage[timeinterval=1, % 1 sec refresh
		timeduration=\LectureTime, %Total time, mins
		resetatpages=2, % Start the timer at page 2
		timewarningfirst=85,timewarningsecond=95, % Warnings when approaching the end (%)
		colorwarningfirst=white!60!yellow,
		fillcolorwarningsecond=white!10!yellow]{ %../../common/
		tdclock}
\fi
%	\usepackage{amsthm}
%	\usepackage{mathtools}	% special math environments

\ifx\MayFloat\undefined	% Packet for non-printable forms
\else	% Packets for printable forms
	\usepackage{float}		% For custom floats
	\floatstyle{plaintop}	% Caption is above the content
%	\newfloat{program}{thp}{lop}[chapter]
%	\floatname{program}{\ifx\LectureLanguage\undefined Programlista\else Listing\fi}
	\newfloat{exercise}{thp}{lep}[chapter]
	\floatname{exercise}{\ifx\LectureLanguage\undefined List of exercises\else A gyakorlatok listája \fi}
	\newfloat{solution}{thp}{lsp}[chapter]
	\floatname{solution}{\ifx\LectureLanguage\undefined A megoldások listája\else List of solutions\fi}
\fi

\usepackage{tikz}   
\usepackage{animate}
\usetikzlibrary{
	shapes,
	positioning,
	arrows,
	fit,
	backgrounds,
	calc,
	tikzmark,
	shadows,
	automata}
%%% A useful supplement when some node is defined during compilation,
%%% see %% see http://www.tex.ac.uk/FAQ-isdef.html
%%% http://tex.stackexchange.com/questions/37709/how-can-i-know-if-a-node-is-already-defined
\makeatletter
\long\def\ifnodedefined#1#2#3{%
    \@ifundefined{pgf@sh@ns@#1}{#3}{#2}%
}
\makeatother

	\usepackage{pgfplots}
	\pgfplotsset{compat=1.13} 
	\usepackage{pgfplotstable}

	\usepackage{pdfrender}
	 \usepackage[backend=bibtex]{biblatex}   	% for beamer bibliography
\ifx\DisableIndex\undefined
	\usepackage{makeidx}
	\ifx\LecturePrintable\undefined
	\else
		\usepackage[columns=2]{idxlayout} 
		\makeindex
	\fi
\fi

\ifx\LectureBibliography\undefined
\else
	\usepackage[backend=bibtex]{biblatex} % defined in MEPackages
	\bibliography{\LectureBibliography}
\fi
\MEDebugMessage{MEPackages successfully loaded}
	
	% Use the common packets
% %
% % The macros for the MultEdu system
% %
	\MEDebugMessage{MEMacros: started to load}
        \def\MEVersion{0.3.2}
	\def\MERevision{MultEdu V\MEVersion}

%% A simple dual language text handler
\newcommand{\MEDtext}[2]{ 
{\ifx\UseSecondLanguage\undefined#1\else#2\fi}}

%%% This output goes only to the handout, but not to slides
\newcommand\articleonly[1]{\ifx\LecturePrintable\undefined\else{{\rmfamily #1}}\fi}
\newcommand\ao[1]{\articleonly{#1}}
% put the argument both in text and the 
\newcommand{\indexit}[1]{\index{#1}{\sffamily\color{NavyBlue}#1}}


	% A macro to use ifstrequal 			
%	\newcommand\myifstrequal{\expandafter\ifstrequal\expandafter}
	\makeatletter
	% % % ========= Macros using kexval =========
	% ========= KEY DEFINITIONS =========
	\define@key{MEMacros}{color}[green]{\def\ME@color{#1}}
	\define@key{MEMacros}{decorations}{\def\ME@decorations{#1}}
	\define@key{MEMacros}{iconheight}[10pt]{\def\ME@color{#1}}
	\define@key{MEMacros}{language}{\def\ME@language{#1}}
	\define@key{MEMacros}{number}{\def\ME@number{#1}}
	\define@key{MEMacros}{options}{\def\ME@options{#1}}
	\define@key{MEMacros}{plain}[true]{\def\ME@plain{#1}}		
	\define@key{MEMacros}{resize}[true]{\def\ME@resize{#1}}
	\define@key{MEMacros}{shrink}[true]{\def\ME@shrink{#1}}		
	\define@key{MEMacros}{tall}[true]{\def\ME@tall{#1}}
	\define@key{MEMacros}{title}{\def\ME@title{#1}}
	\define@boolkey{MEMacros}{wide}[true]{\def\ME@wide{#1}}
	\define@key{MEMacros}{width}[3cm]{\def\ME@width{#1}}
	\presetkeys{MEMacros}{color=green}{}%
	\presetkeys{MEMacros}{number=1}{}%
	\presetkeys{MEMacros}{plain=false}{}%
	\presetkeys{MEMacros}{resize=false}{}% 
	\presetkeys{MEMacros}{shrink=false}{}%
	\presetkeys{MEMacros}{tall=false}{}% 
	\presetkeys{MEMacros}{wide=false}{}%
	\presetkeys{MEMacros}{width=3cm}{}%
\makeatother

			
%%%%
%%% These macros allow conditional printing of ListOfXXX
%%%%
\makeatletter
\AtEndEnvironment{figure}{\gdef\there@is@a@figure{}} 
\AtEndEnvironment{table}{\gdef\there@is@a@table{}} 
\AtEndEnvironment{program}{\gdef\there@is@a@program{}} 
\AtEndDocument{
\ \par
	\ifdefined\there@is@a@figure\label{fig:was:used:in:doc}\fi
	\ifdefined\there@is@a@table\label{tab:was:used:in:doc}\fi
	\ifdefined\there@is@a@program\label{prog:was:used:in:doc}\fi
	} 
\newcommand{\conditionalLoF}{\@ifundefined{r@fig:was:used:in:doc}{}{\clearpage\listoffigures}}%
\newcommand{\conditionalLoT}{\@ifundefined{r@tab:was:used:in:doc}{}{\clearpage\listoftables}}%
\newcommand{\conditionalLoP}{\@ifundefined{r@prog:was:used:in:doc}{}{\clearpage\lstlistoflistings}}%
\makeatother

%%%
%%% General purpose macros
%%%
%%% Make the text argument highlighted
\makeatletter
\newcommand{\highlighted}[2][]{%
	\setkeys{MEMacros}{color=darkmagenta!40!black!80,#1}% 
	{\textit{\textbf
			{\textcolor{\expandafter\ME@color\expandafter}{#2}}}}%
}
\makeatother % Needed to terminate keyval-related macros

% Used to mark part of text, as 'teletype text'
\newcommand{\ttext}[1]{{\bfseries\sffamily #1}}
% Used to mark part of text as 'computer text'
\newcommand{\ctext}[1]{{\ttfamily\bfseries
		 #1}}


%http://tex.stackexchange.com/questions/167708/tcolorbox-spanning-two-columns-in-paracol-environment
\ifx\LecturePrintable\undefined
	\NewTColorBox{NoteBox}{ s O{!htbp} }{%
	colframe=HeadingColor,colback=HeadingColor!10!white,% any tcolorbox options here
}
\else
\NewTColorBox{NoteBox}{ s O{!htbp} }{%
	floatplacement={#2},
	IfBooleanTF={#1}{float*,width=\textwidth}{float},
	colframe=HeadingColor,colback=ivory,% any tcolorbox options here
}
\fi
% % %?? http://tex.stackexchange.com/questions/199659/text-with-semi-transparent-color-filled-box

%% Make a notice entry
%% Usage: \MEnote[kv]{title}{explanation}
\makeatletter
\newcommand\MEnote[3][]{
	\setkeys{MEMacros}{wide=false,title={},#1}% 
	\ifKV@MEMacros@wide
	
		\begin{NoteBox}*[!htbp]
					\maxsizebox{\textwidth}{.5\textheight}
					{
						\begin{minipage}{\textwidth}
							\ttext{\large #2}\par
							#3
						\end{minipage}
					}
		\end{NoteBox}
	\else
		\begin{NoteBox}[!htbp]
			\maxsizebox{\columnwidth}{.5\textheight}
			{
				\begin{minipage}{\columnwidth}
					\ttext{\large #2}\par
					#3
				\end{minipage}
			}
		\end{NoteBox}
	\fi
}
\makeatother


%% Make a notice entry
%% Usage: \MEquote[kv]{saying}{author}
\makeatletter
\newcommand\MEquote[3][]{
	\setkeys{MEMacros}{wide=false,#1}% 
	\ifKV@MEMacros@wide
	
		\begin{NoteBox}*[!htbp]
					\maxsizebox{\textwidth}{.5\textheight}
					{
					\ttext{ #2}\par
					\hfill #3
					}
		\end{NoteBox}
	\else
		\begin{NoteBox}[!htbp]
			\maxsizebox{\columnwidth}{.5\textheight}
			{
				\begin{minipage}{\columnwidth}
					\ttext{ #2}\par
					\hfill #3
				\end{minipage}
			}
		\end{NoteBox}
	\fi
}
\makeatother


\newtcolorbox{mybox}[1][]{
%	\setkeys{MEMacros}{width=.9\textwidth,#1}%  
    ,arc=3mm,
width=.9\textwidth,%    auto outer arc,
    boxsep=0cm,
    toprule=1pt,
    leftrule=1pt,
    bottomrule=1pt,
    rightrule=1pt,
    colframe=\TitlePageColor,
    fontupper=\raggedleft\fontsize{16pt}{14pt}\itshape,
    breakable,
    nobeforeafter,
    enhanced jigsaw,
    opacityframe=0.3,
    opacityback=0.7
}



%% 
%% The document sectioning related macros for the MultEdu system
%%
%% The basic building block is the frame, because 'beamer' requires it

%%% A simple single-language frame for avoiding unnecessary typing
%%% Usage: \MEframe[keys and values]{frame subtitle}{content of the intended frame}
%% % Possible keys: shrink
%% %                         plain
%%% Actually, this is two implementations: one for the printed formats and one for the slides
	\MEDebugMessage{MESections: started to load}
\makeatletter
\ifx\LecturePrintable\undefined	% Making slides
\newcommand{\MEframe}[3][]
{
	\MEframelabel{#2}
	\setkeys{MEMacros}{shrink=true, plain=false, #1}% 
%	\begingroup
		\expandafter\ifstrequal\expandafter{\ME@shrink}{true}
			{\def\myshrink{yes}}{}
		\expandafter\ifstrequal\expandafter{\ME@plain}{true}
			{\def\myplain{yes}}{}
			\ifx\LecturePrintable\undefined	% Making slides
						\begin{frame}[shrink] {\framelabel}{}
	%					Just normal\par
						#3 %\par\phantom{-}
						\end{frame}
%				\ifx\myshrink\undefined
%					\ifx\myplain\undefined
%						%\begin{frame} {#2}{\framelabel}
%						\begin{frame} {MyTitle}{MySubtitle}
%	%					Just normal\par
%						#3\par\phantom{-}
%						\end{frame}
%					\else	%  plain
%						\begin{frame}[plain]
%	%					plain 
%						#3\par\phantom{-}
%						\end{frame}
%					\fi
%				\else % Anyhow, shrink defined
%					\ifx\myplain\undefined
%						\begin{frame}[shrink]  {#2}{\framelabel}
%	%					shrink\par
%						#3\par\phantom{-}
%						\end{frame}
%					\else	% shrink AND plain
%						\begin{frame}[shrink,plain] 
%	%					plain shrink
%						#3\par\phantom{-}
%						\end{frame}
%					\fi %  shrink and plain
%				\fi %shrink
			\else	% Making book
			\begin{frame}%
						#3
			\end{frame}%
			\fi
%	\endgroup%
	
	
	%	\begingroup%
	%	\ifx\LecturePrintable\undefined
	%		\expandafter\ifstrequal\expandafter{\ME@plain}{true}
	%			{ % It is a plain frame
	%				\expandafter\ifstrequal\expandafter{\ME@shrink}{true}
	%				{\begin{frame}[plain,shrink] {#2}{\framelabel}
	%			%	it was PLAIN  and SHRINK
	%				}
	%				{\begin{frame}[plain]{#2}{\framelabel}
	%			%	It was just PLAIN
	%				}
	%		  	}
	%		  	{ % It is not plain
	%				\expandafter\ifstrequal\expandafter{\ME@shrink}{true}
	%				{\begin{frame}[shrink]{#2}{\framelabel}
	%			%	It was ONLY shrink
	%				}
	%			  	 {\begin{frame}{#2}{\framelabel}
	%			  	% It was just empty
	%			  	 }
	%		  	}
	%		 \else
	%		 \fi
	%	\endgroup%
	%\par The key was: #1\par
	%\begin{frame}{#2}{\framelabel}
	%#3
	%\ifx\LecturePrintable\undefined\phantom{-}\fi
	%\end{frame}
} % \MEframe
\else % Making some printed version
	\newcommand{\MEframe}[3][]
	{
		\begin{frame}%{}{}
			#3
		\end{frame}
	}
\fi % \LecturePrintable

%% A dual-language frame for avoiding unnecessary typing
%% Usage: \MEDframe[keys and values]
%{frame subtitle, \LectureLanguage}{content }
%{frame subtitle, \SecondLanguage}{content}
% % Possible keys: shrink
% %                         plain

\newcommand{\MEDframe}[5][]{
	\ifx\UseSecondLanguage\undefined
	    \MEframe[{#1}]{#2}{#3}%
	\else
		\MEframe[{#1}]{#4}{#5}%
	\fi
}

%%%% The base input is formatted as 'beamer' frames
%%%% Frame specific definitions
%%%%
%% Define frame subtitle 
\ifx\LecturePrintable\undefined
%% Implementation for the slides
\newcommand{\MEframelabel}[1]{
	\def\framelabel{#1}
}
\else
%% Implementation for the printed versions
\newcommand{\MEframelabel}[1]{
	\def\framelabel{#1}
	\ifx\undefined\eBook\else \markright{#1}\fi
	\ifx\FancyBook\undefined\else\markright{#1}\fi
}
\fi

%%% Sectioning macros 
%%% except A4 size, limit float range and start new page
%%%
\newcommand{\WEBpagebreak}{%
	\ifx\LecturePrintable\undefined
	\else
		\ifx\undefined\eBook\else\FloatBarrier\clearpage\fi
		\ifx\undefined\WEBBook\else\FloatBarrier\clearpage\fi
	\fi
}

%%% Use one level less for the slides, so 'book' format can be used
%%% Abandon error message for 'chapter' keyword in slides 
%%% These macros use 'book' terminology, and transform it properly for slides
%%% Usage: \MEchapter{short title}{long title}
%%%
\newcommand{\MEchapter}[2][]{
	\WEBpagebreak	% Start on new page for browsing
	\phantomsection	% for hyperreferencing
%	\ifx\LecturePrintable\undefined\fi
	\ifx\LecturePrintable\undefined % Making slides
		\MEDebugMessage{MESections: starting section}
			\MEframelabel{#2}%\section[#1]{#2}
			\ifthenelse{\isempty{#1}}{\section{#2}}{\section[#1]{#2}}		
	\else %Printing book
		\MEDebugMessage{MESections: starting chapter}
		\chapter[#1]{#2}\MEframelabel{#2}
	\fi
%	\ifx\LecturePrintable\undefined\else\MEframelabel{#2}\fi
}

%%% Abandon extra typing for 'section' keyword
%%% Usage: \MEsection{short title}{long title}
%%%
\newcommand{\MEsection}[2][]{
	
	\WEBpagebreak	% Start of new browser page
	\phantomsection % for hyperreferencing

	% Check whether the short title is empty
		\ifx\LecturePrintable\undefined % Making slides
			\ifthenelse{\isempty{#1}}{\subsection{#2}}{\subsection[#1]{#2}}
		\else %Printing book
			\ifthenelse{\isempty{#1}}{\section{#2}}{\section[#1]{#2}}
		\fi
	\MEframelabel{#2}
}


%% Abandone extra typing for 'subsection' keyword
%%% Usage: \MEsubsection{short title}{long title}
\newcommand{\MEsubsection}[2][]{
%	\WEBpagebreak	% Start of new browser page, usually not needed
	\phantomsection
	\ifx\LecturePrintable\undefined % Making slides
			\ifthenelse{\isempty{#1}}{\subsubsection{#2}}{\subsubsection[#1]{#2}}
	\else %Printing book
		\ifthenelse{\isempty{#1}}{\subsection{#2}}{\subsection[#1]{#2}}
	\fi
}

%%% Usage: \MEsubsubsection{short title}{long title}
\newcommand{\MEsubsubsection}[2][]{
	\phantomsection % A hook for hyper references
	\ifx\LecturePrintable\undefined % Making slides
			%\ifthenelse{\isempty{#1}}{}{
			\par\noindent\bfseries #2\par%}
	\else %Printing book
		\ifthenelse{\isempty{#1}}{\subsubsection{#2}}{\subsubsection[#1]{#2}}
	\fi %Printable
}

% Abandon error message for 'chapter' keyword in slides, 2 languages
\newcommand{\MEDchapter}[4][]
{
	\ifx\UseSecondLanguage\undefined
		\MEchapter[#1]{#2}
	\else
		\MEchapter[#3]{#4}
	\fi
}
 
%%% Abandon extra typing for 'section' keyword, 2 languages
\newcommand{\MEDsection}[4][]
{
	\ifx\UseSecondLanguage\undefined
		\MEsection[#1]{#2}
	\else
		\MEsection[#3]{#4}
	\fi
}

%% Abandon extra typing for 'subsection' keyword, 2 languages
\newcommand{\MEDsubsection}[4][]
{
	\ifx\UseSecondLanguage\undefined
		\MEsubsection[#1]{#2}
	\else
		\MEsubsection[#3]{#4}
	\fi
}

%% Abandone extra typing for 'subsubsection' keyword, 2 languages
\newcommand{\MEDsubsubsection}[4][]
{
	\ifx\UseSecondLanguage\undefined
		\MEsubsubsection[#1]{#2}
	\else
		\MEsubsubsection[#3]{#4}
	\fi
}

\def\Chapterillustration{}% Define it as empty
%% Give the new illustration of chapter in memoir/fancy mode
\newcommand\MEchapterillustration[1]
{
	\ifx\DisableChapterIllustration\undefined
		\ifthenelse{\isempty{#1}}
		{	
			\IfGraphicFileExists{fig/DefaultIllustration}
				{\def\Chapterillustration{fig/DefaultIllustration}}
				{}	
		}			
		{	% The argument is not empty
		\IfGraphicFileExists{#1}
			{\def\Chapterillustration{#1}}	% Explicit file present, use that file
			{\DebugMessage{File '#1' not found}
				}
		}
				
		\ifthenelse{\isempty{\Chapterillustration}}
		{% No file was found, exit	
		}
		{ % File found
		\ifx\LecturePrintable\undefined
		%% We are making slides, will be done in \AtBeginSection
		\else
			\ifx\FancyBook\undefined
				\par\noindent\includegraphics[width=\paperwidth]{\ChapterIllustration}
			\else
				\renewcommand\chapterillustration{\Chapterillustration}
			\fi
		\fi
		}
%	\ifx\MEchapterillustration\undefined
%			% No file was found, exit
%	\else
%		\ifx\LecturePrintable\undefined
%%			\def\framelabel{}
%%			\MEframe[plain]{}
%%				{
%%					\includegraphics[width=\textwidth]{\myfile}
%%				}
%		\else % We are printing a book
%			\ifx\FancyBook\undefined
%				\par\noindent\includegraphics[width=\paperwidth]{\ChapterIllustration}
%			\else
%				\renewcommand\chapterillustration{\Chapterillustration}
%			\fi
%		\fi
%	\fi% \myfile
	\fi%\DisableChapterIllustration
}%\MEchapterillustration

	\MEDebugMessage{MESections: finished to load}

%%
%% The program-listings related macros for the MultEdu system
%%
% % It uses some definitions from MEMacros, so it should be after that file
% %
\ifx\MayFloat\undefined
\else
	\floatstyle{plaintop}
	\newfloat{program}{thp}{lpp}[chapter]
	\ifx\LectureLanguage\undefined
	\floatname{program}{Listing}
	\else
	  \ifthenelse{\equal{\LectureLanguage}{magyar}}
	   {\floatname{program}{Programlista}
	   \MEDebugMessage{MEListings: Changed name to Programlista by \LectureLanguage}
	   }
	   {	   \MEDebugMessage{MEListings:  name not changed to Programlista by \LectureLanguage}
	   }%false
	\fi
\fi

\ifx\eBook\undefined
	\def\lstsize{\scriptsize}
\else
	\def\lstsize{\tiny}
\fi

%% Set parameters for the appearance of the listings

\newcommand{\MESetStandardListingFormat}
{
	\lstset{
	    literate=% allow Hungarian umlauts
         {á}{{\'}a}1
         {Á}{{\'}Á}1
         {é}{{\'}e}1
         {É}{{\'}E}1
         {í}{{\'i}}1
         {Í}{{\'I}}1
         {ó}{{\'}o}1
         {Ó}{{\'}O}1
         {ú}{{\'u}}1
         {Ú}{{\'}u}1
         {ö}{{\"o}}1
         {Ö}{{\"}Ö}1
         {ü}{{\"}u}1
         {Ü}{{\"}U}1
	    {ő}{{\H{}o}}1
	    {ö}{{\H{}O}}1
	    {ű}{{\H{}U}}1
	    {Ű}{{\H{}u}}1
	    	}	    
	\lstset{
		%         numbers=left,               % Place for line numbers
		numberstyle=\tiny,          % Style of line numbers
		numbersep=5pt,              % Distance of line numbers from source
		tabsize=2,                 	 % Size of Tabs
		inputencoding=utf8/latin2,	% input encoding, allow accented chars
		extendedchars=true,         %
		escapechar=\@,
		breaklines=true,        % sets automatic line breaking
		columns=fullflexible,  % eliminates the spacing:
		breakatwhitespace=true,    % sets if automatic breaks should only happen at whitespace
        keepspaces=true,
        escapeinside={\%*}{*)},          % if you want to add a comment within your code
		frame=tb, 
		framerule=.5pt, 
		rulecolor= \color{SeparatorColor},
		backgroundcolor=\color{ivory},
%		basicstyle=\ttfamily\color{black}\lstsize\bfseries, 
		basicstyle=\ttfamily\color{black}\normalsize\bfseries, 
		keywordstyle=\bfseries\color{darkmagenta},
		identifierstyle=\bfseries\color{NavyBlue},
		commentstyle=\itshape\bfseries\color{ForestGreen},
		stringstyle=\itshape\bfseries\color{burntorange}, % Color for strings
		lineskip=0pt,aboveskip=4pt,belowskip=2pt,
		framesep=4pt,rulesep=2pt, %framerule=.25pt,
		showspaces=false,           % 
		showtabs=false,             % 
		framexleftmargin=0pt,
		framexrightmargin=0pt,
		showstringspaces=false      % Show empty spaces? 	
	}	%Take optional arguments
}%\MESetStandardListingFormat

\newcommand\MESetListingFormat[2][]	% Optional is the extra arguments
{
    \MESetStandardListingFormat
    \lstset
    	{
%		#1,
		language={#2},		% Select the language	
        numberstyle=\tiny,          % Style of line numbers
        numbersep=5pt,              % Distance of line numbers from source
        tabsize=2,                 	 % Size of Tabs
        inputencoding=utf8/latin2,	% input encoding, allow Hungarian umlauts
        extendedchars=true,         %
        escapechar=\@,
		breaklines=true,        % sets automatic line breaking
		breakatwhitespace=true,    % sets if automatic breaks should only happen at whitespace
		escapeinside={\%*}{*)},          % if you want to add a comment within your code
		frame=tb, 
		framerule=.5pt, 
		rulecolor= \color{SeparatorColor},
		backgroundcolor=\color{ivory},
        basicstyle=\ttfamily\color{black}\lstsize , 
        keywordstyle=\bfseries\color{magenta},
        identifierstyle=\bfseries\color{NavyBlue},
 		commentstyle=\itshape\bfseries\color{ForestGreen},
        stringstyle=\itshape\bfseries\color{burntorange}, % Color for strings
        lineskip=0pt,aboveskip=4pt,belowskip=2pt,
        framesep=4pt,rulesep=2pt, %framerule=.25pt,
        showspaces=false,           % 
        showtabs=false,             % 
        xleftmargin=-1pt,
        framexbottommargin=4pt,
        framextopmargin=4pt,
        gobble=5,
		framexleftmargin=0pt,
		framexrightmargin=0pt,
         showstringspaces=false      % Show empty spaces? 	
	}	%Take optional arguments
	\lstset{#1}
}%MESetListingFormat

\newlength{\FigWidth}\newlength{\FigHeight}
%\makeatletter
%%% Insert a source file in the text, with optional decorations
%%%%Usage \MESourceFile[keys]{source file}{caption}{label}{ScaleFactor}
%\newcommand\MESourceFile[5][]{
%	\setkeys{MEMacros}{wide=false,language={[ANSI]C},options={}, decorations={},#1}% 
%	% Define the environment: a 'Program' if it might float, a simple caption if not
%	\MESetStandardListingFormat
%	\ifx\MayFloat\undefined % A kind of slides
%		\def\startsource{ \begin{center}
%			              \setlength{\FigWidth}{#5\textwidth}
%		                  {\color{HeadingColor}\bfseries\scriptsize #3}
%			              \par\vskip-\baselineskip
%		                }
%		\def\stopsource{\end{center}}
%	\else %% Some printable form, either A4 book,  WEB book or eBook
%		\ifKV@MEMacros@wide %% It is a wide floating version
%		   \def\startsource{ \begin{center}
%			                \setlength{\FigWidth}{#5\textwidth} 
%			                \begin{program*}[!hbt] 
%			               }
%			\def\stopsource{\end{program*}\end{center}}			
%		\else % It is the narrow (one-column) version
%			\if@twocolumn
%			  \setlength{\FigWidth}{#5\columnwidth}
%			\else
%			  \setlength{\FigWidth}{.7\textwidth}
%			  \setlength{\FigWidth}{#5\FigWidth}
%			 \fi
%			\def\startsource{\begin{center}\begin{program}[!hbt]}
%			\def\stopsource{\end{program}\end{center}}
%		\fi
%	\fi
%	%% Start printing the figure here:
%	\setlength{\FigHeight}{.8\textheight} 
%	\noindent\startsource %{\caption{#3}} \hskip-1em
%%	\maxsizebox{#5\FigWidth}{#5\FigHeight}
%	{
%		\mbox{}\phantomsection
%		\noindent\begingroup\protected@edef\x{\endgroup\noexpand
%			\lstinputlisting[language={\ME@language}, \ME@options, label=#4, name=#4,
%			\ifx\LecturePrintable\undefined\else	caption ={#3}\fi
%			]{#2}}
%		\x
%		\ME@decorations % Decorating comments
%	}					
%	\stopsource
%}
%\makeatother
%\newlength{\FigWidth}
%\usepackage{layouts}
\makeatletter
%% Insert a source file in the text, with optional decorations
%%%Usage \MESourceFile[keys]{source file}{caption}{label}
\newcommand\MESourceFile[4][]{
	\setkeys{MEMacros}{wide=false,language={[ANSI]C},options={}, decorations={},#1}% 
	% % Define the environment: a 'Program' if it might float, a simple caption if not
	\MESetStandardListingFormat
	\ifx\MayFloat\undefined % A kind of slides
		\def\startsource{
			\setlength{\FigWidth}{\textwidth}
	%	\vskip.2\baselineskip
		{\color{HeadingColor}%\bfseries\scriptsize #3
			}
			\par%\vskip\baselineskip
		}
		\def\stopsource{}
	\else %% Either A4 book,  WEB book or eBook
		\ifKV@MEMacros@wide %% It is a wide floating version
		   \def\startsource{
			\setlength{\FigWidth}{\textwidth} 
			\begin{program*}[!hbt] 
			}
			\def\stopsource{\end{program*}}			
		\else % It is the narrow (one-column} version)
			\if@twocolumn
			  \setlength{\FigWidth}{\columnwidth}
			\else
			  \setlength{\FigWidth}{.7\textwidth}
			 \fi
			\def\startsource{\begin{program}[!hbt]}
			\def\stopsource{\end{program}}
		\fi
	\fi
	%% Start printing the figure here:
	\noindent\startsource\hskip-1em
	\vspace{-10pt}
%	\maxsizebox{\FigWidth}{.5\textheight}
	{
		\mbox{}\phantomsection
%		\ifx\MayFloat\undefined\else\caption{#3}\fi %\vglue-.7\baselineskip
		\noindent\begingroup\protected@edef\x{\endgroup\noexpand
			\lstinputlisting[language={\ME@language}, \ME@options, label=#4, name=#4,
			caption ={#3}
			]{#2}}
		\x
		\ME@decorations % Decorating comments
	}					
%	\vspace{-5pt}					
	\stopsource
}
\makeatletter  % This is the good version

%\newlength{\mywidth} % Used in defining listing width
% http://mirror.unl.edu/ctan/graphics/pgf/contrib/tikzmark/tikzmark.pdf
% % Prepare for some decorations on the listing files
\makeatletter
\newif\iflst@linemark
	
\lst@AddToHook{EveryLine}{%
 \begingroup
 \advance\c@lstnumber by 1\relax
 \pgfmark{line-\lst@name-\the\c@lstnumber-start}%
 \endgroup
}
	
\lst@AddToHook{EOL}{\pgfmark{line-\lst@name-\the\c@lstnumber-end}%
	\global\lst@linemarktrue
}
	
\lst@AddToHook{OutputBox}{%
 \iflst@linemark
 \pgfmark{line-\lst@name-\the\c@lstnumber-first}%
 \global\lst@linemarkfalse
 \fi
}
	
\def\tkzlst@fnum#1\relax#2\@STOP{%
 \def\@test{#2}%
 \ifx\@test\@empty
 \def\tkzlst@start{0}%
 \else
 \@tempcnta=#1\relax
 \advance\@tempcnta by -1\relax
 \def\tkzlst@start{\the\@tempcnta}%
 \fi
}
	
\lst@AddToHook{Init}{%
 \expandafter\tkzlst@fnum\lst@firstnumber\relax\@STOP
 \pgfmark{line-\lst@name-\tkzlst@start-start}%
}

% % Put a balloon around some lines in a source
% Usage: \MESourcelinesHighlight{BallonName}{SourceName}{FirstLine}{LastLine}
\newcommand\MESourcelinesHighlight[4]{%
  \pgfmathtruncatemacro\pgf@temp{%
   #3-1
  }%
  \iftikzmark{line-#2-\pgf@temp-start}{%
   \iftikzmark{line-#2-#3-first}{%
     \xdef\b@lines{({pic cs:line-#2-\pgf@temp-start} -| {pic cs:line-#2-#3-first})}%
   }{%
     \iftikzmark{line-#2-#3-start}{%
       \xdef\b@lines{({pic cs:line-#2-\pgf@temp-start} -| {pic cs:line-#2-#3-start})}%
     }{%
       \xdef\b@lines{(pic cs:line-#2-\pgf@temp-start)}%
     }%
   }%
  }{%
   \xdef\b@lines{}%
  }%
  \foreach \k in {#3,...,#4} {%
   \iftikzmark{line-#2-\k-first}{%
     \xdef\b@lines{\b@lines (pic cs:line-#2-\k-first) }
   }{}
   \iftikzmark{line-#2-\k-end}{%
     \xdef\b@lines{\b@lines (pic cs:line-#2-\k-end) }
   }{}
  }%
  \ifx\b@lines\pgfutil@empty
  \else
  \edef\pgf@temp{\noexpand\tikz[remember picture,overlay]\noexpand\node[fit={\b@lines}, color=ForestGreen,yshift=-2pt,
     draw, fill=green!30, opacity=0.4,  inner sep=0pt, rounded corners=2pt] (#1) {};
  }%
		\pgf@temp
  \fi
		}
\makeatother

% Prepare a comment relating to a balloon as rounded rectangle 
% Usage \MESourceBalloonComment[keys]{Balloon}{ShiftPosition}{Comment}{CommentShape}
% Keys: width[=3cm]
% 		color[=deeppeach]
\makeatletter
% ========= KEY DEFINITIONS =========
\newcommand\MESourceBalloonComment[5][]
{	% First draw the text box
	\setkeys{MEMacros}{width=3cm,#1}% 
	\setkeys{MEMacros}{color=deeppeach,#1}% 
   \begingroup%
  \tikz[remember picture,overlay]
  \node[rectangle, draw,  rounded corners,drop shadow,  align=left,  fill=\ME@color, text width=\ME@width, font=\lstsize] %\bfseries]
 at  ($(#2.east)+(\ME@width,0) +(#3)$) (#5) {\begin{minipage}{\ME@width}\lstsize#4\end{minipage}} ;
   % Now draw the connecting arrow
   \tikz[remember picture,overlay] 
   \draw[->,thick,color=burntorange] 
    (#5.west) --  ($ (#5.west) + (-.2cm,0) $)  |- (#2.east)  ;
	\endgroup
}
\makeatother

\makeatletter
% % Put numbered balls after the line 'Lineno'  in source 'Source'
% Usage: \MESourcelineListBalls[keys]{ListingLabel}{List of lines}
% Possible keys: color	% Color of the balls
%                        number	% Starting seq number
\newcounter{qan}\newcounter{qano}
\newcommand\MESourcelineListBalls[3][]{%
	\setkeys{MEMacros}{color=orange,#1}% 
	\setkeys{MEMacros}{number=1,#1}% 
\setcounter{qan}{\ME@number-1}
\setcounter{qano}{0}
   \begingroup%
    \foreach \x in {#3}
    {  \addtocounter{qan}{1}
    	\addtocounter{qano}{1}
      \only<\arabic{qano}>%
  {\tikz[remember picture,overlay]
    {\expandafter\node[circle, inner sep=2pt, draw,fill=\ME@color,ball color=\ME@color, shading=ball, font=\scriptsize\bfseries, drop shadow]
   at  ([xshift=+10pt,yshift=+2pt]{pic cs:line-#2-\x-end}) {\lstsize\arabic{qan}};\expandafter}}
   }
	\endgroup
}
\makeatother

%% Put a numbered ball in the text, to reference balls on a listing
%% Usage: \MEBall{ListingLabel}{Number}
\newcommand{\MEBall}[2]{%
{\tiny#1}~\hskip-4pt\raisebox{-.08cm}{
\tikz \node (1ex,1ex)
 [circle,draw,ball color=green, shading=ball,  font=\bfseries, scale=0.55] {#2};}%
~\hskip-4pt}% MEBall


\makeatletter
% Prepare a source comment relating to a source line
% Usage \MESourcelineComment[keys]{Sourcelabel}{LineNo}{ShiftPosition}{Comment}{CommentShape}
% Keys: width[=3cm]
% ========= KEY DEFINITIONS =========
%	\newlength{\commentboxlength}
\newcommand\MESourcelineComment[6][]
{
	\setkeys{MEMacros}{width=3cm,#1}% 
	\setkeys{MEMacros}{color=deeppeach,#1}% 
%	\setlength{\commentboxlength}{\ME@width}
   \begingroup%
		  \tikz[remember picture,overlay]
		  \node[rectangle, draw,  rounded corners,drop shadow,  align=left,
				fill=\ME@color, text width=\ME@width, font=\lstsize]
		   at  ($(#4) +(\ME@width,0) +({pic cs:line-#2-#3-end})$) (#6) {\begin{minipage}{\ME@width}\lstsize #5 \end{minipage}} ;
%		\tikz[remember picture,overlay] \draw[->,thick,color=burntorange]  (#6.south) |- ({pic cs:line-#2-#3-end});
		\tikz[remember picture,overlay] \draw[->,thick,color=burntorange]  (#6.west)  --  ($ (#6.west) + (-.2cm,0) $) |- ({pic cs:line-#2-#3-end});
	\endgroup
}
\makeatother


\makeatletter
% Prepare a figure comment relating to a source text
% Usage \MESourcelineFigure[keys]{SourceName}{LineNo}{ShiftPosition}{Graphics file}
% Keys: width[=3cm]
% ========= KEY DEFINITIONS =========
\define@key{MESourcelineFigure}{width}{\def\pb@width{#1}}
\newcommand\MESourcelineFigure[5][]
{
	\setkeys{MESourcelineFigure}{width=3cm,#1}% 
	\begingroup%
	\tikz[remember picture,overlay]
	\node[ rectangle,  draw,  rounded corners, drop shadow]
	at  ($(#4)+({pic cs:line-#2-#3-end})$)  {\includegraphics[width=\pb@width]{#5}} ;
	\endgroup
}
\makeatother

\newlength{\mywidth} % Used in defining listing width
%% Prepare a figure from comparing the source files provided; 
%%Usage \MESourceFileCompare[keys]{source file1}{source file2}{caption}{label}
\makeatletter
% ========= KEY DEFAULTS =========
\newcommand\MESourceFileCompare[5][]{
\setkeys{MEMacros}{language={[ANSI]C},options={},#1}% 
	  \begingroup%
		% Set up the listing format, pass language and extra options
	\begingroup\edef\x{\endgroup\noexpand\MESetListingFormat[\ME@options]{\ME@language}}\x
			\if@twocolumn%% %	\makeatletter%
					 \setlength\mywidth{\columnwidth}
				\else% \@twocolumnfalse
					 \setlength\mywidth{.46\textwidth}
				\fi
		% % Define the environment: a 'Program' if it might float, a simple caption if not
		\begin{center}
		\ifx\MayFloat\undefined
			{\color{HeadingColor}\bfseries\scriptsize #4}
		\else
	  	    		{
	  	    		\begin{program*}[h!btp]	\caption{#4}\par\vskip-\baselineskip
	  	    		 }
		\fi\mbox{}\phantomsection
		\vskip-1\baselineskip
		\begin{tabular}{p{.48\mywidth}p{.48\mywidth}}
		\lstinputlisting[label=#5, name=#5, linewidth=\mywidth ] {#2}&
		\lstinputlisting[label=#5, name=#5,linewidth=\mywidth ] {#3}\\
		\end{tabular}
	\ifx\MayFloat\undefined	% nothing needed
	\else
		{	
		\end{program*}\vskip-\baselineskip}
	\fi
	\end{center}
  \endgroup%
}

%%Usage \MESourceFileWithResult[keys]{source file}{result file}{caption}{label}
\newcommand\MESourceFileWithResult[5][]{
	\setkeys{MEMacros}{language={[ANSI]C},options={},#1}% 
	\begingroup%
	% Set up the listing format, pass language and extra options
	\begingroup\edef\x{\endgroup\noexpand\MESetListingFormat[\ME@options]{\ME@language}}\x
	\if@twocolumn%
	\setlength\mywidth{\columnwidth}
	\else% \@twocolumnfalse
	\setlength\mywidth{.46\textwidth}
	\ifx\eBook\undefined \lstset{basicstyle=\tiny} \fi
	\fi
	% % Define the environment: a 'Program' if it might float, a simple caption if not
	\ifx\MayFloat\undefined
	{\color{HeadingColor}\bfseries\scriptsize #4}
	\else
	{
		\begin{program*}[h!btp]	\caption{#4}\par\ifx\WEBBook\undefined\vskip-\baselineskip\else\fi
		}
		\vglue-1.7\baselineskip
		\fi\mbox{}\phantomsection
		\vskip-.6\baselineskip\noindent
		\begin{tabular}{p{.48\mywidth}p{.48\mywidth}}
		%		\ifx\MayFloat\undefined\else\caption{#3}\fi %\vglue-.7\baselineskip
		\noindent\begingroup\protected@edef\x{\endgroup\noexpand
			\lstinputlisting[language={\ME@language}, \ME@options, label=#5, name=#5, linewidth=\mywidth,
			]{#2}}
		\x
		\ME@decorations % Decorating comments
			&
			\lstset{language={bash},backgroundcolor=\color{gray!20}}				
			\lstinputlisting[ linewidth=\mywidth,
				 ]
				 {#3}\\
		\end{tabular}
		\par\vskip-.6\baselineskip
		\ifx\MayFloat\undefined	% nothing needed
		\else
		{	
			\caption{#4}\label{#5}
		\end{program*}}
		\fi
		\endgroup%
	}
\makeatother

% % Define new "language" for diff viewers
\lstdefinelanguage{diff}{
  morecomment=[f][\color{blue}]{@@},     % group identifier
  morecomment=[f][\color{red}]-,         % deleted lines 
  morecomment=[f][\color{green}]+,       % added lines
  morecomment=[f][\color{magenta}]{---}, % Diff header lines (must appear after +,-)
  morecomment=[f][\color{magenta}]{+++},
}
%% Define the Do It Yourself calculator language
%%
\lstdefinelanguage{[DIY]Assembler}
{morekeywords={NOP, HALT, SETIM, CLRIM, ADD, ADDC, SUB, SUBC, AND, OR, XOR, CMPA,
							SHL, SHR, ROLC, RORC, INCA, DECA, INCX, DECX, LDA, STA,
							BLDX, BLDSP, BSTSP, BLDIV, PUSHA, POPA, PUSHSR, POPSR, JMP, JSR,
							JZ, JNZ, JN, JNN, LC, JNC, JO, JNO, RTS, RTI},
%directives={.ORG,.END,.BYTE},
sensitive=false,
morecomment=[l]{\#},
}

%% Define the ARM processor mnemonics
%%
\lstdefinelanguage{[ARM]Assembler}
{morekeywords={
	ADC, ADD, AND, ADR,
							B, BEQ, BNE, BCS, BHS, BCC, BLO, BMI, BPL, BVS, BVC,
			BIC, BL, BX, CMN, CMP, EOR, LDM, LDR, 
								MLA, MOV, MRS, MSR, MOV, MVN, ORR, RSB, RSC, SBC,
								STM, STR, SUB, SUBS,
								SWI, SWP, TEQ, TST},
%directives={AREA,CODE,READONLY,ENTRY,END},
sensitive=false,
morecomment=[l]{;},
}

% % Define x86-64 mnemonics
\lstdefinelanguage
   [x64]{Assembler}     % add a "x64" dialect of Assembler
   [x86masm]{Assembler} % based on the "x86masm" dialect
   % with these extra keywords:
   {morekeywords={CDQE,CQO,CMPSQ,CMPXCHG16B,JRCXZ,LODSQ,MOVSXD, %
                  POPFQ,PUSHFQ,SCASQ,STOSQ,IRETQ,RDTSCP,SWAPGS, %
                  rax,rdx,rcx,rbx,rsi,rdi,rsp,rbp, %
                  r8,r8d,r8w,r8b,r9,r9d,r9w,r9b}} % etc.


% % Define x86-64 mnemonics
\lstdefinelanguage
   [y86]{Assembler}     % add a "Y86" dialect of Assembler
   [x86masm]{Assembler} % based on the "x86masm" dialect
   % with these extra keywords:
   {morekeywords={ halt, nop, movl, rrmovl, irmovl, rmmovl, mrmovl, rrmovl, jump,%
                  xorl, andl, je, addl, jne, subl, pushl, popl, jXX,%
                  cmovXX, cmovl, cmovle, cmove, cmovne, cmovge, cmovg, return,%
                   rax,rbx,rcx,rdx,xmm0,rbp, load,mulss,%
                   OPl, ZF, SF, OF,%
                   eno, ecc, esv,
                   QCreate, QTCreate, QFCreate, QTerm, QCall, QAlloc,
                   QWait, QPWait, QIWait, QInt
                  },
      morecomment=[l]{\#},
      sensitive=false,
%     directives={.pos, .align, .long,}
     } % etc.                 


%%http://tex.stackexchange.com/questions/235161/figures-with-nice-caption-style
%% ========= KEY DEFAULTS =========
	\MEDebugMessage{MEFigures: started to load}
\makeatletter  % This is the good version
%%%Usage MEfigure[keys]{image file}{caption}{label}{copyright}{ScaleFactor}
%% Possible keys are wide
\newcommand\MEfigure[6][]{
	\setkeys{MEMacros}{wide=false,#1}% 
	% % Define the environment: a 'figure' if it might float, a simple caption if not
	\ifx\MayFloat\undefined
		\def\startsource{ 
	%	{%\bfseries\scriptsize \color{HeadingColor}{#3}}\par\vskip-\baselineskip
		}
		\setlength{\mywidth}{#6\textwidth}
		\def\stopsource{}
	\else
	\mbox{}\phantomsection % For hyperlinking!			
		\ifKV@MEMacros@wide
			\def\startsource{\par\begin{figure*}[h!bt]	
					}
			\def\stopsource{\end{figure*}}
			\setlength{\mywidth}{#6\textwidth}
		\else
			\setlength{\mywidth}{#6\columnwidth}
			\ifx\WEBBook\undefined\else\setlength{\mywidth}{.55\mywidth}\fi
			\ifx\eBook\undefined\else\setlength{\mywidth}{.7\mywidth}\fi
			\def\startsource{\par\begin{figure}[h!btp]}
			\def\stopsource{\end{figure}}
		\fi
	\fi
	
	\par\startsource	  
	{{\centering
	
			\vbox{
				\ifx\WEBBook\undefined\vskip .3\baselineskip\else\vskip .3\baselineskip\fi
				\begin{minipage}{\mywidth}
					\ifthenelse{\equal{#5}{}}{}{%\vskip-.1\baselineskip
					{\tiny \copyright #5}	\hfill
						}  %\copyright
				\end{minipage}
				\vskip-.8\baselineskip
				\noindent\makebox[\mywidth]{\color{SeparatorColor}\rule{\mywidth}{1pt}}\par\vskip.2\baselineskip
						\maxsizebox{\mywidth}{.75\textheight}
						{ 
													
						\includegraphics[width=\mywidth,keepaspectratio]{#2}
						}
					\par\vskip-.8\baselineskip\par
				\noindent\makebox[\mywidth]{\color{SeparatorColor}\rule{\mywidth}{1pt}}\par
			
				\begin{minipage}{\mywidth} 			% % Add a caption
					\ifx\MayFloat\undefined
						{\bfseries\scriptsize {\color{HeadingColor}#3}}
					\else{\color{HeadingColor}\caption{{#3}}\label{#4}}
					\fi 
				\end{minipage}
			}
		
%	\ifx\LecturePrintable\undefined\else\par\vspace{-15pt}\fi				
}}
	\stopsource%
}

%%%Usage MEfigurewithtext[keys]{image file}{caption}{label}{copyright}{ScaleFactor}{text}
\newcommand\MEfigurewithtext
[ 7][]{
	% % Define the environment: a 'figure' if it might float, a simple caption if not
	\ifx\MayFloat\undefined
		\def\startsource{}
		\setlength{\mywidth}{\columnwidth}
		\def\stopsource{}
		\def\CombinedMode{yes}
	\else
	\mbox{}\phantomsection % For hyperlinking!			
			\setlength{\mywidth}{\columnwidth}
			\ifx\WEBBook\undefined\else\setlength{\mywidth}{.55\mywidth}\fi
			\ifx\eBook\undefined\else\setlength{\mywidth}{.7\mywidth}\fi
			\def\startsource{\par\begin{figure}[!hbtp]}
			\def\stopsource{\end{figure}}
	\fi
 
 \ifx\CombinedMode\undefined 
   %% This is the conventional mode: the figure, plus the text  	
	\par\startsource	  
	{{\centering
	
			\vbox{
%				\ifx\WEBBook\undefined\else\vskip-.3\baselineskip\fi
				\begin{minipage}{\mywidth}
					\ifthenelse{\equal{#5}{}}{}{%\vskip-.1\baselineskip
					{\tiny \copyright #5}	\hfill
						}  %\copyright
				\end{minipage}
				\vskip-.8\baselineskip
				\noindent\makebox[\mywidth]{\color{SeparatorColor}\rule{\mywidth}{1pt}}\par\vskip.2\baselineskip
						\maxsizebox{\mywidth}{.75\textheight}
						{ 
													
						\includegraphics[width=\mywidth,keepaspectratio]{#2}
						}
					\par\vskip-.8\baselineskip\par
				\noindent\makebox[\mywidth]{\color{SeparatorColor}\rule{\mywidth}{1pt}}\par
			
				\begin{minipage}{\mywidth} 			% % Add a caption
					\ifx\MayFloat\undefined
						{\bfseries\scriptsize {\color{HeadingColor}#3}}
					\else{\color{HeadingColor}\caption{{#3}}\label{#4}}
					\fi 
				\end{minipage}
			}
}}
	\stopsource%
	\ifx\LecturePrintable\undefined #7\fi
\else
  \begin{tabular}{cc}
  %This is a combined mode make a tabular environment, left column figure, right column text
    \begin{minipage}[b]{#6\textwidth}
    \setlength{\mywidth}{\textwidth}
	\par\startsource	  
	{{\centering
	
			\vbox{
%				\ifx\WEBBook\undefined\else\vskip-.3\baselineskip\fi
				\begin{minipage}{\mywidth}
					\ifthenelse{\equal{#5}{}}{}{\vskip .1\baselineskip
					{\tiny \copyright #5}	\hfill
						}  %\copyright
				\end{minipage}
				\vskip-.8\baselineskip
				\noindent\makebox[\mywidth]{\color{SeparatorColor}\rule{\mywidth}{1pt}}\par\vskip.2\baselineskip
						\maxsizebox{\mywidth}{.75\textheight}
						{ 
													
						\includegraphics[width=\mywidth,keepaspectratio]{#2}
						}
					\par\vskip-.8\baselineskip\par
				\noindent\makebox[\mywidth]{\color{SeparatorColor}\rule{\mywidth}{1pt}}\par
			
				\begin{minipage}{\mywidth} 			% % Add a caption
					\ifx\MayFloat\undefined
						{\bfseries\scriptsize {\color{HeadingColor}#3}}
					\else{\color{HeadingColor}\caption{{#3}}\label{#4}}
					\fi 
				\end{minipage}
			}
}}
	\stopsource%
				\end{minipage}
  &
  \ifx\LecturePrintable\undefined
    \begin{minipage}[b]{\textwidth-#6\textwidth}
 %       {#7}
    \end{minipage}%
    \fi
  \\
  \end{tabular}
  
\fi	%CombinedMode	
%	\ifx\LecturePrintable\undefined\else\par\vspace{-15pt}\fi				
}%



%%%Usage MEtikzfigure[keys]{tikz source}{caption}{label}{copyright}{ScaleFactor}
\newcommand\MEtikzfigure[6][]{
	\setkeys{MEMacros}{wide=false,#1}% 
	% % Define the environment: a 'figure' if it might float, a simple caption if not
	\ifx\MayFloat\undefined
		\def\startsource{ %\vskip.2\baselineskip
	%	{\bfseries\scriptsize \color{HeadingColor}{#3}}\par%\vskip-\baselineskip
		}
		\setlength{\mywidth}{#6\textwidth}
		\def\stopsource{}
	\else
		\ifKV@MEMacros@wide
			\def\startsource{\par\begin{figure*}[!hbt]	
					}
			\def\stopsource{\end{figure*}}
			\setlength{\mywidth}{#6\textwidth}
		\else
			\setlength{\mywidth}{#6\columnwidth}
			\ifx\WEBBook\undefined\else\setlength{\mywidth}{.55\mywidth}\fi
			\ifx\eBook\undefined\else\setlength{\mywidth}{.7\mywidth}\fi
			\def\startsource{\par\begin{figure}[!hbtp]}
			\def\stopsource{\end{figure}}
		\fi
	\fi
	
	\par\startsource	  
	\mbox{}\phantomsection % For hyperlinking!
	{{\centering
			\vbox{
				\ifx\WEBBook\undefined\else\vskip-.3\baselineskip\fi
				\begin{minipage}{\mywidth}
					\ifthenelse{\equal{#5}{}}{}{\vskip-.1\baselineskip{\tiny \copyright #5}	\hfill
						}  %\copyright
				\end{minipage}
				\par\vskip-\baselineskip
				\noindent\makebox[\mywidth]{\color{SeparatorColor}\rule{\mywidth}{1pt}}\par\vskip.2\baselineskip
						\maxsizebox{\mywidth}{.75\textheight}
						{ #2 }
%					\par\vskip-.8\baselineskip\par
				\noindent\makebox[\mywidth]{\color{SeparatorColor}\rule{\mywidth}{1pt}}\par		
				\begin{minipage}{\mywidth} 			% % Add a caption
					\ifx\MayFloat\undefined
						{\scriptsize {\color{HeadingColor}#3}}
					\else{\color{HeadingColor}\caption{{#3}}\label{#4}}
					\fi 
				\end{minipage}
			}		
	\ifx\LecturePrintable\undefined\else\par\vspace{-7pt}\fi				
}}
	\stopsource%
}



\makeatother

%http://tex.stackexchange.com/questions/99070/check-for-a-valid-file-before-using-includegraphics
% Usage: IfGraphicFileExists{FileName WITHOUT extension}{TRUE bracnh} {FALSE branch} 
% File must have extension which 'graphix' can handle
\makeatletter
\newif\ifgraphicexist

\catcode`\*=11
\newcommand\IfGraphicFileExists[1]{%
%\newcommand\IfGraphicFileExists[1]{%
 \begingroup
 \global\graphicexisttrue
   \let\input@path\Ginput@path
  \filename@parse{#1}%
  \ifx\filename@ext\relax
    \@for\Gin@temp:=\Gin@extensions\do{%
      \ifx\Gin@ext\relax
        \Gin@getbase\Gin@temp
      \fi}%
  \else
    \Gin@getbase{\Gin@sepdefault\filename@ext}%
    \ifx\Gin@ext\relax
       \global\graphicexistfalse
       \def\Gin@base{\filename@area\filename@base}%
       \edef\Gin@ext{\Gin@sepdefault\filename@ext}%
    \fi
  \fi
  \ifx\Gin@ext\relax
         \global\graphicexistfalse
    \else 
       \@ifundefined{Gin@rule@\Gin@ext}%
         {\global\graphicexistfalse}%
         {}%
    \fi  
  \ifx\Gin@ext\relax 
   \gdef\imageextension{unknown}%
  \else
   \xdef\imageextension{\Gin@ext}%
  \fi 
 \endgroup 
 \ifgraphicexist
  \expandafter \@firstoftwo
 \else
  \expandafter \@secondoftwo
 \fi 
 } 
\catcode`\*=12
\makeatother 

	\MEDebugMessage{MEFigures: finished to load}



	\MEDebugMessage{METables: started to load}

\makeatletter  % This is the good version
%%%Usage MEtable[keys]{source tabular}{caption}{label}{copyright}{ScaleFactor}
%% Possible keys are wide, 
%% Keys: wide 	: if two-column size
%% ========= KEY DEFAULTS =========
\newcommand\MEtable[6][]{%
	\setkeys{MEMacros}{wide=false,#1}% 
	% % Define the environment: a 'table' if it might float, a simple caption if not
	\ifx\MayFloat\undefined
		\def\startsource{%\vskip.2\baselineskip
			{\color{HeadingColor}\bfseries\scriptsize #3}\par\vskip-1.8\baselineskip}
		\setlength{\mywidth}{#6\textwidth}
		\def\stopsource{}
	\else
		\ifKV@MEMacros@wide
			\def\startsource{\begin{table*}[!hbtp]	}
			\def\stopsource{\end{table*}}
			\setlength{\mywidth}{#6\textwidth}
		\else
			\setlength{\mywidth}{#6\columnwidth}
			\ifx\WEBBook\undefined\else\setlength{\mywidth}{.5\mywidth}\fi
			\ifx\eBook\undefined\else\setlength{\mywidth}{.7\mywidth}\fi
			\def\startsource{\begin{table}[!hbtp]}
			\def\stopsource{\end{table}}
		\fi
	\fi
	\startsource	  
	\mbox{}\phantomsection % For hyperlinking!
		\begin{center}
			\vbox{
%				\ifx\WEBBook\undefined\else\vskip-1.3\baselineskip\fi
				\begin{minipage}{\mywidth} 			% % Add a caption
					\ifx\MayFloat\undefined
					\else{\color{HeadingColor}\caption{#3}\label{#4}}
					\fi 
				\end{minipage}
%				\vglue-0.2\baselineskip\par
				\begin{minipage}{\mywidth}
					\ifthenelse{\equal{#5}{}}{}{\vskip-.1\baselineskip{\tiny \copyright #5}	\hfill
						}  %\copyright
				\end{minipage}
				\par %\vskip-\baselineskip
				\noindent\makebox[\mywidth]{\color{SeparatorColor}\rule{\mywidth}{1pt}}\par
						\maxsizebox{\mywidth}{.75\textheight}{ %\hskip-.4em
						#2}
				\par\noindent\makebox[\mywidth]{\color{SeparatorColor}\rule{\mywidth}{1pt}}\par
%				\par\vskip-.7\baselineskip
			}
		\end{center}
%	\ifx\LecturePrintable\undefined\else\par\vskip-.8\baselineskip\fi
	\stopsource%
}

\makeatother

	\MEDebugMessage{METables: finished to load}

	\MEDebugMessage{MEMacros: finished to load}


\usepackage[plainpages=false,pdfpagelabels,colorlinks=true,linkcolor=webgreen,citecolor=ForestGreen]{hyperref}
\usepackage{midfloat}
\IfFileExists{src/Glossary.tex}
{
	
	\newglossaryentry{computer}
	{
		name={computer},
		description={is a programmable machine that receives input,
			stores and manipulates data, and provides
			output in a useful format}
	}
	
	\newglossaryentry{sampleone}{name={sampleone},description={another example}}
	%\newglossaryentry{sample}{name={sample},description={an example}}
	
	\newacronym{abc}{ABC}{a sample acronym}
	\newacronym{def}{DEF}{another sample acronym}
%	\newacronym[type=better]{xyz}{XYZ}{another sample acronym}
%	\newacronym[type=common]{xyzw}{XYZW}{yet another sample acronym}
	\newacronym{xyz}{XYZ}{another sample acronym}
	\newacronym{xyzw}{XYZW}{yet another sample acronym}

}
{} % Glossary

	% % Configure pgfplot diagram appearance
	\pgfplotscreateplotcyclelist{my color list}{%
	solid, color=webblue, every mark/.append style={solid, fill=webblue}, mark=*\\%
	densely dashdotted, color=webgreen, every mark/.append style={solid, fill=webred},mark=diamond*\\%
	densely dotted, color=webbrown, every mark/.append style={solid, fill=webgreen}, mark=triangle*\\%
	loosely dashed, color=webred, every mark/.append style={solid, fill=webbrown},mark=*\\%
	dotted, color=webblue, every mark/.append style={solid, fill=webyellow}, mark=square*\\%
	densely dotted, color=webgreen, every mark/.append style={solid, fill=gray}, mark=otimes*\\%
	dashed, color=webbrown, every mark/.append style={solid, fill=gray},mark=diamond*\\%
	densely dashed, every mark/.append style={solid, fill=gray},mark=square*\\%
	dashdotted, every mark/.append style={solid, fill=gray},mark=otimes*\\%
	dasdotdotted, every mark/.append style={solid},mark=star\\%
	}
	
	\pgfplotscreateplotcyclelist{my black white}{%
	solid, every mark/.append style={solid, fill=gray}, mark=*\\%
	densely dashdotted,every mark/.append style={solid, fill=gray},mark=diamond*\\%
	densely dotted, every mark/.append style={solid, fill=gray}, mark=triangle*\\%
	loosely dashed, every mark/.append style={solid, fill=gray},mark=*\\%
	dotted, every mark/.append style={solid, fill=gray}, mark=square*\\%
	densely dotted, every mark/.append style={solid, fill=gray}, mark=otimes*\\%
	dashed, every mark/.append style={solid, fill=gray},mark=diamond*\\%
	densely dashed, every mark/.append style={solid, fill=gray},mark=square*\\%
	dashdotted, every mark/.append style={solid, fill=gray},mark=otimes*\\%
	dasdotdotted, every mark/.append style={solid},mark=star\\%
	}
	
			
%%%http://tex.stackexchange.com/questions/20034/glossaries-print-first-occurence-by-using-acronym-desc
%\AtBeginDocument{%
%	\defglsdisplayfirst[\acronymtype]{%
%		\glsentryshort{\glslabel} (\glsentrylong{\glslabel})#4%
%%		\glsentryshort{\glslabel} \footnote{\glsentrylong{\glslabel})#4}%
%	}%
%}
			
%%%% http://archive.cs.uu.nl/mirror/CTAN/macros/latex/contrib/glossaries/samples/sample-custom-acronym.tex
%%  % This is a sample file to illustrate how to define a custom
%%  % acronym style. This example defines the acronyms so that on first use
%%  % they display the short form in the text and with the long form
%%  % and description in a footnote. In the main body of the
%%  % document the short form will be displayed in small caps, but in
%%  % the list of acronyms the short form is displayed in normal
%%  % capitals. To ensure this, the short form should be written in
%%  % lower case when the acronym is defined, and \MakeTextUppercase is
%%  % used when it's displayed in the list of acronyms.
%%  
%%  \newacronymstyle{custom-fn}% new style name
%%  {% Check for long form in case of a mixed glossary
%%  	\ifglshaslong{\glslabel}{\glsgenacfmt}{\glsgenentryfmt}%
%%  }%
%%  {% Style definitions:
%%  	% User needs to supply the description:
%%  	\renewcommand*{\GenericAcronymFields}{}%
%%  	% Need to ensure hyperlinks are switched off on first use:
%%  	\glshyperfirstfalse
%%  	% Redefine the commands used by \glsgenacfmt on first use:
%%  	\renewcommand*{\genacrfullformat}[2]{%
%%  		\firstacronymfont{\glsentryshort{##1}}##2%
%%  		\footnote{\glsentrylong{##1}: \glsentrydesc{##1}}%
%%  	}%
%%  	\renewcommand*{\Genacrfullformat}[2]{%
%%  		\firstacronymfont{\Glsentryshort{##1}}##2%
%%  		\footnote{\glsentrylong{##1}: \glsentrydesc{##1}}%
%%  	}%
%%%  	\renewcommand*{\genplacrfullformat}[2]{%
%%%  		\firstacronymfont{\glsentryshortpl{##1}}##2%
%%%  		\footnote{\glsentrylongpl{##1}: \glsentrydesc{##1}}%
%%%  	}%
%%%  	\renewcommand*{\Genplacrfullformat}[2]{%
%%%  		\firstacronymfont{\Glsentryshortpl{##1}}##2%
%%%  		\footnote{\glsentrylongpl{##1}: \glsentrydesc{##1}}%
%%%  	}%
%%%  	% Redefine the no-link full forms:
%%%  	\renewcommand*{\glsentryfull}[1]{%
%%%  		\glsentrylong{##1}\space(\acronymfont{\glsentryshort{##1}})%
%%%  	}%
%%%  	\renewcommand*{\Glsentryfull}[1]{%
%%%  		\Glsentrylong{##1}\space(\acronymfont{\glsentryshort{##1}})%
%%%  	}%
%%%  	\renewcommand*{\glsentryfullpl}[1]{%
%%%  		\glsentrylongpl{##1}\space(\acronymfont{\glsentryshortpl{##1}})%
%%%  	}%
%%%  	\renewcommand*{\Glsentryfullpl}[1]{%
%%%  		\Glsentrylongpl{##1}\space(\acronymfont{\glsentryshortpl{##1}})%
%%%  	}%
%%%  	% Redefine the link full forms:
%%%  	\renewcommand*{\acrfullfmt}[3]{%
%%%  		\glslink[##1]{##2}{%
%%%  			\glsentrylong{##2}##3\space(\acronymfont{\glsentryshort{##2}})%
%%%  		}%
%%%  	}%
%%%  	\renewcommand*{\Acrfullfmt}[3]{%
%%%  		\glslink[##1]{##2}{%
%%%  			\Glsentrylong{##2}##3\space(\acronymfont{\glsentryshort{##2}})%
%%%  		}%
%%%  	}%
%%%  	\renewcommand*{\ACRfullfmt}[3]{%
%%%  		\glslink[##1]{##2}{%
%%%  			\MakeTextUppercase{%
%%%  				\glsentrylong{##2}##3\space
%%%  				(\acronymfont{\glsentryshort{##2}})%
%%%  			}%
%%%  		}%
%%%  	}%
%%%  	\renewcommand*{\acrfullplfmt}[3]{%
%%%  		\glslink[##1]{##2}{%
%%%  			\glsentrylongpl{##2}##3\space
%%%  			(\acronymfont{\glsentryshortpl{##2}})%
%%%  		}%
%%%  	}%
%%%  	\renewcommand*{\Acrfullplfmt}[3]{%
%%%  		\glslink[##1]{##2}{%
%%%  			\Glsentrylongpl{##2}##3\space
%%%  			(\acronymfont{\glsentryshortpl{##2}})%
%%%  		}%
%%%  	}%
%%%  	\renewcommand*{\ACRfullplfmt}[3]{%
%%%  		\glslink[##1]{##2}{%
%%%  			\MakeTextUppercase{%
%%%  				\glsentrylongpl{##2}##3\space
%%%  				(\acronymfont{\glsentryshortpl{##2}})%
%%%  			}%
%%%  		}%
%%%  	}%
%%%  	% Use smallcaps for the acronym in the document text:
%%%  	\renewcommand*{\acronymfont}[1]{\textsc{##1}}%
%%%  	\renewcommand*{\acrpluralsuffix}{\glstextup{\glspluralsuffix}}%
%%%  	% Sort acronyms according to the long form:
%%%  	\renewcommand*{\acronymsort}[2]{##2}%
%%%  	% Set the name in the list of acronyms to the long form followed by
%%%  	% the short form (in upper case) in parentheses:
%%%  	\renewcommand*{\acronymentry}[1]{%
%%%  		\Glsentrylong{##1}\space(\MakeTextUppercase{\glsentryshort{##1}})}%
%%  }
%% % Now set the new acronym style (to override the default style)
%% \setacronymstyle{custom-fn}
%%\setabbreviationstyle[acronym]
%\fi% \DisableGlossary
%
%%\usepackage{memhfixc} %Any re­cent ver­sion of hy­per­ref will au­to­mat­i­cally load this pack­age if it finds it­self run­ning un­der the mem­oir class.
%
%%\LDebugMessage{Preparing eBook-style book started}
%%\usepackage{tocloft}% http://ctan.org/pkg/tocloft
%%\setlength{\cftsubsecnumwidth}{3em}% Set length of number width in ToC for \subsection
%%		\usepackage[margin=10pt,font=small,labelfont=bf,labelsep=endash,
%%		font={color=HeadingColor},textfont={sf}]{caption} % Format float captions
%
%
%\title{\LectureTitle}
%\ifx\LectureSubtitle\undefined
%\else
%\subtitle{\LectureSubtitle}
%\fi
%\author{\LectureAuthor}
%\ifx\LectureRevision\undefined
%	\date{\today}
%\else
%	\date{\LectureRevision}
%\fi

\newsavebox{\ChpNumBox}
%\definecolor{ChapBlue}{rgb}{0.00,0.65,0.65}
\makeatletter
\newcommand*{\thickhrulefill}{%
\leavevmode\leaders\hrule height 1\p@ \hfill \kern \z@}
\newcommand*\BuildChpNum[2]{%
\begin{tabular}[t]{@{}c@{}}
\makebox[0pt][c]{#1\strut} \\[.5ex]
\colorbox{HeadingColor}{%
\ifx\eBook\undefined
	\ifx\WEBBook\undefined
	\rule[-10em]{0pt}{0pt}%
	\else
	\rule[-5em]{0pt}{0pt}%
	\fi
\else
	\rule[-5em]{0pt}{0pt}%
\fi
\rule{1ex}{0pt}\color{white}#2\strut
\rule{1ex}{0pt}}%
\end{tabular}}

\makechapterstyle{BlueBox}{%
\renewcommand{\chapnamefont}{\large\scshape}
\renewcommand{\chapnumfont}{\huge\bfseries}
\renewcommand{\chaptitlefont}{\raggedright\huge\bfseries}
\setlength{\beforechapskip}{10pt}
\setlength{\midchapskip}{16pt}
\setlength{\afterchapskip}{20pt}
\renewcommand{\printchaptername}{}
\renewcommand{\chapternamenum}{}
\renewcommand{\printchapternum}{%
\sbox{\ChpNumBox}{%
\BuildChpNum{\chapnamefont\@chapapp}%
{\chapnumfont\thechapter}}}
\renewcommand{\printchapternonum}{%
\sbox{\ChpNumBox}{%
\BuildChpNum{\chapnamefont\vphantom{\@chapapp}}%
{\chapnumfont\hphantom{\thechapter}}}}
\renewcommand{\afterchapternum}{}
\renewcommand{\printchaptertitle}[1]{%
\usebox{\ChpNumBox}\hfill
\parbox[t]{\hsize-\wd\ChpNumBox-1em}{%
\vspace{\midchapskip}%
\thickhrulefill\par
\chaptitlefont ##1\par}}%
}
\chapterstyle{BlueBox}
	\setsecheadstyle{\color{HeadingColor}\bfseries\memRTLraggedright}
	\setsubsecheadstyle{\color{HeadingColor}\bfseries\small\memRTLraggedright}

	\setlength{\parskip}{6pt plus 2pt minus 3pt}
	\setlength{\parindent}{0pt}
	\setlength{\textfloatsep}{6.0pt plus 1.0pt minus 2.0pt}
	\setlength{\floatsep}{6.0pt plus 1.0pt minus 2.0pt}
	\setlength{\intextsep}{6.0pt plus 1.0pt minus 2.0pt}
% http://ctan.ijs.si/tex-archive/macros/latex/contrib/layouts/layman.pdf
	\raggedbottom
		% Alter some LaTeX defaults for better treatment of figures:
    % See p.105 of "TeX Unbound" for suggested values.
    % See pp. 199-200 of Lamport's "LaTeX" book for details.
    %   General parameters, for ALL pages:
    \renewcommand{\topfraction}{0.9}	% max fraction of floats at top
    \renewcommand{\bottomfraction}{0.8}	% max fraction of floats at bottom
    %   Parameters for TEXT pages (not float pages):
    \setcounter{topnumber}{4}
    \setcounter{bottomnumber}{4}
    \setcounter{totalnumber}{6}     % 2 may work better
    \setcounter{dbltopnumber}{4}    % for 2-column pages
    \renewcommand{\dbltopfraction}{0.9}	% fit big float above 2-col. text
    \renewcommand{\textfraction}{0.1}	% allow minimal text w. figs
    %   Parameters for FLOAT pages (not text pages):
    \renewcommand{\floatpagefraction}{0.8}	% require fuller float pages
	% N.B.: floatpagefraction MUST be less than topfraction !!
    \renewcommand{\dblfloatpagefraction}{0.8}	% require fuller float pages

%\setlist{nosep,before=\vspace{-0.5\baselineskip},after=\vspace{-0.5\baselineski‌​p}}
\tightlists

% Set page headings
\makeevenhead{headings}{\color{burntorange}\thepage}{}{\color{burntorange}\scriptsize\slshape\leftmark}
\makeoddhead{headings}{\color{burntorange}\scriptsize\slshape\rightmark}{}{\color{burntorange}\thepage}

	\MEDebugMessage{MESetupPrintedFormat: frontmatter started to load}
%%			
%%%	% % % A macro for representing the front matter needed in the selected form
%
\newcommand\MEfrontmatterRoot{
	\MEDebugMessage{MESetupPrintedFormat: frontmatter started to execute}
	%% Check if the user wants to have his own front matter
	\IfFileExists{src/FrontMatter.tex}
	{%% Have his own frontmatter things; use it
		\input{src/FrontMatter.tex}
		\MEDebugMessage{MESetupPrintedFormat: Using user's FrontMatter}
	}
	{%% No, we need to use the built-in front matter
	\IfFileExists{src/Heading.tex}
	{
		\pagenumbering{gobble}
		\maketitle
		\frontmatter
{\footnotesize
\setlength{\parindent}{0pt}
\setlength{\parskip}{\baselineskip}

\ifx\DisableCopyright\undefined
  \IfFileExists{src/Copyright.tex}
  { % The user has his own copyright
    \ifx\LectureLanguage\undefined
\includegraphics{../../common/images/CC88x31} 
This Work is licensed under a Creative Commons Attribution 4.0 International License.

  \textcopyright{} Copyright \copyright\ 2011-\the\year\ \LectureAuthor\ (\LectureEmail) \\
  All rights reserved


  Printed in the World, using recycled electrons

\else
%\LectureLanguage\\
%\detokenize{\LectureLanguage}\\
%{\detokenize{magyar}}\\
%  \ifthenelse{\equal{\LectureLanguage}{\detokenize{magyar}}}
  { % % implement Hungarian copyright
\includegraphics{../../common/images/CC88x31}  Ez a Mű a \url{"http://creativecommons.org/licenses/by/4.0/} Creative Commons Nevezd meg! 4.0 Nemzetközi Licenc feltételeinek megfelelően felhasználható.
  
  \textcopyright{} Szerzői jogok \copyright\ 2011-\the\year\ \LectureAuthor\ (\LectureEmail) \\
  Minden jog fenntartva

 Kizárólag újra hasznosított elektronokkal nyomtatható

  }
\fi

    \MEDebugMessage{MESetupPrintedFormat: Using user's copyright}
  }
  {
	\ifx\LectureLanguage\undefined
\includegraphics{../../common/images/CC88x31} 
This Work is licensed under a Creative Commons Attribution 4.0 International License.

  \textcopyright{} Copyright \copyright\ 2011-\the\year\ \LectureAuthor\ (\LectureEmail) \\
  All rights reserved


  Printed in the World, using recycled electrons

\else
%\LectureLanguage\\
%\detokenize{\LectureLanguage}\\
%{\detokenize{magyar}}\\
%  \ifthenelse{\equal{\LectureLanguage}{\detokenize{magyar}}}
  { % % implement Hungarian copyright
\includegraphics{../../common/images/CC88x31}  Ez a Mű a \url{"http://creativecommons.org/licenses/by/4.0/} Creative Commons Nevezd meg! 4.0 Nemzetközi Licenc feltételeinek megfelelően felhasználható.
  
  \textcopyright{} Szerzői jogok \copyright\ 2011-\the\year\ \LectureAuthor\ (\LectureEmail) \\
  Minden jog fenntartva

 Kizárólag újra hasznosított elektronokkal nyomtatható

  }
\fi

    \MEDebugMessage{MESetupPrintedFormat: Using default copyright}
  }
\fi  

\ifx\DisableAbstract\undefined
	\IfFileExists{src/Abstract}
	{
	    {\bfseries \MEAbstract}\par
	     \MEDtext
{
For teaching my own courses, I developed a set of macros, since to display  the course material
under different circumstances different forms of teaching materials are needed. On the lectures
I present the theoretical material in form of slides, and the explanation referring to the slides
(of course in somewhat compressed form) I offer for my students, in a booklet-like form. My students
studies that material either from printed hard copies, or from screen, using a browser or sometimes
on mobile devices. The field is in continuous development, so I need to develop my teaching material
also continuously. Because of this, it is a must to develop those different forms of the material
synchronously. The simplest way to do so, is to use the same source, with proper formatting instructions.
It is a serious challange, to develop course material for the today's students, who are used to
lessons with high resolution, attractive graphics, good computer background.

As a common base, I used LaTeX, from which I produce the slides from the lectures using package beamer,
and the reading material using package 'memoir'. This latter one can even reach the "on demand printing"
quality. The printed material attempts to catch the attention with attractive graphical appearance,
using above the average amount of figures (of course on can make also 'book-like' book, too). The
booklet-like version contains all figures from the lectures, and some comprehensive version of
the text of the lecture. The same text appears in screen-oriented form in the WEB-book format, and
in the eBook compatible (native PDF) format.In those two forms (mainly targeting the small-screen
mobile devices) bigger fonts are used and one figure/screen is displayed.

To satisfy those, somewhat contradictional  requirements, one must make bargains,and more time and care
must be invented in the formatting. The macros allow to support also foreign languages, and even
to prepare course in English and your own language side by side. Using the possibilities of LaTeX,
animations, movies, web-pages, sound files, etc. can also be embedded, but one has to think about
the equivalent appearance on hard copies.
}
{
Kurzusaim oktatásához saját makrókészletet fejlesztettem, mivel az oktatandó anyag megjelenítéséhez
különböző körülmények között különböző formákra van szükség. Az elméleti anyagot az előadásokon
diasorozat alapján mutatom be, és a diákhoz fűzött magyarázatokat (természetesen tömörítve) 
jegyzet-szerű formában a hallgatóság számára is rendelkezésre bocsátom. A hallgatóság ezt az 
anyagot részben kinyomtatva, részben képernyőn olvasva (akár mobil eszközökön is) tanulmányozza.
A terület folyamatos fejlődése miatt a tananyag is állandó fejlesztésre szorul, ezért feltétlenül
szükséges, hogy az említett megjelenési formákat egymással szinkronban lehessen fejleszteni.
Ennek legegyszerűbb megvalósítási formája, hogy egyazon forrásból, megfelelő formattálási utasításokkal
készítem a tananyagokat.
Számítógéppel alaposan megtámogatott, nagy felbontáshoz és vonzó grafikához
szokott hallgatóság számára a fenti feltételeknek megfelelő tananyagot készíteni komoly kihívás.

Közös alapként a LaTeX nyelvet használtam, amely nyelven készült forrásból az elterjedten használt
Beamer prezentáció készítő makró csomaggal állítom elő az előadáson bemutatandó diákat, és a memoir
könyv készítő makró csomaggal a hallgatóság számára rendelkezésre bocsátandó "tananyagot". Ez utóbbi
akár az "on demand printing" minőséget is elérheti. Vonzó grafikus megjelenéssel, a szokásoshoz képest 
sokkal több ábrával igyekszik felkelteni az anyag a hallgatóság figyelmét (de lehet belőle kevésbé "fancy",
inkább "plain" stílusú, de még mindig könyv minőségű változatot is készíteni). A jegyzet-szerű 
változat az előadáson bemutatott ábrákat és szöveget teljes egészében tartalmazza, az előadás
szövegének egy tömörített változatával kiegészítve. Ugyanez a könyv-szerű anyag jelenik meg a képernyős
olvasásra szánt WEB-es formátumban, illetve az eBook kompatibilis (natív PDF) változatban. Ebben a változatban az anyag egy-képernyőnyi darabokra van tördelve,
és (főként kisebb képernyőjű mobil eszközökre gondolva) nagyobb betűkkel, egy ábra/képernyő módon
jelenik meg.

A többféle, egymásnak ellentmondó megjelenítési igény természetesen csak kompromisszumokkal
oldható meg, így a tananyag megírása során a szöveg megjelenés formázására több gondot és időt kell fordítani.
A makrócsomaggal akár egyidejűleg idegen nyelvre is lehet ugyanazon tananyagot fejleszteni.
A LaTeX lehetőségeivel akár animáció, mozgófilm, WEB-lap, hang, stb. is beépíthető, természetesen gondolni
kell a nyomtatott anyag ekvivalens megjelenítésére.
}

 %\vspace{2cm}
	}
	{\vspace{10cm}\MEDebugMessage{MESetupPrintedFormat: Abstract info not found}}
\fi
\vskip2\baselineskip
}%of footnotesize


\begin{center}
\begin{tabular}{ll}
First edition:                        & August 2016 \\
\end{tabular}
\end{center}
\vfill
\ifx\LectureLogo\undefined
\else
	\LectureLogo
\fi
\vfill
\noindent


\begin{center}
{\small \LectureRevision\\
10 09 08 07 06 05 04 03 02 01\hspace{2em}19 18 17 16 15 14 13}
\end{center}

\clearpage
		
\pagenumbering{roman}			
\ifx\DisableTableOfContents\undefined
	\tableofcontents	
\fi
}
{
\MEDebugMessage{MESetupPrintedFormat: Title page not generated}
}
}
	\MEDebugMessage{MESetupPrintedFormat: frontmatter finished to execute}
}%FrontMatterRoot

	\MEDebugMessage{MESetupPrintedFormat: frontmatter finished to load}

	\MEDebugMessage{MESetupPrintedFormat: backmatter started to load}
%	% % % A macro for representing the back matter needed in the selected form
%	% % %
%	%% Usage MEbackmatter
\newcommand\MEbackmatter{
	%	%% Check if the user wants to have his own back matter
	\MEDebugMessage{MESetupPrintedFormat: backmatter started to execute}
		% http://tex.stackexchange.com/questions/472/how-can-i-have-two-or-more-distinct-indexes
		% http://www.latex-community.org/forum/viewtopic.php?f=51&t=8096
		\ifx\DisableIndex\undefined		
			\clearpage\printindex
		\else 
			\MEDebugMessage{Index printing disabled}
		\fi
		\IfFileExists{src/Glossary.tex}
		{
			\glossarystyle{altlisthypergroup}
			\ifx\DisableAcronyms\undefined
			\clearpage
			\printglossary[type=acronym]
			\else 
			\MEDebugMessage{Printing acronyms disabled}
			\fi
			\ifx\DisableGlossary\undefined
			  \clearpage
			  \printglossary
			\else 
			  \MEDebugMessage{Printing glossary disabled}
			\fi
		}
		{\MEDebugMessage{No Glossary.tex file found}
		}	% % If no glossary file exists, make nothing
		\conditionalLoT    % Print list of tables, if any
		\conditionalLoF    % Print list of figures, if any
		\conditionalLoP	   % Print list of program listings, if any
		%						
		%
		\onecolumn		
		\ifx\DisableBibliography\undefined
			\ifx\LectureBibliography\undefined
				\MEDebugMessage{MESetupPrintedFormat:No bibliography file found}
			\else
				\clearpage\printbibliography
			\fi			
		\else
			\MEDebugMessage{MESetupPrintedFormat: Bibliography printing disabled}
		\fi %\DisableBibliography
		\IfFileExists{src/BackMatter.tex}
		{%% Have his own backmatter things; use it
			\input{src/BackMatter.tex}
		}
		{%% No, we need to use the built-in main matter
			\backmatter
		}
\ifx\FancyBook\undefined
\else
\cleartoverso
%%%%%%%%%%%%%%%%%%%%%%
%% Back cover
%%%%%%%%%%%%%%%%%%%%%%

%% Temporarily enlarge this page to push
%% down the bottom margin
%\enlargethispage{3\baselineskip}
\thispagestyle{empty}
\pagecolor[HTML]{0C0303}
%\pagecolor[HTML]{0E0407}
%\pagecolor[HTML]{030C03}
\begin{center}
\begin{minipage}{.8\textwidth}
\color{pearl}\Large\bfseries
\IfFileExists{src/BackCoverText}
   { 
   \input{src/BackCoverText}
   }
	{%% No back cover text provided
	  {\bfseries Back cover}
		\MEDebugMessage{MESetupPrintedFormat: Back cover text not found}
	}

\end{minipage}
\end{center}

\vspace*{\stretch{1}}

\begin{center}
%\colorbox{white}{\EANisbn[SC4]}

%\vspace*{\baselineskip}
%
%\textbf{\textcolor{LightGoldenrod!50!Gold}{\LecturePublisher \textbullet\ \texttt{http://www.inf.unideb.hu/\~{}jvegh}}}
%
%\vspace*{\baselineskip}
\ifx\LecturePublisher\undefined
\else
  \textbf{\textcolor{pearl}{Cover Illustration by  \textbullet\ \texttt{\LecturePublisher}}}
\fi
\end{center}

\fi		
	\MEDebugMessage{MESetupPrintedFormat: backmatter finished to execute}
}

	\MEDebugMessage{MESetupPrintedFormat: backmatter finished to load}

%\ifx\DisableLogo\undefined
%    \pgfdeclaremask{UC}{../../common/images/under_construction-mask}
%		\pgfdeclareimage[width=3.7cm]{um-logo}{../../common/images/um-logo}
%		\pgfdeclareimage[mask=UC,width=3cm]{UC-logo}{../../common/images/under_construction}
%\begin{center}
%\begin{tabular}{ccc}
%	& %Possible supporter 
%	\pgfuseimage{um-logo}
%	& %Publisher, if available
% %   \pgfuseimage{UC-logo}
%	\\ % %Under construction
%\end{tabular}
%\end{center}
%\fi %\DisableLogo

%\ifx\DisableLogo\undefined
%%% Define logo images
%\IfFileExists{fig/LogoImageLeft.png}
%{\def\LogoImageLeft{fig/LogoImageLeft.png}} % User has own 1st logo
%{\IfFileExists{../../common/images/LogoImageLeft.png}% No, try to use default logo
%	{\def\LogoImageLeft{../../common/images/LogoImageLeft.png}}
%	{}	% % No logo is available
%}
%\IfFileExists{fig/LogoImageRight.png}
%{\def\LogoImageRight{fig/LogoImageRight.png}} % User has own 1st logo
%{\IfFileExists{../../common/images/LogoImageRight.png}	% No, try to use default logo
%	{\def\LogoImageRight{../../common/images/LogoImageRight.png}}
%	{}	% % No logo is available
%}
%%% Define logo file, if any
%\IfFileExists{src/Logo1}
%{\input{src/Logo1}}	% User has own 1st logo
%{\IfFileExists{../../common/MELogo1}}
%{\input{../../common/MELogo1}}
%{}	% % No logo is available
%\IfFileExists{src/Logo2}
%{\input{src/Logo2}}	% User has own 2nd logo
%{\IfFileExists{../../common/MELogo2}}
%{\input{../../common/MELogo2}}
%{}	% % No logo is available
%\else			
%\MEDebugMessage{ Logo making disabled}
%\fi % DisableLogo
%
%% % Declare author, title, publisher, etc
%%	
%% Define the author, title and publisher
%% Please use accents, no encoding and fonts defined
  \ifthenelse{\equal{\LectureLanguage}{english}}
  { % In English
	\def\LectureAuthor{J\'anos V\'egh}
	\def\LectureTitle{How to use package MultEdu}
	\def\LectureSubtitle{(How to prepare interesting and attractive teaching material)}
	 %Optional
%	\def\LecturePublisher{Miskolc University Faculty of \dots and Informatics}
	\def\LectureRevision{V\Version\ \at2016.08.19}
  }% true
  {% NOT english
  }
  \ifthenelse{\equal{\LectureLanguage}{magyar}}
  {	% in Hungarian
	\def\LectureAuthor{V\'egh J\'anos}
	\def\LectureTitle{Hogyan haszn\'aljuk\\ a MultEdu csomagot} 
	\def\LectureSubtitle{(Hogyan k\'esz\'\i{}ts\"unk \'erdekes\\\ \'es vonz\'o tananyagot)} %opcionális
%	\def\LecturePublisher{Miskolci Egyetem \dots és Informatikai Kara}
	\def\LectureRevision{V\Version\ \at2016.08.19}
  }% true
  {% NOT magyar
  }
\def\LectureEmail{Janos.Vegh\at unideb.hu}
	
	
%% Define the bibliography file
%% By default, no file assumed. Allows for language-dependent bibliographies
\ifx\LectureLanguage\undefined
\def\LectureBibliography{src/Bibliography}
\else

\IfFileExists{src/Bibliographyhu}
{\def\LectureBibliography{src/Bibliography,src/Bibliographyhu}}
{\def\LectureBibliography{src/Bibliography}}
\fi

	
			% Define title, publisher, etc
%%\usepackage{bookman}	% Font type
%

\usepackage{layouts}
\tryintextsep{0pt plus2pt minus1pt}
\tryfloatsep{0pt plus2pt minus1pt}
