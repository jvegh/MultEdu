%% 
%% The document sectioning related macros for the MultEdu system
%%
%% The basic building block is the frame, because 'beamer' requires it

%%% A simple single-language frame for avoiding unnecessary typing
%%% Usage: \MEframe[keys and values]{frame subtitle}{content of the intended frame}
%% % Possible keys: shrink
%% %                         plain
%%% Actually, this is two implementations: one for the printed formats and one for the slides
	\MEDebugMessage{MESections: started to load}
%%%% The base input is formatted as 'beamer' frames
%%%% Frame specific definitions
%%%%
%% Define frame subtitle 
\ifx\LecturePrintable\undefined
%% Implementation for the slides
\newcommand{\MEframelabel}[1]{
	\def\framelabel{#1}
}
\else
%% Implementation for the printed versions
\newcommand{\MEframelabel}[1]{
	\def\framelabel{#1}
	\ifx\undefined\eBook\else \markright{#1}\fi
	\ifx\FancyBook\undefined\else\markright{#1}\fi
}
\fi %\LecturePrintable\undefined

\makeatletter
\newcommand{\MEframe}[3][]
{
	\MEframelabel{#2}
	\setkeys{MEMacros}{shrink=true, plain=false, #1}% 
%   \def\startframe{\begin{frame} {#2}{}}
	\ifKV@MEMacros@plain
	  %% The user want a plain frame, no further options considered
	  \def\startframe{\begin{frame}[plain] {\framelabel}{}}
		%message("Just plain")
	\else %@plain
	  %% Not plain, consider if to shrink
	  \ifKV@MEMacros@shrink
			  \def\startframe{\begin{frame}[shrink] {\framelabel}{}}
		%  message("just shrink")
	  \else %@shrink
			\def\startframe{\begin{frame} {\framelabel}{}}
		%message("just nothing")
	  \fi %@shrink	  
	\fi %@plain		
\startframe
#3
\end{frame}
%\par\phantom{-}\vskip-5\parskip
} % \MEframe
\makeatother

	
	%	\begingroup%
	%	\ifx\LecturePrintable\undefined
	%		\expandafter\ifstrequal\expandafter{\ME@plain}{true}
	%			{ % It is a plain frame
	%				\expandafter\ifstrequal\expandafter{\ME@shrink}{true}
	%				{\begin{frame}[plain,shrink] {#2}{\framelabel}
	%			%	it was PLAIN  and SHRINK
	%				}
	%				{\begin{frame}[plain]{#2}{\framelabel}
	%			%	It was just PLAIN
	%				}
	%		  	}
	%		  	{ % It is not plain
	%				\expandafter\ifstrequal\expandafter{\ME@shrink}{true}
	%				{\begin{frame}[shrink]{#2}{\framelabel}
	%			%	It was ONLY shrink
	%				}
	%			  	 {\begin{frame}{#2}{\framelabel}
	%			  	% It was just empty
	%			  	 }
	%		  	}
	%		 \else
	%		 \fi
	%	\endgroup%
	%\par The key was: #1\par
	%\begin{frame}{#2}{\framelabel}
	%#3
	%\ifx\LecturePrintable\undefined\phantom{-}\fi
	%\end{frame}

%	\expandafter\ifstrequal\expandafter{\ME@shrink}{true}
%		{% We are schrinking
%			\expandafter\ifstrequal\expandafter{\ME@plain}{true}
%				{\def\startframe{\begin{frame}[shrink,plain] {\framelabel}{}}}
%				{\def\startframe{\begin{frame}[shrink] {\framelabel}{}}}
%		%\def\myshrink{yes}
%		}
%		{% We are not schrinking
%			\expandafter\ifstrequal\expandafter{\ME@plain}{true}
%			{\def\startframe{\begin{frame}[plain] {\framelabel}{}}}
%			{\def\startframe{\begin{frame} {\framelabel}{}}}
%		}
%%		\expandafter\ifstrequal\expandafter{\ME@plain}{true}
%%			{\def\myplain{yes}}{}
%	
		
%			\def\startframe{\begin{frame}[shrink] {\framelabel}{}}
%			\ifx\LecturePrintable\undefined	% Making slides

%% A dual-language frame for avoiding unnecessary typing
%% Usage: \MEDframe[keys and values]
%{frame subtitle, \LectureLanguage}{content }
%{frame subtitle, \SecondLanguage}{content}
% % Possible keys: shrink
% %                         plain

%				\ifx\myshrink\undefined
%					\ifx\myplain\undefined
%						%\begin{frame} {#2}{\framelabel}
%						\begin{frame} {MyTitle}{MySubtitle}
%	%					Just normal\par
%						#3\par\phantom{-}
%						\end{frame}
%					\else	%  plain
%						\begin{frame}[plain]
%	%					plain 
%						#3\par\phantom{-}
%						\end{frame}
%					\fi
%				\else % Anyhow, shrink defined
%					\ifx\myplain\undefined
%						\begin{frame}[shrink]  {#2}{\framelabel}
%	%					shrink\par
%						#3\par\phantom{-}
%						\end{frame}
%					\else	% shrink AND plain
%						\begin{frame}[shrink,plain] 
%	%					plain shrink
%						#3\par\phantom{-}
%						\end{frame}
%					\fi %  shrink and plain
%				\fi %shrink

%			\else	% Making book
%			\begin{frame}%
%						#3
%			\end{frame}%
%			\fi
%%	\endgroup%

\newcommand{\MEDframe}[5][]{
	\ifx\UseSecondLanguage\undefined
	    \MEframe[{#1}]{#2}{#3}%
	\else
		\MEframe[{#1}]{#4}{#5}%
	\fi
}

%%% Sectioning macros 
%%% except A4 size, limit float range and start new page
%%%
\newcommand{\WEBpagebreak}{%
	\ifx\LecturePrintable\undefined
	\else
		\ifx\undefined\eBook\else\FloatBarrier\clearpage\fi
		\ifx\undefined\WEBBook\else\FloatBarrier\clearpage\fi
	\fi
}

%%% Use one level less for the slides, so 'book' format can be used
%%% Abandon error message for 'chapter' keyword in slides 
%%% These macros use 'book' terminology, and transform it properly for slides
%%% Usage: \MEchapter{short title}{long title}
%%%
\newcommand{\MEchapter}[2][]{
	\WEBpagebreak	% Start on new page for browsing
	\phantomsection	% for hyperreferencing
%	\ifx\LecturePrintable\undefined\fi
	\ifx\LecturePrintable\undefined % Making slides
		\MEDebugMessage{MESections: starting section}
			\MEframelabel{#2}%\section[#1]{#2}
			\ifthenelse{\isempty{#1}}{\section{#2}}{\section[#1]{#2}}		
	\else %Printing book
		\MEDebugMessage{MESections: starting chapter}
		\chapter[#1]{#2}\MEframelabel{#2}
	\fi
%	\ifx\LecturePrintable\undefined\else\MEframelabel{#2}\fi
}

%%% Abandon extra typing for 'section' keyword
%%% Usage: \MEsection{short title}{long title}
%%%
\newcommand{\MEsection}[2][]{
	
	\WEBpagebreak	% Start of new browser page
	\phantomsection % for hyperreferencing

	% Check whether the short title is empty
		\ifx\LecturePrintable\undefined % Making slides
			\ifthenelse{\isempty{#1}}{\subsection{#2}}{\subsection[#1]{#2}}
		\else %Printing book
			\ifthenelse{\isempty{#1}}{\section{#2}}{\section[#1]{#2}}
		\fi
	\MEframelabel{#2}
}


%% Abandone extra typing for 'subsection' keyword
%%% Usage: \MEsubsection{short title}{long title}
\newcommand{\MEsubsection}[2][]{
%	\WEBpagebreak	% Start of new browser page, usually not needed
	\phantomsection
	\ifx\LecturePrintable\undefined % Making slides
			\ifthenelse{\isempty{#1}}{\subsubsection{#2}}{\subsubsection[#1]{#2}}
	\else %Printing book
		\ifthenelse{\isempty{#1}}{\subsection{#2}}{\subsection[#1]{#2}}
	\fi
}

%%% Usage: \MEsubsubsection{short title}{long title}
\newcommand{\MEsubsubsection}[2][]{
	\phantomsection % A hook for hyper references
	\ifx\LecturePrintable\undefined % Making slides
%			\par\noindent{\bfseries #2}\par%
	\else %Printing book
		\ifthenelse{\isempty{#1}}{\subsubsection{#2}}{\subsubsection[#1]{#2}}
	\fi %Printable
}

% Abandon error message for 'chapter' keyword in slides, 2 languages
\newcommand{\MEDchapter}[4][]
{
	\ifx\UseSecondLanguage\undefined
		\MEchapter[#1]{#2}
	\else
		\MEchapter[#3]{#4}
	\fi
}
 
%%% Abandon extra typing for 'section' keyword, 2 languages
\newcommand{\MEDsection}[4][]
{
	\ifx\UseSecondLanguage\undefined
		\MEsection[#1]{#2}
	\else
		\MEsection[#3]{#4}
	\fi
}

%% Abandon extra typing for 'subsection' keyword, 2 languages
\newcommand{\MEDsubsection}[4][]
{
	\ifx\UseSecondLanguage\undefined
		\MEsubsection[#1]{#2}
	\else
		\MEsubsection[#3]{#4}
	\fi
}

%% Abandone extra typing for 'subsubsection' keyword, 2 languages
\newcommand{\MEDsubsubsection}[4][]
{
	\ifx\UseSecondLanguage\undefined
		\MEsubsubsection[#1]{#2}
	\else
		\MEsubsubsection[#3]{#4}
	\fi
}


\def\Chapterillustration{}% Define it as empty
%% Give the new illustration of chapter in memoir/fancy mode
\newcommand\MEchapterillustration[1]
{
	\ifx\DisableChapterIllustration\undefined
		\ifthenelse{\equal{#1}{}}
		{	
			\IfGraphicFileExists{fig/DefaultIllustration}
				{
				\def\Chapterillustration{fig/DefaultIllustration}}
				{\IfGraphicFileExists{fig/CoverIllustration}
				{
				\def\Chapterillustration{fig/CoverIllustration}}
				{}
				}	
		}			
		{	% The argument is not empty
		\IfGraphicFileExists{#1}
			{\renewcommand\chapterillustration{#1}
		%	\def\Chapterillustration{#1}
			}	% Explicit file present, use that file
			{\DebugMessage{File '#1' not found}
				}
		}
	\fi%\DisableChapterIllustration
}%\MEchapterillustration

	\MEDebugMessage{MESections: finished to load}
