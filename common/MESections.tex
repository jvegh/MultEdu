%% 
%% The document sectioning related macros for the MultEdu system
%%
%% The basic building block is the frame, because 'beamer' requires it

%%% A simple single-language frame for avoiding unnecessary typing
%%% Usage: \MEframe[keys and values]{frame subtitle}{content of the intended frame}
%% % Possible keys: shrink
%% %                         plain
%%% Actually, this is two implementations: one for the printed formats and one for the slides
\makeatletter
\ifx\LecturePrintable\undefined	% Making slides
\newcommand{\MEframe}[3][]
{
	\MEframelabel{#2}
	\setkeys{MEMacros}{shrink=true, plain=false, #1}% 
%	\begingroup
		\expandafter\ifstrequal\expandafter{\ME@shrink}{true}
			{\def\myshrink{yes}}{}
		\expandafter\ifstrequal\expandafter{\ME@plain}{true}
			{\def\myplain{yes}}{}
			
			\ifx\LecturePrintable\undefined	% Making slides
						\begin{frame}[shrink] {\framelabel}{}
	%					Just normal\par
	\vspace{-1\baselineskip}\par
						#3\par\phantom{-}
						\end{frame}
%				\ifx\myshrink\undefined
%					\ifx\myplain\undefined
%						%\begin{frame} {#2}{\framelabel}
%						\begin{frame} {MyTitle}{MySubtitle}
%	%					Just normal\par
%						#3\par\phantom{-}
%						\end{frame}
%					\else	%  plain
%						\begin{frame}[plain]
%	%					plain 
%						#3\par\phantom{-}
%						\end{frame}
%					\fi
%				\else % Anyhow, shrink defined
%					\ifx\myplain\undefined
%						\begin{frame}[shrink]  {#2}{\framelabel}
%	%					shrink\par
%						#3\par\phantom{-}
%						\end{frame}
%					\else	% shrink AND plain
%						\begin{frame}[shrink,plain] 
%	%					plain shrink
%						#3\par\phantom{-}
%						\end{frame}
%					\fi %  shrink and plain
%				\fi %shrink
			\else	% Making book
			\begin{frame}%
						#3
			\end{frame}%
			\fi
%	\endgroup%
	
	
	%	\begingroup%
	%	\ifx\LecturePrintable\undefined
	%		\expandafter\ifstrequal\expandafter{\ME@plain}{true}
	%			{ % It is a plain frame
	%				\expandafter\ifstrequal\expandafter{\ME@shrink}{true}
	%				{\begin{frame}[plain,shrink] {#2}{\framelabel}
	%			%	it was PLAIN  and SHRINK
	%				}
	%				{\begin{frame}[plain]{#2}{\framelabel}
	%			%	It was just PLAIN
	%				}
	%		  	}
	%		  	{ % It is not plain
	%				\expandafter\ifstrequal\expandafter{\ME@shrink}{true}
	%				{\begin{frame}[shrink]{#2}{\framelabel}
	%			%	It was ONLY shrink
	%				}
	%			  	 {\begin{frame}{#2}{\framelabel}
	%			  	% It was just empty
	%			  	 }
	%		  	}
	%		 \else
	%		 \fi
	%	\endgroup%
	%\par The key was: #1\par
	%\begin{frame}{#2}{\framelabel}
	%#3
	%\ifx\LecturePrintable\undefined\phantom{-}\fi
	%\end{frame}
} % \MEframe
\else % Making some printed version
	\newcommand{\MEframe}[3][]
	{
		\begin{frame}%{}{}
			#3
		\end{frame}
	}
\fi % \LecturePrintable

%% A dual-language frame for avoiding unnecessary typing
%% Usage: \MEDframe[keys and values]
%{frame subtitle, \LectureLanguage}{content }
%{frame subtitle, \SecondLanguage}{content}
% % Possible keys: shrink
% %                         plain

\newcommand{\MEDframe}[5][]{
	\ifx\UseSecondLanguage\undefined
	    \MEframe[{#1}]{#2}{#3}%
	\else
		\MEframe[{#1}]{#4}{#5}%
	\fi
}

%%%% The base input is formatted as 'beamer' frames
%%%% Frame specific definitions
%%%%
%% Define frame subtitle 
\ifx\LecturePrintable\undefined
%% Implementation for the slides
\newcommand{\MEframelabel}[1]{
	\def\framelabel{#1}
}
\else
%% Implementation for the printed versions
\newcommand{\MEframelabel}[1]{
	\def\framelabel{#1}
	\ifx\undefined\eBook\else \markright{#1}\fi
	\ifx\FancyBook\undefined\else\markright{#1}\fi
}
\fi

%%% Sectioning macros 
%%% except A4 size, limit float range and start new page
%%%
\newcommand{\WEBpagebreak}{%
	\ifx\LecturePrintable\undefined
	\else
		\ifx\undefined\eBook\else\FloatBarrier\clearpage\fi
		\ifx\undefined\WEBBook\else\FloatBarrier\clearpage\fi
	\fi
}

%%% Use one level less for the slides, so 'book' format can be used
%%% Abandon error message for 'chapter' keyword in slides 
%%% These macros use 'book' terminology, and transform it properly for slides
%%% Usage: \MEchapter{short title}{long title}
%%%
\newcommand{\MEchapter}[2][]{
	\WEBpagebreak	% Start on new page for browsing
	\phantomsection	% for hyperreferencing
	\ifx\LecturePrintable\undefined\MEframelabel{#2}\fi
	\ifx\LecturePrintable\undefined % Making slides
			\section[#1]{#2}
	\else %Printing book
		\chapter[#1]{#2}
	\fi
	\ifx\LecturePrintable\undefined\else\MEframelabel{#2}\fi
}

%%% Abandon extra typing for 'section' keyword
%%% Usage: \MEsection{short title}{long title}
%%%
\newcommand{\MEsection}[2][]{
	
	\WEBpagebreak	% Start of new browser page
	\phantomsection % for hyperreferencing

	% Check whether the short title is empty
		\ifx\LecturePrintable\undefined % Making slides
			\ifthenelse{\isempty{#1}}{\subsection{#2}}{\subsection[#1]{#2}}
		\else %Printing book
			\ifthenelse{\isempty{#1}}{\section{#2}}{\section[#1]{#2}}
		\fi
	\MEframelabel{#2}
}


%% Abandone extra typing for 'subsection' keyword
%%% Usage: \MEsubsection{short title}{long title}
\newcommand{\MEsubsection}[2][]{
%	\WEBpagebreak	% Start of new browser page, usually not needed
	\phantomsection
	\ifx\LecturePrintable\undefined % Making slides
			\ifthenelse{\isempty{#1}}{\subsubsection{#2}}{\subsubsection[#1]{#2}}
	\else %Printing book
		\ifthenelse{\isempty{#1}}{\subsection{#2}}{\subsection[#1]{#2}}
	\fi
}

%%% Usage: \MEsubsubsection{short title}{long title}
\newcommand{\MEsubsubsection}[2][]{
	\phantomsection % A hook for hyper references
	\ifx\LecturePrintable\undefined % Making slides
		%	\ifthenelse{\isempty{#1}}{}{\bfseries #2}
	\else %Printing book
		\ifthenelse{\isempty{#1}}{\subsubsection{#2}}{\subsubsection[#1]{#2}}
	\fi %Printable
}

% Abandon error message for 'chapter' keyword in slides, 2 languages
\newcommand{\MEDchapter}[4][]
{
	\ifx\UseSecondLanguage\undefined
		\MEchapter[#1]{#2}
	\else
		\MEchapter[#3]{#4}
	\fi
}
 
%%% Abandon extra typing for 'section' keyword, 2 languages
\newcommand{\MEDsection}[4][]
{
	\ifx\UseSecondLanguage\undefined
		\MEsection[#1]{#2}
	\else
		\MEsection[#3]{#4}
	\fi
}

%% Abandon extra typing for 'subsection' keyword, 2 languages
\newcommand{\MEDsubsection}[4][]
{
	\ifx\UseSecondLanguage\undefined
		\MEsubsection[#1]{#2}
	\else
		\MEsubsection[#3]{#4}
	\fi
}

%% Abandone extra typing for 'subsubsection' keyword, 2 languages
\newcommand{\MEDsubsubsection}[4][]
{
	\ifx\UseSecondLanguage\undefined
		\MEsubsubsection[#1]{#2}
	\else
		\MEsubsubsection[#3]{#4}
	\fi
}


%% Give the new illustration of chapter in memoir/fancy mode
\newcommand\MEchapterillustration[1]
{
	\ifx\DisableChapterIllustration\undefined
		\ifthenelse{\isempty{#1}}
		{	
			\IfFileExists{fig/DefaultIllustration.png}
				{\def\myfile{fig/DefaultIllustration.png}}
				{}	
		}			
		{	% The argument is not empty
		\IfFileExists{#1}
			{\def\myfile{#1}}	% Explicit file present, use that file
			{%\DebugMessage{File '#1' not found}
				}
		}
		
		
	\ifx\myfile\undefined
			% No file was found, exit
	\else
		\ifx\LecturePrintable\undefined
			\def\framelabel{}
			\MEframe[plain]{}
				{
					\includegraphics[width=\textwidth]{\myfile}
				}
		\else % We are printing a book
			\ifx\FancyBook\undefined
				\par\noindent\includegraphics[width=\paperwidth]{\myfile}
			\else
				\renewcommand\chapterillustration{\myfile}
			\fi
		\fi
	\fi% \myfile
	\fi%\DisableChapterIllustration
}%\MEchapterillustration
