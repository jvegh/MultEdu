%%% Basically, the following main formats can be generated from the same single source
%%% 1./ A book-like handout for the students. To produce it, define \def\LBook{YES}
%%% It has several sub-formats
%%% 		1.a/ A 'memoir-book' format: a LateX-'memoir'  based format, a really beautiful, printable, book quality format
%%% 		1.b/ A 'simple-book' format: a LateX-'book' based format,  uses simple (less fancy, but cusomizable) 'book' format
%%% 		1.c/ A 'WEB-book' format: browsable (optimized for screen appearence), also for mobile devices
%%% The another main purpose for producing slides for the lectures
%%% 2./ A presentation (slide series) for the lecturer. To produce it, undefine (i.e. comment out) \def\LBook{YES}
%%% It has several sub-formats
%%%		 2.a/ A 'beamer-slide' format: to produce beautiful slides, for presentation; using 'Beamer-LaTeX'
%%%		 2.b/ A 'WEB-slide' format, browsable (optimized for screen appearence), also for mobile devices, using AcroTeX
%%%
%%% The base format is LaTeX-Beamer, so in non-beamer case beamerarticle class is used
%%% In the book formats, extra material (relative to the slides) will also be printed
	
%% Define the lecture-specific strings, by commenting/uncommenting the option lines
%% These are the output specifications, declare them first
%\def\LectureEncoding{latin1}			% By default, it is  UTF-8

\def\LectureLanguage{magyar}
%% Define the author, title and publisher
%% Please use accents, no encoding and fonts defined
\ifx\LectureLanguage\undefined
	% printing in Hungarian
	\def\LectureAuthor{V\'egh J\'anos}
	\def\LectureTitle{Hogyan használjuk a MultEdu csomagot} 
	\def\LectureSubtitle{(Hogyan készítsünk érdekes tananyagot)} %opcionális
%	\def\LecturePublisher{Miskolci Egyetem \dots és Informatikai Kara}
\else
	% In English
	\def\LectureAuthor{J\'anos V\'egh}
	\def\LectureTitle{How to use package MultEdu}
	\def\LectureSubtitle{(How to prepare interesting and attractive teaching material)} %Optional
%	\def\LecturePublisher{Miskolc University Faculty of \dots and Informatics}
\fi %% LectureLanguage
\def\LectureRevision{V0.01 (using \MERevision) \at 2016.07.15}
\def\LectureEmail{Janos.Vegh\at unideb.hu}
	
	
%% Define the bibliography file
%% By default, no file assumed. Allows for language-dependent bibliographies
\ifx\LectureLanguage\undefined
\def\LectureBibliography{src/Bibliography}
\else

\IfFileExists{src/Bibliographyhu}
{\def\LectureBibliography{src/Bibliography,src/Bibliographyhu}}
{\def\LectureBibliography{src/Bibliography}}
\fi

	
