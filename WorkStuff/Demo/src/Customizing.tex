
%%% This is the chapter about customizing output documents for the MultEdu system

\MEDchapter[Customizing]{Customizing document}{Méretre alakítás}{Méretre alakítás}


\MESetListingFormat{TeX}
\lstset{ basicstyle=\ttfamily\color{black}\normalsize}


\MEDframe{Customizing document}
{
The MultEdu system works perfectly with its default settings, but it cannot read your mind.
The settings can be changed using definitions of form \lstinline|\\def\{\\xxx\}|.
The place where the settings can be changed, depends on the compilation mode,
as described in chapter~\ref{sec:compiling}. The default values of the settings
is given at the individual settings.
The sections in this chapter provide a detailed description of the possible settings.
}
{Méretre alakítás}
{
A MultEdu rendszer tökéletesen működik alapértelmezett beállításokkal is,
de nem gondolatolvasó. A beállításokat \lstinline|\\def\{\\xxx\}| formájú definíciókkal
lehet megváltoztatni. A beállítások helye üzemmódtól függ,
a részleteket lásd ~\aref{sec:compiling} szakaszban. 
Az alapértelmezett beállítások az egyes beállítások hatásának részletes 
leírásánál találhatók.
A fejezet következő szakaszai 
az üzemmódok használatát mutatja be.
}


\MEDsection[Defaults]{Default settings}{Alap beállítások}{Alap beállítások}

\MEDframe{Facilities to set options for MultEdu}
{
The MultEdu system can interpret as an intention to change the default
behavior either the presence of a file at a predefined file with a predefined name,
or the thesence of definition of form \lstinline|\\def\{Option\{Value\}\}|.
In the absence of such occurrences, Multedu uses the default settings 
when generating the output file.
}
{A MultEdu csomag beállítási lehetőségei}
{
A MultEdu beállítási lehetőségként vagy fájlok megadott helyen és néven 
való előfordulását, vagy pedig \lstinline|\\def\{Option\{Value\}\}| formájú definiciók előfordulását tudja értelmezni. Ezek hiánya esetén az alapértelmezett
viselkedés lét életbe az eredmény fájl előállítása során. 

A beállítási lehetőségek lehetnek kötelezően használandók;
az eredmény fájlt nagy mértékben befolyásolók, vagy csak kisebb
finomítást jelentők;
csak bizonyos típusú eredmény fájl készítésekor hatásosak.
}


\MEDsection[Options]{Options for using package MultEdu}{Beállítások}{A MultEdu csomag beállítási lehetőségei}


\MEDsubsection[Beamer]{Options for Beamer-based formats}{Beamer}{Beamer alapú formátum beállítások}

\MEDframe{Screen width}
{
Multedu allows to utilize two popular screen widths.
The default is the spreading format with aspect ratio 16:9.
To set ratio 4:3, use
\noindent\vskip-\parskip\lstinline|\{\\def\\DisableWideScreen\{YES\}\}|\par
}
{Képernyő szélesség}
{
A Multedu lehetővé teszi kétféle elterjedt formátum használatát.
Egyre gyakoribb 16:9  arányú képformátum így az az alap beállítás.
A 4:3 arányú képformátumot a 
\noindent\vskip-\parskip\lstinline|\{\\def\\DisableWideScreen\{YES\}\}|\par
definiálásával lehet beállítani.
}

\MEDframe{Table of contents}
{
Sometimes (mainly in the case of short presentations) the table of contents is not necessary at all.
It can be disabled through defining 
\noindent\vskip-\parskip\lstinline|\{\\def\\DisableTOC\{YES\}\}|\par
It might also happen, that chapter-level TOC is still needed, but the section level not.
This can be reached through defining
\noindent\vskip-\parskip\lstinline|\{\\def\\DisableSectionTOC\{YES\}\}|\par
}
{Tartalomjegyzék}
{
Néha (főként rövid bemutatók esetén) egyáltalán nincs szükség tartalomjegyzékre.
Ezt a 
\noindent\vskip-\parskip\lstinline|\{\\def\\DisableTOC\{YES\}\}|\par
definiálásával lehet elérni. Az is előfordul, hogy a fejezet-szintű
tartalomjegyzék még szükséges, de a szakasz szintű már nem. Ezt a
\noindent\vskip-\parskip\lstinline|\{\\def\\DisableSectionTOC\{YES\}\}|\par
definiálásával lehet elérni.
}


\MEDsection[Files]{Files for package MultEdu}{Fájlok}{A MultEdu csomag fájljai}

\MEDframe{Location of the files}
{
The files affecting the appearance of your documents must fit the 
overall structure of files, as described in section~\ref{sec:subdirectories}.
It is a good policy to change files only in your project subdirectory,
since the commonly used files of the package are overwritten when using batch compile.
}
{A fájlok helye}
{
A használt fájloknak illeszkedni kell a fájlok általános rendszerébe,
lásd \ref{sec:subdirectories} szakasz.
Tanácsos csak a projekt könyvtárba tartozó fájlokat változtatni,
mivel a csomag közösen használt fájljai a kötegelt feldolgozás során
felülíródnak.
}

\MEDsubsection[Heading]{\lstinline{src/Heading}}{Alapértelmezett}{Alapértelmezett}

\MEDframe{File \ctext{Heading}}
{
Some kind of heading usually belongs to the document.
As an example see file \ctext{src/Heading.tex} of this user's guide.
%\begin{lstlisting}
%\ifthenelse{\equal{\LectureLanguage}{english}}
%  { % In English
%	\def\LectureAuthor{J\'anos V\'egh}
%	\def\LectureTitle{How to use package MultEdu}
%	\def\LectureSubtitle{(How to prepare interesting and attractive teaching material)}
%	 %Optional
%%	\def\LecturePublisher{Miskolc University Faculty of \dots and Informatics}
%	\def\LectureRevision{V\Version\ (using \MERevision) \at 2016.08.19}
%  }% true
%  {% NOT english
%  }
%\end{lstlisting}
}
{A \ctext{Heading} file}
{
A dokumentumokhoz tartozik néhány fejzet leíró definíció.
Mintaként a felhasználói leírás \ctext{src/Heading.tex} fájlja szolgál.
%  \ifthenelse{\equal{\LectureLanguage}{magyar}}
%  {	% in Hungarian
%	\def\LectureAuthor{V\'egh J\'anos}
%	\def\LectureTitle{Hogyan haszn\'aljuk\\ a MultEdu csomagot} 
%	\def\LectureSubtitle{(Hogyan k\'esz\'\i{}ts\"unk \'erdekes\\ \'es vonz\'o tananyagot)} %opcionális
%%	\def\LecturePublisher{Miskolci Egyetem \dots és Informatikai Kara}
%	\def\LectureRevision{V\Version\ (a \MERevision\ felhaszn\'al\'as\'aval) \at 2016.08.19}
%  }% true
%  {% NOT magyar
%  }

}
\index{Heading.texx}
\index{src/Heading.tex}


\MEDframe{File \ctext{Heading}}
{
Line \lstinline|\\def\\LectureAuthor\{J\\'anos V\\'egh\}| defines the author,
lines \lstinline|\\def\\LectureTitle\{How to use package MultEdu\}|
and \lstinline|\\def\\LectureSubtitle\{(How to prepare interesting and attractive teaching material)\}| the main title and its subtitle.
Also a university name or conference name can be defined in \lstinline|\\def\\LecturePublisher\{University or conference\}| line.
It is good practice to define \lstinline|\\def\\LectureRevision\{V\\Version\\ (using \\MERevision) \\at year.mm.dd\}|, too.

}
{A \ctext{Heading} file}
{
\ao{A fejzetet olyan fázisban olvassa a program, amikor még nem használhatók
a magyar ékezetes betűk, ezért azokat a szokásos \LaTeX{} kódolással kell írni.}
A fejzet tartalma:

A \lstinline|\\def\\LectureAuthor\{V\\'egh J\\'anos\}| sor adja meg a szerzőt,
a \lstinline|\\def\\LectureTitle\{Hogyan haszn\\'aljuk\\\\ a MultEdu csomagot\}|
a címét, a \lstinline|\\def\\LectureSubtitle\{(Hogyan k\\'esz\\'\\i\{\}ts\\"unk \\'erdekes\\\\ \\'es vonz\\'o tananyagot)\}| pedig a dokumentum címét és alcímét.
Megadhatunk egy \lstinline|\\def\\LecturePublisher\{Egyetem neve vagy konferencia neve\}| meghatározást is.
Javasolt egy \lstinline|\\def\\LectureRevision\{V\\Version\\ (a \\MERevision\\ felhaszn\\'al\\'as\\'aval) \\at year.mm.dd\}| formájú sor használata is

}

\MEDframe{File \ctext{Heading}}
{ When using dual-language source files, one has to prepare the source
in a form which allows to select source lines depending on the language.
To prepare dual-language documents, the definitions should be put in frame 
like 
\noindent\lstinline|\\ifthenelse\{\\equal\{\\LectureLanguage\}\{english\}\}|\par
\noindent\vskip-\parskip\lstinline|  \{\% in English|\par
\noindent\vskip-\parskip\lstinline|  \}\% true|\par
\noindent\vskip-\parskip\lstinline|  \{\% NOT english|\par
\noindent\vskip-\parskip\lstinline|  \}|\par
}
{A \ctext{Heading} file}
{ Kétnyelvű dokumentumok készítéséhez a fentieket
\noindent\lstinline|\\ifthenelse\{\\equal\{\\LectureLanguage\}\{magyar\}\}|\par
\noindent\vskip-\parskip\lstinline|  \{\% in Hungarian|\par
\noindent\vskip-\parskip\lstinline|  \}\% true|\par
\noindent\vskip-\parskip\lstinline|  \{\% NOT magyar|\par
\noindent\vskip-\parskip\lstinline|  \}|\par
blokkban kell elhelyezni.
}

\MEDframe{File \ctext{Heading}}
{
Also here you can give e-mail address

\lstinline|\\def\\LectureEmail\{Janos.Vegh\\at unideb.hu\}|

Furthermore, one can provide  \ctext{BibTeX},
even conditionally, depending on the language or the presence of some files

\noindent\vskip-\parskip\lstinline|\\IfFileExists\{src/Bibliographyhu\}|\par
\noindent\vskip-\parskip\lstinline|\{\\def\\LectureBibliography\{src/Bibliography
,src/Bibliographyhu\}\}|\par
\noindent\vskip-\parskip\lstinline|\{\\def\\LectureBibliography\{src/Bibliography\}\}|\par
 
}
{A \ctext{Heading} file}
{
Megadhatunk számítógépes címet is

\lstinline|\\def\\LectureEmail\{Janos.Vegh\\at unideb.hu\}|

Ugyancsak itt célszerű megadni a dokumentumban használt \ctext{BibTeX} fájlokat, akár a nyelv, vagy a fájl tényleges fellelhetősége alapján:

\noindent\vskip-\parskip\lstinline|\\IfFileExists\{src/Bibliographyhu\}|\par
\noindent\vskip-\parskip\lstinline|\{\\def\\LectureBibliography\{src/Bibliography, src/Bibliographyhu\}\}|\par
\noindent\vskip-\parskip\lstinline|\{\\def\\LectureBibliography\{src/Bibliography\}\}|\par
 
}
