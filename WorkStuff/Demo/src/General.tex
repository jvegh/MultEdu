%%%
%%% This is the chapter about sectioning documents for the MultEdu system

\MEDchapter[General]{General information}{Általános}{Általános információ}


\MESetListingFormat{TeX}
\lstset{ basicstyle=\ttfamily\color{black}\normalsize}


\MEDsection[Introduction]{Introduction}{Bevezetés}{Bevezetés}

\MEDframe{Introduction}
{
\ao{For teaching my courses I needed to develop course material, in different forms of appearance;
the present package is a by-product of this activity.
Good course materials develop quickly, especially, if the field of science in question itself is renewed daily.
In informatics, the technology, the statistics, the products, the tools, etc. change year by year,
and this alone is a good reason to renew the teaching material for a new semester.}

The todays education needs the course material in various forms: in the lecture room for the projected 
picture well organized text with many pictures are needed, which also serve as a good guide for the lecturer, too.
To prepare for the exams, the explanation provided by the lecturer when projecting the slides is also needed.
\ao{That means, the course material should be available in printable and browsable form, as well as for mobile devices.
A frequent case is that the same material shall be provided in foreign language(s), for foreign students.
In many cases one can rely to good textbooks, but for the more specialized courses this course material
serves as a basic tool for preparing to the exams.}

\ao{The present macro package is developed for my own purposes, which I attempted to develop in such a way,
that when developing course material, one should not deal with the technology of displaying.
In this way also others can use the package, when they follow the rules. The package is quite good
in some fields, at some point I needed to make a bargain between the different needs, not perfect
in some points, and of course in much more aspects I did not have the time to develop features.}

The present document is a demo and test at the same time. It attempts to describe the many features,
and also tests if the features really work. Because of the many features, and their interference,
this job needs a lot of work and time, so the documentation does not always match the actual features,
especially in this initial phase.
}
{Bevezetés}
{
\ao{Kurzusaim tartásához saját tananyagot fejlesztettem, különböző megjelenési formákban; a jelen csomag ennek mellékterméke.
A jó kurzusok tananyaga gyorsan fejlődik, különösen akkor, ha maga a tudományág is naponta megújul.
Az informatikában évről évre változik a technológia, a statisztikák, a termékek, a segédeszközök, stb.;
és már csak emiatt is minden tanévre frissíteni kell a tananyagot.}

Manapság a tananyagot a hallgatóság változatos formákban igényli: előadáson nagy méretű, jól áttekinthető,
kivetíthető anyagot kell használni, amely képekkel gazdagon illusztrált és az előadó számára is jó sorvezetőként szolgál.
A vizsgára készüléshez pedig arra a magyarázatra is szükség van, amit az előadó a kivetített anyaghoz élő szóban hozzáfűz.
\ao{Azaz, olyan magyarázó szöveggel ellátott anyagra is szükség van, amelyet kinyomtatva, asztali gép vagy mobil eszköz
képernyőjén lehet elolvasni. Esetleg ugyanazt a változatot idegen nyelven is közzétenni, külföldi hallgatók számára.
Bár sokszor lehet elérhető könyvekre és megvásárolható jegyzetekre hagyatkozni, a kicsit is speciálisabb anyagok
esetén ez a segédlet lesz a felkészüléshez szükséges tananyag.}

\ao{A jelen makró csomag olyan, amilyet saját kurzusaim készítéséhez fejlesztettem, és igyekeztem olyanná tenni, hogy
tananyag fejlesztés közben már ne kelljen a megjelenítés technikájával foglalkozni, és ilyen módon
mások is tudják hasonló célra felhasználni, ha követik a fejlesztés logikáját.  A csomag bizonyos vonatkozásokban 
egészen jó, néhol tudatosan kompromisszumot kellett kötni a sokféle igény között, néhol még nem tökéletes, 
és sok vonatkozásban még nem jutott időm további tulajdonságok fejlesztésére.}

A jelen dokumentum egyúttal bemutató és tulajdonság tesztelő is. A dokumentum megkísérli bemutatni, mit hogyan 
kell és lehet használni, egyúttal azt is megvizsgálva, hogy tényleg működik-e az elvárt módon. A sokféle tulajdonság
és különösen azok kölcsönhatása miatt sok munkát és időt igényel a fejlesztés, ezért a tényleges tulajdonságok
nem mindig egyeznek meg a dokumentációval, különösen a kezdeti fázisban. 


}

\MEDframe{Introduction}
{
The macro package can be used at (at least) three different levels.
Even the lowest level assumes some familiarity  with \LaTeX.
At the very basic level, you might just take the package,
replace and modify files in the distribution.
At the advanced level (this assumes reading the User's manual \smiley)
the user learns the facilities provided in the package,
and prepares his/her courses actively using those facilities.
Power users might add their own macros (preferably uploaded to the distribution), i.e. take part in the development.
}
{Bevezetés}
{
A makró csomag (legalább) három különböző felhasználói szinten 
alkalmazható. Már a legalacsonyabb szinten is szükségesek a \LaTeX-re
vonatkozó elemi ismeretek.
Az alap szinten a felhasználó egyszerűen csak helyettesíti és módosítja a rendszert bemutató dokumentumokat.
Haladó szinten (ehhez már el kell olvasni a felhasználói leírást is \smiley) megtanulja a csomagban található makrók által biztosított lehetőségeket, és azokat aktívan használva fejleszti dokumentumait.
Tapasztalt felhasználóként saját makrókat is készíthet (jó, ha azokat
a letölthető anyaghoz hozzáadja), azaz aktívan részt vesz a fejlesztésben.
}

\MEDsection[Installing]{Installing and utilizing MultEdu}{Beüzemelés}{A MultEdu beüzemelése és használata}

\MEDframe{Installing MultEdu}
{
Multedu, as any package based on \LaTeX, assumes that the user has
experiences with using \LaTeX. I.e. some \LaTeX{} distribution must already be installed on the system of the user. If you want to use the
batch processing facility, the CMake system must also be installed.

For the simplicity of utilization and starting up, 
the best way is to create a main directory for your family of projects and a subdirectory for your first project, 
as described below. The quickest way is to copy \lstinline|./Workstuff| (after deleting and renaming some files)
and to prepare your own "Hello World" program.
Making minor changes to that source you may experience
some features of the package. Then, it is worth at least to skim the user's manual, to see what features you need. After that, you may start your own development.
At the beginning text only, later you can learn the advanced
possibilities.
\ao{Do not forget: LaTeX is hard, it needs accurate coding,
and so is MultEdu, too.
Frequent saving and using versioning can help a lot.}
}
{A MultEdu beüzemelése}
{
A MultEdu (mint minden \LaTeX{} alapú rendszer) feltételezi, hogy 
a felhasználó már rendelkezik tapasztalatokkal a \LaTeX{}  használatában.
Azaz, a felhasználó rendszerén már működnie kell valamilyen \LaTeX{}
rendszernek.

Az egyszerű használat és a gyors elindulás érdekében célszerű a 
lentebb megadott módon saját projekt csoportjainak egy főkönyvtárat és
azon belül az egyes projekteknek alkönyvtárakat létrehozni.
A leggyorsabb magát a \lstinline|./Workstuff| könyvtárat (a megfelelő átnevezésekkel és törlésekkel) lemásolni, és csekély 
módosításokkal elkészíteni saját 'Helló Világ' programját.
Ezután érdemes legalább átlapozni a
felhasználói kézikönyvet, ami után már elkezdheti saját fejlesztését.
Eleinte csak szöveget, aztán sorjában megtanulni a használni kívánt tulajdonságok programozását.
\ao{Ne feledje: a LaTeX nehéz nyelv, pontos kódolást igényel,
és ezért ilyen a MultEdu is. A gyakori mentések és a verziókövető rendszerek használata nagy segítséget jelentenek.}
}

\MEDsection[Structure]{Structure of MultEdu}{Szerkezet}{A MultEdu könyvtár szerkezete}

\MEDframe{Directory structure}
{
The MultEdu system is assumed to be used with the directory structure below.
It comes with two main subdirectories: \lstinline|./common| comprises all files of
the MultEdu system, and  \lstinline|./Workstuff| models the users subdirectory structure.
\par\vskip-\parskip\noindent\lstinline*.*
\par\vskip-\parskip\noindent\lstinline*|-- common*
\par\vskip-\parskip\noindent\lstinline*|-- WorkStuff*

You may add your project groups stuff like
\par\vskip-\parskip\noindent\lstinline*.*
\par\vskip-\parskip\noindent\lstinline*|-- Exams*
\par\vskip-\parskip\noindent\lstinline*|-- Labs*
\par\vskip-\parskip\noindent\lstinline*|-- Lectures*
\par\vskip-\parskip\noindent\lstinline*|-- Papers*

which directories have a subdirectory structure similar to that of \lstinline*|-- WorkStuff*
}
{Könyvtár szerkezet}
{
A MultEdu rendszert az alábbi könyvtár szerkezetben célszerű használni.
Két fő könyvtára: a \lstinline|./common|, amely tartalmazza a MultEdu összes fájlját,
 és a \lstinline|./Workstuff|, amely a felhasználói könyvtár szerkezetet modellezi.
\par\vskip-\parskip\noindent\lstinline*.*
\par\vskip-\parskip\noindent\lstinline*|-- common*
\par\vskip-\parskip\noindent\lstinline*|-- WorkStuff*

A felhasználói projekt csoportokat ilyen szerkezetben érdemes hozzáadni:
\par\vskip-\parskip\noindent\lstinline*.*
\par\vskip-\parskip\noindent\lstinline*|-- Exams*
\par\vskip-\parskip\noindent\lstinline*|-- Labs*
\par\vskip-\parskip\noindent\lstinline*|-- Lectures*
\par\vskip-\parskip\noindent\lstinline*|-- Papers*

amely könyvtáraknak a \lstinline*|-- WorkStuff* könyvtárhoz
hasonló belső alkönyvtárai vannak
}

\MEDsubsection[\lstinline*common*]{Subdirectory \lstinline*common*}{\lstinline*common*}{A \lstinline*common* alkönyvtár}

\MEDframe{Directory structure of subdirectory \lstinline*common*}
{
Subdirectory \lstinline|./common| comprises some special subsubdirectories
and general purpose macro files. \ao{MultEdu attempts to be as user-friendly as possible: it uses default settings, files, images, etc., to allow a quick start for a new development.}
\par\vskip-\parskip\noindent\lstinline*.*
\par\vskip-\parskip\noindent\lstinline*|-- common*
\par\vskip-\parskip\noindent\lstinline*|    |-- defaults*
\par\vskip-\parskip\noindent\lstinline*|    |-- formats*
\par\vskip-\parskip\noindent\lstinline*|    |-- images*

Subsubdirectory \lstinline|./defaults| contains some default text, like copyright.
\ao{In general, if the user does not provide its own elements, MultEdu uses the defaults instead
(provided that using and presenting it is not disabled, see later.)}

Subsubdirectory \lstinline|./formats| contains the possible format specification macros,
here you can add your own format macros.

Subsubdirectory \lstinline|./images| contains some images, partly the ones which are used 
as defaults.
}
{A \lstinline*common* alkönyvtár szerkezete}
{
A \lstinline|./common| különleges célú al-alkönyvtárakat, valamint általános célú makró fájlokat tartalmaz.  \ao{A Multedu megpróbál a lehető legbarátságosabb lenni:
alapértelmezett beállításokat, fájlokat, képeket, stb használ, hogy gyorsan el 
lehessen kezdeni egy új fejlesztést.}
\par\vskip-\parskip\noindent\lstinline*.*
\par\vskip-\parskip\noindent\lstinline*|-- common*
\par\vskip-\parskip\noindent\lstinline*|    |-- defaults*
\par\vskip-\parskip\noindent\lstinline*|    |-- formats*
\par\vskip-\parskip\noindent\lstinline*|    |-- images*

A \lstinline|./defaults| al-alkönyvtár olyan alapértelmezett szöveget tárol, mint 
a szerzői jogok.
\ao{Alapértelmezetten, ha a felhasználó nem adja meg saját dokumentum elemeit,
a MultEdu automatikusan az alapértelmezetteket használja helyettük
(feltéve, hogy azok használata nincs megtiltva, lásd később).
}

A \lstinline|./formats| al-alkönyvtár tartalmazza a formátumokat meghatározó makrókat\ao{;
itt adhatja hozzá a felhasználó esetleges saját formátum leíró makróit.}

Az \lstinline|./images|  al-alkönyvtár képeket tartalmaz\ao{, amelyek egy része
alapértelmezett képként használatos.}
}

\MEDsubsection[\lstinline*Workstuff*]{Subdirectory \lstinline*Workstuff*}{\lstinline*Workstuff*}{A \lstinline*Workstuff* alkönyvtár}
\MEDframe{Files in subdirectory \lstinline|Workstuff|}
{
Subdirectory \lstinline|./Workstuff| contains the files of the present demo,
and serves as an example of using the system (a kind of User's Guide). It contains a sample project \lstinline|./Workstuff/Demo|, which has three main files.
\par\vskip-\parskip\noindent\lstinline*|-- WorkStuff*
\par\vskip-\parskip\noindent\lstinline*|    |-- Demo*
\par\vskip-\parskip\noindent\lstinline*|    .    |-- CMakeLists.txt*
\par\vskip-\parskip\noindent\lstinline*|    .    |-- Demo.tex*
\par\vskip-\parskip\noindent\lstinline*|    .    |-- Main.tex*

The real main source file is \lstinline|Main.tex|, and \lstinline|Demo.tex| is a lightweight envelope to it. 
(if you want to use UseLATEX, you need to use the file with name \lstinline|Main.tex|, the envelop must be concerted with the CMakeLists.txt file)
}
{A \lstinline|Workstuff| alkönyvtár fájljai}
{
A \lstinline|./Workstuff|  al-alkönyvtár tartalmazza  (a példa programként is szolgáló)
felhasználói leírás fájljait. Egy olyan \lstinline|./Workstuff/Demo| projektet tartalmaz, amelyik (a saját főkönyvtárában)
három fájlból áll. 
\par\vskip-\parskip\noindent\lstinline*|-- WorkStuff*
\par\vskip-\parskip\noindent\lstinline*|    |-- Demo*
\par\vskip-\parskip\noindent\lstinline*|    .    |-- CMakeLists.txt*
\par\vskip-\parskip\noindent\lstinline*|    .    |-- Demo.tex*
\par\vskip-\parskip\noindent\lstinline*|    .    |-- Main.tex*

A valódi főprogram  \lstinline|Main.tex|, és ehhez készült egy \lstinline|Demo.tex| nagyon egyszerű boríték. 
Ha használja a UseLATEX csomagot, a  \lstinline|Main.tex| file
használata (ezzel a névvel) kötelező, a boríték fájl nevét pedig a CMakeLists.txt fájllal egyeztetni kell.}

\MEDframe{Subsubdirectories in \lstinline|./Workstuff|}
{
\ao{Subdirectory \lstinline|./Workstuff| has some subsubdirectories, for different goals.}
\par\vskip-\parskip\noindent\lstinline*|-- WorkStuff*
\par\vskip-\parskip\noindent\lstinline*|    |-- Demo*
\par\vskip-\parskip\noindent\lstinline*|    .    |-- build*
\par\vskip-\parskip\noindent\lstinline*|    .    .    .   |-- build*
\par\vskip-\parskip\noindent\lstinline*|    .    |-- dat*
\par\vskip-\parskip\noindent\lstinline*|    .    |-- fig*
\par\vskip-\parskip\noindent\lstinline*|    .    |-- lst*
\par\vskip-\parskip\noindent\lstinline*|    .    |-- src*

The file  \lstinline|Main.tex| inputs files in the sub-subdirectories.


Subsubdirectory 
\par\vskip-\parskip\noindent\lstinline*|   .   |-- src*  is the place for the user's source files, 
\par\vskip-\parskip\noindent\lstinline*|   .   |-- fig* for the images. 
\par\vskip-\parskip\noindent\lstinline*|   .   |-- lst* for the program source files,
\par\vskip-\parskip\noindent\lstinline*|   .   |-- dat* for the other data \ao{(like tabulated data for pgfplot or TikZ figures source code)}. 
 
\ao{You may use (and handle!, especially in CMakeLists.txt) further subsubdirectories.}
}
{A \lstinline|./Workstuff| al-alkönyvtárai}
{
\ao{A \lstinline|./Workstuff| al-alkönyvtárai különböző célokat szolgálnak.}
Célszerű a felhasználói projekt könyvtárakat is hasonlóan berendezni.
\par\vskip-\parskip\noindent\lstinline*|-- WorkStuff*
\par\vskip-\parskip\noindent\lstinline*|    |-- Demo*
\par\vskip-\parskip\noindent\lstinline*|    .    |-- build*
\par\vskip-\parskip\noindent\lstinline*|    .    .    .   |-- build*
\par\vskip-\parskip\noindent\lstinline*|    .    |-- dat*
\par\vskip-\parskip\noindent\lstinline*|    .    |-- fig*
\par\vskip-\parskip\noindent\lstinline*|    .    |-- lst*
\par\vskip-\parskip\noindent\lstinline*|    .    |-- src*

A fő  \lstinline|Main.tex| menet közben magába olvassa az alkönyvtárakban levő egyéb fájlokat.
 
\par\vskip-\parskip\noindent\lstinline*|   .   |-- src*  tartalmazza a felhasználó forráskód fájljait, 
\par\vskip-\parskip\noindent\lstinline*|   .   |-- fig* a képeit, 
\par\vskip-\parskip\noindent\lstinline*|   .   |-- lst* a programlisták forrás kódját,
\par\vskip-\parskip\noindent\lstinline*|   .   |-- dat* a többi adatot \ao{(például
táblázatok, adatok a pgfplot vagy kód a TikZ ábrák számára)}. 
 
\ao{További alkönyvtárak is készíthetők, de azokat a felhasználónak kell kezelni, és módosítania kell a CMakeLists.txt fájlt is.}
}

\MEDframe{Subsubdirectories and files for CMake}
{
It is also possible to use CMake package UseLATEX for compiling
your text to different formats and languages in batch mode;
producing the documents in different languages and formats in one single step.
File \lstinline|CMakeLists.txt| serves for that goal.

Subsubdirectories \par\vskip-\parskip\noindent\lstinline*|-- build* and \par\vskip-\parskip\noindent\lstinline*|   .  .  |-- build*
\par\vskip-\parskip\noindent are only needed
if using CMake\ao{; they contain temporary files created during processing.
The system also makes its own copy of the subdirectory \lstinline|common| in 
your project directory (corresponding to subdirectory \lstinline|Demo|). Those files can be deleted any time: when compiling,
CMake will regenerate them}.
}
{A CMake al-alkönyvtárai és fájljai}
{
A CMake rendszeren keresztül a UseLATEX csomag is használható arra,
hogy egy szerkesztés után, a kötegelt feldolgozási módot használva, egyetlen lépésben elő lehessen állítani a forrásnyelvi fájlból a különböző nyelvű és formátumú dokumentumokat; erre való a  \lstinline|CMakeLists.txt| fájl.

A \par\vskip-\parskip\noindent\lstinline*|-- build* és \par\vskip-\parskip\noindent\lstinline*|   .  .  |-- build*
\par\vskip-\parskip\noindent alkönyvtárak csak akkor kellenek
ha a CMake rendszert használjuk\ao{; ezek a feldolgozás során szükséges 
átmeneti fájlokat tartalmazzák. A rendszer készít a projekt könyvtárába (ami a \lstinline|Demo| alkönyvtárnak felel meg 
egy saját másolatot a  \lstinline|common| alkönyvtárról.
Ezek a fájlok bármikor törölhetők: amikor fordít, a CMaker újra generálja azokat}. 
}


\MEDsection[Defaults]{Default files for package MultEdu}{Alapértelmezett}{A MultEdu csomag alapértelmezett fájljai}

\MEDsubsection[Heading]{\ctext{Heading}}{Alapértelmezett}{Alapértelmezett}

\MEDframe{File \ctext{Heading}}
{
Some kind of heading usually belongs to the document.
As an example see file \ctext{src/Heading.tex} of this user's guide.
%\begin{lstlisting}
%\ifthenelse{\equal{\LectureLanguage}{english}}
%  { % In English
%	\def\LectureAuthor{J\'anos V\'egh}
%	\def\LectureTitle{How to use package MultEdu}
%	\def\LectureSubtitle{(How to prepare interesting and attractive teaching material)}
%	 %Optional
%%	\def\LecturePublisher{Miskolc University Faculty of \dots and Informatics}
%	\def\LectureRevision{V\Version\ (using \MERevision) \at 2016.08.19}
%  }% true
%  {% NOT english
%  }
%\end{lstlisting}
}
{A \ctext{Heading} file}
{
A dokumentumokhoz tartozik néhány fejzet leíró definíció.
Mintaként a felhasználói leírás \ctext{src/Heading.tex} fájlja szolgál.
%  \ifthenelse{\equal{\LectureLanguage}{magyar}}
%  {	% in Hungarian
%	\def\LectureAuthor{V\'egh J\'anos}
%	\def\LectureTitle{Hogyan haszn\'aljuk\\ a MultEdu csomagot} 
%	\def\LectureSubtitle{(Hogyan k\'esz\'\i{}ts\"unk \'erdekes\\ \'es vonz\'o tananyagot)} %opcionális
%%	\def\LecturePublisher{Miskolci Egyetem \dots és Informatikai Kara}
%	\def\LectureRevision{V\Version\ (a \MERevision\ felhaszn\'al\'as\'aval) \at 2016.08.19}
%  }% true
%  {% NOT magyar
%  }

}


\MEDframe{File \ctext{Heading}}
{
Line \lstinline|\\def\\LectureAuthor\{J\\'anos V\\'egh\}| defines the author,
lines \lstinline|\\def\\LectureTitle\{How to use package MultEdu\}|
and \lstinline|\\def\\LectureSubtitle\{(How to prepare interesting and attractive teaching material)\}| the main title and its subtitle.
Also a university name or conference name can be defined in \lstinline|\\def\\LecturePublisher\{University or conference\}| line.
It is good practice to define \lstinline|\\def\\LectureRevision\{V\\Version\\ (using \\MERevision) \\at 2016.08.19\}|, too.

}
{A \ctext{Heading} file}
{
\ao{A fejzetet olyan fázisban olvassa a program, amikor még nem használhatók
a magyar ékezetes betűk, ezért azokat a szokásos \LaTeX{} kódolással kell írni.}
A fejzet tartalma:

A \lstinline|\\def\\LectureAuthor\{V\\'egh J\\'anos\}| sor adja meg a szerzőt,
a \lstinline|\\def\\LectureTitle\{Hogyan haszn\\'aljuk\\\\ a MultEdu csomagot\}|
a címét, a \lstinline|\\def\\LectureSubtitle\{(Hogyan k\\'esz\\'\\i\{\}ts\\"unk \\'erdekes\\\\ \\'es vonz\\'o tananyagot)\}| pedig a dokumentum címét és alcímét.
Megadhatunk egy \lstinline|\\def\\LecturePublisher\{Egyetem neve vagy konferencia neve\}| meghatározást is.
Javasolt egy \lstinline|\\def\\LectureRevision\{V\\Version\\ (a \\MERevision\\ felhaszn\\'al\\'as\\'aval) \\at 2016.09.19\}| formájú sor használata is

}

\MEDframe{File \ctext{Heading}}
{ When using dual-language source files, one has to prepare the source
in a form which allows to select source lines depending on the language.
To prepare dual-language documents, the definitions should be put in frame 
like 
\noindent\lstinline|\\ifthenelse\{\\equal\{\\LectureLanguage\}\{english\}\}|\par
\noindent\vskip-\parskip\lstinline|  \{\% in English|\par
\noindent\vskip-\parskip\lstinline|  \}\% true|\par
\noindent\vskip-\parskip\lstinline|  \{\% NOT english|\par
\noindent\vskip-\parskip\lstinline|  \}|\par
}
{A \ctext{Heading} file}
{ Kétnyelvű dokumentumok készítéséhez a fentieket
\noindent\lstinline|\\ifthenelse\{\\equal\{\\LectureLanguage\}\{magyar\}\}|\par
\noindent\vskip-\parskip\lstinline|  \{\% in Hungarian|\par
\noindent\vskip-\parskip\lstinline|  \}\% true|\par
\noindent\vskip-\parskip\lstinline|  \{\% NOT magyar|\par
\noindent\vskip-\parskip\lstinline|  \}|\par
blokkban kell elhelyezni.
}

\MEDframe{File \ctext{Heading}}
{
Also here you can give e-mail address

\lstinline|\\def\\LectureEmail\{Janos.Vegh\\at unideb.hu\}|

Furthermore, one can provide  \ctext{BibTeX},
even conditionally, depending on the language or the presence of some files

\noindent\vskip-\parskip\lstinline|\\IfFileExists\{src/Bibliographyhu\}|\par
\noindent\vskip-\parskip\lstinline|\{\\def\\LectureBibliography\{src/Bibliography
,src/Bibliographyhu\}\}|\par
\noindent\vskip-\parskip\lstinline|\{\\def\\LectureBibliography\{src/Bibliography\}\}|\par
 
}
{A \ctext{Heading} file}
{
Megadhatunk számítógépes címet is

\lstinline|\\def\\LectureEmail\{Janos.Vegh\\at unideb.hu\}|

Ugyancsak itt célszerű megadni a dokumentumban használt \ctext{BibTeX} fájlokat, akár a nyelv, vagy a fájl tényleges fellelhetősége alapján:

\noindent\vskip-\parskip\lstinline|\\IfFileExists\{src/Bibliographyhu\}|\par
\noindent\vskip-\parskip\lstinline|\{\\def\\LectureBibliography\{src/Bibliography, src/Bibliographyhu\}\}|\par
\noindent\vskip-\parskip\lstinline|\{\\def\\LectureBibliography\{src/Bibliography\}\}|\par
 
}



\MEDsection[Options]{Options for using package MultEdu}{Beállítások}{A MultEdu csomag beállítási lehetőségei}

\MEDframe{Options for using package MultEdu}
{
a
}
{A MultEdu csomag beállítási lehetőségei}
{
a
}

\MEDsubsection[Beamer]{Options for Beamer-based formats}{Beamer}{Beamer alapú formátum beállítások}

\MEDframe{Screen width}
{
Multedu allows to utilize two popular screen width.
The default is the spreading format with aspect ratio 16:9.
To set ratio 4:3, use
\noindent\vskip-\parskip\lstinline|\{\\def\\DisableWideScreen\{YES\}\}|\par
}
{Képernyő szélesség}
{
A Multedu lehetővé teszi kétféle elterjedt formátum használatát.
Egyre gyakoribb 16:9  arányú képformátum így az az alap beállítás.
A 4:3 arányú képformátumot a 
\noindent\vskip-\parskip\lstinline|\{\\def\\DisableWideScreen\{YES\}\}|\par
definiálásával lehet beállítani.
}

\MEDframe{Table of contents}
{
Sometimes (mainly in the case of short presentations) the table of contents is not necessary at all.
It can be disabled through defining 
\noindent\vskip-\parskip\lstinline|\{\\def\\DisableTOC\{YES\}\}|\par
It might also happen, that chapter-level TOC is still needed, but the section level not.
This can be reached through defining
\noindent\vskip-\parskip\lstinline|\{\\def\\DisableSectionTOC\{YES\}\}|\par
}
{Tartalomjegyzék}
{
Néha (főként rövid bemutatók esetén) egyáltalán nincs szükség tartalomjegyzékre.
Ezt a 
\noindent\vskip-\parskip\lstinline|\{\\def\\DisableTOC\{YES\}\}|\par
definiálásával lehet elérni. Az is előfordul, hogy a fejezet-szintű
tartalomjegyzék még szükséges, de a szakasz szintű már nem. Ezt a
\noindent\vskip-\parskip\lstinline|\{\\def\\DisableSectionTOC\{YES\}\}|\par
definiálásával lehet elérni.
}

\MEDsection[Tips]{Tips for using package MultEdu}{Ötletek}{Ötletek a MultEdu csomag használatához}

\MEDframe{Options for using package MultEdu}
{
a
}
{Ötletek a MultEdu csomag használatához}
{
a
}


%\makeatletter
%\DeclareRobustCommand{\stdLaTeX}{L\kern-.36em
%  {%
%    \sbox\z@ T%
%    \vbox to\ht0{\hbox{$\m@th$%
%        \csname S@\f@size\endcsname
%        \fontsize\sf@size\z@
%        \math@fontsfalse\selectfont
%        A}%
%      \vss}%
%  }%
%  \kern-.15em
%  \TeX}
%%
%\DeclareRobustCommand{\faqLaTeX}{L%
%  {%
%    \setbox0\hbox{T}%
%    \setbox\@tempboxa\hbox{$\m@th$%
%      \csname S@\f@size\endcsname
%      \fontsize\sf@size\z@
%      \math@fontsfalse\selectfont
%      A}%
%    \@tempdima\ht0
%    \advance\@tempdima-\ht\@tempboxa
%    \@tempdima\strip@pt\fontdimen1\font\@tempdima
%    \advance\@tempdima-.36em
%    \kern\@tempdima
%    \vbox to\ht0{\box\@tempboxa
%      \vss}%
%  }%
%  \kern-.15em
%  \TeX}
%\makeatother  
%
%\stdLaTeX
%,\faqLaTeX

%\begin{verbatim}
%.
%├── common
%│   ├── defaults
%│   │   └── Copyright.tex
%│   ├── formats
%│   │   ├── beamer_ME.tex
%│   │   ├── beamertry.tex
%│   │   ├── eBook.tex
%│   │   ├── memoir_A4Fancy.tex
%│   │   ├── memoir_A4Simple.tex
%│   │   ├── memoir_eBook.tex
%│   │   ├── memoir_WEB.tex
%│   │   ├── MESetupBeamerFormat.tex
%│   │   ├── MESetupPrintedFormat.tex
%│   │   └── WEB.tex
%│   ├── images
%│   │   ├── CC88x31.png
%│   │   ├── CoverIllustration.jpeg
%│   │   ├── LatexTalk.png
%│   │   ├── LatexText.png
%│   │   ├── under_construction.jpg
%│   │   └── under_construction-mask.jpg
%│   ├── MEColors.tex
%│   ├── MEFigures.tex
%│   ├── MELanguages.tex
%│   ├── MEListings.tex
%│   ├── MEMacros.tex
%│   ├── MEPackages.tex
%│   ├── MESections.tex
%│   ├── METables.tex
%│   └── UseLATEX.cmake
%├── LICENSE
%├── README.md
%└── WorkStuff
%    └── Demo
%        ├── build
%        │   └── build
%        ├── CMakeListsMid.txt
%        ├── CMakeListsNew.txt
%        ├── CMakeLists_ShouldBeGood.txt
%        ├── CMakeLists.txt
%        ├── dat
%        ├── Demo.tex
%        ├── fig
%        │   ├── CoverIllustration.jpeg
%        │   ├── forloops.png
%        │   └── phone_anchestors.jpg
%        ├── lst
%        │   ├── calculatorresult.txt
%        │   ├── expensive_calculator.cpp
%        │   ├── forloops.v
%        │   ├── HelloWorld.cpp
%        │   ├── lower1.c
%        │   └── lower2.c
%        ├── Main.tex
%        └── src
%            ├── Abstract.tex
%            ├── Bibliography.bib
%            ├── Copyright.tex
%            ├── Defines.tex
%            ├── Defines.tex.in
%            ├── Figures.tex
%            ├── General.tex
%            ├── Glossary.tex
%            ├── Heading.tex
%            ├── Listings.tex
%            ├── Options.tex
%            ├── Organizing.tex
%            └── Sectioning.tex
%\end{verbatim}
