%%%
%%% This is the chapter about sectioning documents for the MultEdu system

\MEDchapter[General]{General information}{Általános}{Általános információ}


\MESetListingFormat{TeX}
\lstset{ basicstyle=\ttfamily\color{black}\normalsize}
\MEDsection[Installing]{Installing MultEdu}{Beüzemelés}{A MultEdu beüzemelése}

%\MEDframe{Document units}
%{
%}
%{Document units}
%{
%}

\MEDsection[Structure]{Structure of MultEdu}{Szerkezet}{A MultEdu könyvtár szerkezete}

\MEDframe{Directory structure}
{
The MultEdu system is assumed to be used with the directury structure below.
It comes with two main subdirectories: \lstinline|./common| comprises all files of
the MultEdu system, and  \lstinline|./Workstuff| models the users subdirectory structure.
\par\vskip-\parskip\noindent\lstinline*.*
\par\vskip-\parskip\noindent\lstinline*|-- common*
\par\vskip-\parskip\noindent\lstinline*|-- WorkStuff*


You may add your stuff like
\par\vskip-\parskip\noindent\lstinline*.*
\par\vskip-\parskip\noindent\lstinline*|-- Exams*
\par\vskip-\parskip\noindent\lstinline*|-- Labs*
\par\vskip-\parskip\noindent\lstinline*|-- Lectures*
\par\vskip-\parskip\noindent\lstinline*|-- Papers*
}
{Könyvtár szerkezet}
{
A MultEdu rendszert az alábbi könyvtár szerkezetben célszerű használni.
Két fő könyvtára: a \lstinline|./common|, amely tartalmazza a MultEdu összes fájlját,
 és a \lstinline|./Workstuff|, amely a felhasználói könyvtár szerkezetet modellezi.
\par\vskip-\parskip\noindent\lstinline*.*
\par\vskip-\parskip\noindent\lstinline*|-- common*
\par\vskip-\parskip\noindent\lstinline*|-- WorkStuff*
*

A felhasználó projekt csoportokat például ilyen szerkezetben érdemes hozzáadni:
\par\vskip-\parskip\noindent\lstinline*.*
\par\vskip-\parskip\noindent\lstinline*|-- Exams*
\par\vskip-\parskip\noindent\lstinline*|-- Labs*
\par\vskip-\parskip\noindent\lstinline*|-- Lectures*
\par\vskip-\parskip\noindent\lstinline*|-- Papers*
}

\MEDsubsection[\lstinline*common*]{Subdirectory \lstinline*common*}{\lstinline*common*}{A \lstinline*common* alkönyvtár}

\MEDframe{Directory structure of subdirectory \lstinline*common*}
{
Subdirectory \lstinline|./common| comprises some special subsubdirectories
and general purpose macro files. \ao{MultEdu attempts to be as user-friendly as possible: it uses default settings, files, images, etc., to allow a quick start for a new development.}
\par\vskip-\parskip\noindent\lstinline*.*
\par\vskip-\parskip\noindent\lstinline*|-- common*
\par\vskip-\parskip\noindent\lstinline*|    |-- defaults*
\par\vskip-\parskip\noindent\lstinline*|    |-- formats*
\par\vskip-\parskip\noindent\lstinline*|    |-- images*

Subsubdirectory \lstinline|./common| contains some default text, like copyright.
\ao{In general, if the user does not provide its own elements, MultEdu uses the defaults instead
(provided that using and presenting it is not disabled, see later.)}

Subsubdirectory \lstinline|./formats| contains the possible format specification macros,
here you can add your owns.

Subsubdirectory \lstinline|./images| contains some images, partly the ones which are used 
as defaults.
}
{A \lstinline*common* alkönyvtár szerkezete}
{
A \lstinline|./common| különleges célú al-alkönyvtárakat, valamint általános célú makró fájlokat tartalmaz.  \ao{A Multedu megpróbál a lehető legbarátságosabb lenni:
alapértelmezett beállításokat, fájlokat, képeket, stb használ, hogy gyorsan el 
lehessen kezdeni egy új fejlesztést.}
\par\vskip-\parskip\noindent\lstinline*.*
\par\vskip-\parskip\noindent\lstinline*|-- common*
\par\vskip-\parskip\noindent\lstinline*|    |-- defaults*
\par\vskip-\parskip\noindent\lstinline*|    |-- formats*
\par\vskip-\parskip\noindent\lstinline*|    |-- images*

A \lstinline|./common| al-alkönyvtár olyan alapértelmezett szöveget tárol, mint 
a szerzői jogok.
\ao{Alapértelmezetten, ha a felhasználó nem adja meg saját dokumentum elemeit,
a MultEdu automatikusan az alpértelmezetteket használja helyettük
(feltéve, hogy azok használata nincs megtiltva, lásd később).
}

A \lstinline|./formats| al-alkönyvtár tartalmazza a formátumokat meghatározó makrókat;
itt adhatja hozzá a felhasználó esetleges saját makróit.

Az \lstinline|./images|  al-alkönyvtár képeket tartalmaz, amelyek egy része
alapértelmezett képként használatos.
}

\MEDsubsection[\lstinline*Workstuff*]{Subdirectory \lstinline*Workstuff*}{\lstinline*Workstuff*}{A \lstinline*Workstuff* alkönyvtár}
\MEDframe{Files in subdirectory \lstinline|Workstuff|}
{
Subdirectory \lstinline|./Workstuff| contains the files of the present demo,
and serves as an example of using the system. It contains a sample project \lstinline|./Workstuff/Demo|, which has three main files.
\par\vskip-\parskip\noindent\lstinline*|-- WorkStuff*
\par\vskip-\parskip\noindent\lstinline*|    |-- Demo*
\par\vskip-\parskip\noindent\lstinline*|        |-- CMakeLists.txt*
\par\vskip-\parskip\noindent\lstinline*|        |-- Demo.tex*
\par\vskip-\parskip\noindent\lstinline*|        |-- Main.tex*

The real main source file is \lstinline|Main.tex|, and \lstinline|Demo.tex| is a lightweight envelope to it. It is also possible to use CMake package UseLATEX for compiling
your text to different formats and languages in batch mode.
}
{A \lstinline|Workstuff| alkönyvtár fájljai}
{
A \lstinline|./Workstuff|  al-alkönyvtár tartalmazza  (a példa programként is szolgáló)
felhasználói leírás fájljait. Egy olyan \lstinline|./Workstuff/Demo| projektet tartalmaz, amelyik (a saját főkönyvtárában)
három fájlból áll. 
\par\vskip-\parskip\noindent\lstinline*|-- WorkStuff*
\par\vskip-\parskip\noindent\lstinline*|    |-- Demo*
\par\vskip-\parskip\noindent\lstinline*|    .    |-- CMakeLists.txt*
\par\vskip-\parskip\noindent\lstinline*|    .   |-- Demo.tex*
\par\vskip-\parskip\noindent\lstinline*|    .    |-- Main.tex*

A valódi főprogram  \lstinline|Main.tex|, és ehhez készült egy \lstinline|Demo.tex| nagyon egyszerű boríték. A CMake rendszeren keresztül a UseLATEX csomag is használható arra,
hogy egy szerkesztés után, a kötegelt feldolgozási módot használva, egyetlen lépésben elő lehessen állítani a forrásnyelvi fájlból a különböző nyelvű és formátumú dokumentumokat; erre való a  \lstinline|CMakeLists.txt| fájl.
}

\MEDframe{Subsubdirectories in \lstinline|./Workstuff|}
{
\ao{Subdirectory \lstinline|./Workstuff| has some subsubdirectories, for different goals.}
\par\vskip-\parskip\noindent\lstinline*|-- WorkStuff*
\par\vskip-\parskip\noindent\lstinline*|    |-- Demo*
\par\vskip-\parskip\noindent\lstinline*|    .    |-- build*
\par\vskip-\parskip\noindent\lstinline*|    .    .    .   |-- build*
\par\vskip-\parskip\noindent\lstinline*|    .    |-- dat*
\par\vskip-\parskip\noindent\lstinline*|    .    |-- fig*
\par\vskip-\parskip\noindent\lstinline*|    .    |-- lst*
\par\vskip-\parskip\noindent\lstinline*|    .    |-- src*

The file  \lstinline|Main.tex| inputs files in the sub-subdirectories.

Subsubdirectories \par\vskip-\parskip\noindent\lstinline*|-- build* and \par\vskip-\parskip\noindent\lstinline*|   .  .  |-- build*
\par\vskip-\parskip\noindent are only needed
if using CMake\ao{; they contain temporary files created during processing.
The system also makes its own copy of the subdirectory \lstinline|common| in 
subdirectory \lstinline|Demo|. Those files can be deleted any time: when compiling,
CMake will regenerate them}.

Subsubdirectory 
\par\vskip-\parskip\noindent\lstinline*|   .   |-- src*  is the place for the user's source files, 
\par\vskip-\parskip\noindent\lstinline*|   .   |-- fig* for the images. 
\par\vskip-\parskip\noindent\lstinline*|   .   |-- lst* for the program source files,
\par\vskip-\parskip\noindent\lstinline*|   .   |-- dat* for the other data \ao{(like tabulated data for pgfplot or TikZ figures source code)}. 
 
\ao{You may use (and handle!) further subsubdirectories}
}
{A \lstinline|./Workstuff| al-alkönyvtárai}
{
\ao{A \lstinline|./Workstuff| al-alkönyvtárai különböző célokat szolgálnak.}
\par\vskip-\parskip\noindent\lstinline*|-- WorkStuff*
\par\vskip-\parskip\noindent\lstinline*|    |-- Demo*
\par\vskip-\parskip\noindent\lstinline*|    .    |-- build*
\par\vskip-\parskip\noindent\lstinline*|    .    .    .   |-- build*
\par\vskip-\parskip\noindent\lstinline*|    .    |-- dat*
\par\vskip-\parskip\noindent\lstinline*|    .    |-- fig*
\par\vskip-\parskip\noindent\lstinline*|    .    |-- lst*
\par\vskip-\parskip\noindent\lstinline*|    .    |-- src*

A fő  \lstinline|Main.tex| menet közben magába olvassa az alkönyvtárakban levő egyéb fájlokat.
A \par\vskip-\parskip\noindent\lstinline*|-- build* és \par\vskip-\parskip\noindent\lstinline*|   .  .  |-- build*
\par\vskip-\parskip\noindent alkönyvtárak csak akkor kellenek
ha a CMake rendszert használjuk\ao{; ezek a feldolgozás során szükséges 
átmeneti fájlokat tartalmazzák. A rendszer készít a \lstinline|Demo| alkönyvtárba 
egy saját másolatot a  \lstinline|common| alkönyvtárról.
Ezek a fájlok bármikor törölhetők: amikor fordít, a CMaker újra generálja azokat. 
}.

 
\par\vskip-\parskip\noindent\lstinline*|   .   |-- src*  tartalmazza a felhasználó forráskód fájljait, 
\par\vskip-\parskip\noindent\lstinline*|   .   |-- fig* a képeit, 
\par\vskip-\parskip\noindent\lstinline*|   .   |-- lst* a programlisták forrás kódját,
\par\vskip-\parskip\noindent\lstinline*|   .   |-- dat* a többi adatot \ao{(például
táblázatok, adatok a pgfplot vagy kód a TikZ ábrák számára)}. 
 
\ao{További alkönyvtárak is készíthetők, de azokat a felhasználónak kell kezelni.}
}

%\makeatletter
%\DeclareRobustCommand{\stdLaTeX}{L\kern-.36em
%  {%
%    \sbox\z@ T%
%    \vbox to\ht0{\hbox{$\m@th$%
%        \csname S@\f@size\endcsname
%        \fontsize\sf@size\z@
%        \math@fontsfalse\selectfont
%        A}%
%      \vss}%
%  }%
%  \kern-.15em
%  \TeX}
%%
%\DeclareRobustCommand{\faqLaTeX}{L%
%  {%
%    \setbox0\hbox{T}%
%    \setbox\@tempboxa\hbox{$\m@th$%
%      \csname S@\f@size\endcsname
%      \fontsize\sf@size\z@
%      \math@fontsfalse\selectfont
%      A}%
%    \@tempdima\ht0
%    \advance\@tempdima-\ht\@tempboxa
%    \@tempdima\strip@pt\fontdimen1\font\@tempdima
%    \advance\@tempdima-.36em
%    \kern\@tempdima
%    \vbox to\ht0{\box\@tempboxa
%      \vss}%
%  }%
%  \kern-.15em
%  \TeX}
%\makeatother  
%
%\stdLaTeX
%,\faqLaTeX

%\begin{verbatim}
%.
%├── common
%│   ├── defaults
%│   │   └── Copyright.tex
%│   ├── formats
%│   │   ├── beamer_ME.tex
%│   │   ├── beamertry.tex
%│   │   ├── eBook.tex
%│   │   ├── memoir_A4Fancy.tex
%│   │   ├── memoir_A4Simple.tex
%│   │   ├── memoir_eBook.tex
%│   │   ├── memoir_WEB.tex
%│   │   ├── MESetupBeamerFormat.tex
%│   │   ├── MESetupPrintedFormat.tex
%│   │   └── WEB.tex
%│   ├── images
%│   │   ├── CC88x31.png
%│   │   ├── CoverIllustration.jpeg
%│   │   ├── LatexTalk.png
%│   │   ├── LatexText.png
%│   │   ├── under_construction.jpg
%│   │   └── under_construction-mask.jpg
%│   ├── MEColors.tex
%│   ├── MEFigures.tex
%│   ├── MELanguages.tex
%│   ├── MEListings.tex
%│   ├── MEMacros.tex
%│   ├── MEPackages.tex
%│   ├── MESections.tex
%│   ├── METables.tex
%│   └── UseLATEX.cmake
%├── LICENSE
%├── README.md
%└── WorkStuff
%    └── Demo
%        ├── build
%        │   └── build
%        ├── CMakeListsMid.txt
%        ├── CMakeListsNew.txt
%        ├── CMakeLists_ShouldBeGood.txt
%        ├── CMakeLists.txt
%        ├── dat
%        ├── Demo.tex
%        ├── fig
%        │   ├── CoverIllustration.jpeg
%        │   ├── forloops.png
%        │   └── phone_anchestors.jpg
%        ├── lst
%        │   ├── calculatorresult.txt
%        │   ├── expensive_calculator.cpp
%        │   ├── forloops.v
%        │   ├── HelloWorld.cpp
%        │   ├── lower1.c
%        │   └── lower2.c
%        ├── Main.tex
%        └── src
%            ├── Abstract.tex
%            ├── Bibliography.bib
%            ├── Copyright.tex
%            ├── Defines.tex
%            ├── Defines.tex.in
%            ├── Figures.tex
%            ├── General.tex
%            ├── Glossary.tex
%            ├── Heading.tex
%            ├── Listings.tex
%            ├── Options.tex
%            ├── Organizing.tex
%            └── Sectioning.tex
%\end{verbatim}
