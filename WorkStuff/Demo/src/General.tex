%%%
%%% This is the chapter about sectioning documents for the MultEdu system

\MEDchapter[General]{General information}{Általános}{Általános információ}


\MESetListingFormat{TeX}
\lstset{ basicstyle=\ttfamily\color{black}\normalsize}


\MEDsection[Introduction]{Introduction}{Bevezetés}{Bevezetés}

\MEDframe{Introduction}
{
\ao{For teaching my courses I needed to develop course material, in different forms of appearance;
the present package is a by-product of this activity.
Good course materials develop quickly, especially, if the field of science in question itself is renewed daily.
In informatics, the technology, the statistics, the products, the tools, etc. change year by year,
and this alone is a good reason to renew the teaching material for a new semester.}

The todays education needs the course material in various forms: in the lecture room for the projected 
picture well organized text with many pictures are needed, which also serve as a good guide for the lecturer, too.
To prepare for the exams, the explanation provided by the lecturer when projecting the slides is also needed.
\ao{That means, the course material should be available in printable and browsable form, as well as for mobile devices.
A frequent case is that the same material shall be provided in foreign language(s), for foreign students.
In many cases one can rely to good textbooks, but for the more specialized courses this course material
serves as a basic tool for preparing to the exams.}

\ao{The present macro package is developed for my own purposes, which I attempted to develop in such a way,
that when developing course material, one should not deal with the technology of displaying.
In this way also others can use the package, when they follow the rules. The package is quite good
in some fields, at some point I needed to make a bargain between the different needs, not perfect
in some points, and of course in much more aspects I did not have the time to develop features.}

The present document is a demo and test at the same time. It attempts to describe the many features,
and also tests if the features really work. Because of the many features, and their interference,
this job needs a lot of work and time, so the documentation does not always match the actual features,
especially in this initial phase.
}
{Bevezetés}
{
\ao{Kurzusaim tartásához saját tananyagot fejlesztettem, különböző megjelenési formákban; a jelen csomag ennek mellékterméke.
A jó kurzusok tananyaga gyorsan fejlődik, különösen akkor, ha maga a tudományág is naponta megújul.
Az informatikában évről évre változik a technológia, a statisztikák, a termékek, a segédeszközök, stb.;
és már csak emiatt is minden tanévre frissíteni kell a tananyagot.}

Manapság a tananyagot a hallgatóság változatos formákban igényli: előadáson nagy méretű, jól áttekinthető,
kivetíthető anyagot kell használni, amely képekkel gazdagon illusztrált és az előadó számára is jó sorvezetőként szolgál.
A vizsgára készüléshez pedig arra a magyarázatra is szükség van, amit az előadó a kivetített anyaghoz élő szóban hozzáfűz.
\ao{Azaz, olyan magyarázó szöveggel ellátott anyagra is szükség van, amelyet kinyomtatva, asztali gép vagy mobil eszköz
képernyőjén lehet elolvasni. Esetleg ugyanazt a változatot idegen nyelven is közzétenni, külföldi hallgatók számára.
Bár sokszor lehet elérhető könyvekre és megvásárolható jegyzetekre hagyatkozni, a kicsit is speciálisabb anyagok
esetén ez a segédlet lesz a felkészüléshez szükséges tananyag.}

\ao{A jelen makró csomag olyan, amilyet saját kurzusaim készítéséhez fejlesztettem, és igyekeztem olyanná tenni, hogy
tananyag fejlesztés közben már ne kelljen a megjelenítés technikájával foglalkozni, és ilyen módon
mások is tudják hasonló célra felhasználni, ha követik a fejlesztés logikáját.  A csomag bizonyos vonatkozásokban 
egészen jó, néhol tudatosan kompromisszumot kellett kötni a sokféle igény között, néhol még nem tökéletes, 
és sok vonatkozásban még nem jutott időm további tulajdonságok fejlesztésére.}

A jelen dokumentum egyúttal bemutató és tulajdonság tesztelő is. A dokumentum megkísérli bemutatni, mit hogyan 
kell és lehet használni, egyúttal azt is megvizsgálva, hogy tényleg működik-e az elvárt módon. A sokféle tulajdonság
és különösen azok kölcsönhatása miatt sok munkát és időt igényel a fejlesztés, ezért a tényleges tulajdonságok
nem mindig egyeznek meg a dokumentációval, különösen a kezdeti fázisban. 


}

\MEDframe{Introduction}
{
The macro package can be used at (at least) three different levels.
Even the lowest level assumes some familiarity  with \LaTeX.
At the very basic level, you might just take the package,
replace and modify files in the distribution.
At the advanced level (this assumes reading the User's manual \smiley)
the user learns the facilities provided in the package,
and prepares his/her courses actively using those facilities.
Power users might add their own macros (preferably uploaded to the distribution), i.e. take part in the development.
}
{Bevezetés}
{
A makró csomag (legalább) három különböző felhasználói szinten 
alkalmazható. Már a legalacsonyabb szinten is szükségesek a \LaTeX-re
vonatkozó elemi ismeretek.
Az alap szinten a felhasználó egyszerűen csak helyettesíti és módosítja a rendszert bemutató dokumentumokat.
Haladó szinten (ehhez már el kell olvasni a felhasználói leírást is \smiley) megtanulja a csomagban található makrók által biztosított lehetőségeket, és azokat aktívan használva fejleszti dokumentumait.
Tapasztalt felhasználóként saját makrókat is készíthet (jó, ha azokat
a letölthető anyaghoz hozzáadja), azaz aktívan részt vesz a fejlesztésben.
}

\MEDsection[Installing]{Installing and utilizing \lstinline|MultEdu|}{Beüzemelés}{A \lstinline|MultEdu| beüzemelése és használata}

\MEDframe{Installing \lstinline|MultEdu|}
{
\gls{MultEdu}, as any package based on \LaTeX, assumes that the user has
experiences with using \LaTeX. I.e. some \LaTeX{} distribution must already be installed on the system of the user. If you want to use the
batch processing facility, the CMake system must also be installed.

For the simplicity of utilization and starting up, 
the best way is to create a main directory for your family of projects and a subdirectory for your first project, 
as described below. The quickest way is to copy \lstinline|./Workstuff| (after deleting and renaming some files)
and to prepare your own "Hello World" program.
Making minor changes to that source you may experience
some features of the package. Then, it is worth at least to skim the user's manual, to see what features you need. After that, you may start your own development.
At the beginning text only, later you can learn the advanced
possibilities.
\ao{Do not forget: LaTeX is hard, it needs accurate coding,
and so is \gls{MultEdu}, too.
Frequent saving and using versioning can help a lot.}
}
{A \lstinline|MultEdu| beüzemelése}
{
A \gls{MultEdu} (mint minden \LaTeX{} alapú rendszer) feltételezi, hogy 
a felhasználó már rendelkezik tapasztalatokkal a \LaTeX{}  használatában.
Azaz, a felhasználó rendszerén már működnie kell valamilyen \LaTeX{}
rendszernek.

Az egyszerű használat és a gyors elindulás érdekében célszerű a 
lentebb megadott módon saját projekt csoportjainak egy főkönyvtárat és
azon belül az egyes projekteknek alkönyvtárakat létrehozni.
A leggyorsabb magát a \lstinline|./Workstuff| könyvtárat (a megfelelő átnevezésekkel és törlésekkel) lemásolni, és csekély 
módosításokkal elkészíteni saját 'Helló Világ' programját.
Ezután érdemes legalább átlapozni a
felhasználói kézikönyvet, ami után már elkezdheti saját fejlesztését.
Eleinte csak szöveget, aztán sorjában megtanulni a használni kívánt tulajdonságok programozását.
\ao{Ne feledje: a LaTeX nehéz nyelv, pontos kódolást igényel,
és ezért ilyen a \gls{MultEdu} is. A gyakori mentések és a verziókövető rendszerek használata nagy segítséget jelentenek.}
}

\MEDsection[Structure]{Structure of \lstinline|MultEdu|}{Szerkezet}{A \lstinline|MultEdu| könyvtár szerkezete}

\MEDframe{Directory structure}
{
The \gls{MultEdu} system is assumed to be used with the directory structure below.
It comes with two main subdirectories: \lstinline|./common| comprises all files of
the \gls{MultEdu} system, and  \lstinline|./Workstuff| models the users subdirectory structure.
\par\vskip-\parskip\noindent\lstinline*.*
\par\vskip-\parskip\noindent\lstinline*|-- common*
\par\vskip-\parskip\noindent\lstinline*|-- WorkStuff*

You may add your project groups stuff like
\par\vskip-\parskip\noindent\lstinline*.*
\par\vskip-\parskip\noindent\lstinline*|-- Exams*
\par\vskip-\parskip\noindent\lstinline*|-- Labs*
\par\vskip-\parskip\noindent\lstinline*|-- Lectures*
\par\vskip-\parskip\noindent\lstinline*|-- Papers*

which directories have a subdirectory structure similar to that of \lstinline*|-- WorkStuff*
}
{Könyvtár szerkezet}
{
A \gls{MultEdu} rendszert az alábbi könyvtár szerkezetben célszerű használni.
Két fő könyvtára: a \lstinline|./common|, amely tartalmazza a \gls{MultEdu} összes fájlját,
 és a \lstinline|./Workstuff|, amely a felhasználói könyvtár szerkezetet modellezi.
\par\vskip-\parskip\noindent\lstinline*.*
\par\vskip-\parskip\noindent\lstinline*|-- common*
\par\vskip-\parskip\noindent\lstinline*|-- WorkStuff*

A felhasználói projekt csoportokat ilyen szerkezetben érdemes hozzáadni:
\par\vskip-\parskip\noindent\lstinline*.*
\par\vskip-\parskip\noindent\lstinline*|-- Exams*
\par\vskip-\parskip\noindent\lstinline*|-- Labs*
\par\vskip-\parskip\noindent\lstinline*|-- Lectures*
\par\vskip-\parskip\noindent\lstinline*|-- Papers*

amely könyvtáraknak a \lstinline*|-- WorkStuff* könyvtárhoz
hasonló belső alkönyvtárai vannak
}

\MEDsubsection[\lstinline*common*]{Subdirectory \lstinline*common*}{\lstinline*common*}{A \lstinline*common* alkönyvtár}

\MEDframe{Directory structure of subdirectory \lstinline*common*}
{
Subdirectory \lstinline|./common| comprises some special subsubdirectories
and general purpose macro files. \ao{\gls{MultEdu} attempts to be as user-friendly as possible: it uses default settings, files, images, etc., to allow a quick start for a new development.}
\par\vskip-\parskip\noindent\lstinline*.*
\par\vskip-\parskip\noindent\lstinline*|-- common*
\par\vskip-\parskip\noindent\lstinline*|    |-- defaults*
\par\vskip-\parskip\noindent\lstinline*|    |-- formats*
\par\vskip-\parskip\noindent\lstinline*|    |-- images*

Subsubdirectory \lstinline|./defaults| contains some default text, like copyright.
\ao{In general, if the user does not provide its own elements, \gls{MultEdu} uses the defaults instead
(provided that using and presenting it is not disabled, see later.)}

Subsubdirectory \lstinline|./formats| contains the possible format specification macros,
here you can add your own format macros.

Subsubdirectory \lstinline|./images| contains some images, partly the ones which are used 
as defaults.
}
{A \lstinline*common* alkönyvtár szerkezete}
{
A \lstinline|./common| különleges célú al-alkönyvtárakat, valamint általános célú makró fájlokat tartalmaz.  \ao{A \gls{MultEdu} megpróbál a lehető legbarátságosabb lenni:
alapértelmezett beállításokat, fájlokat, képeket, stb használ, hogy gyorsan el 
lehessen kezdeni egy új fejlesztést.}
\par\vskip-\parskip\noindent\lstinline*.*
\par\vskip-\parskip\noindent\lstinline*|-- common*
\par\vskip-\parskip\noindent\lstinline*|    |-- defaults*
\par\vskip-\parskip\noindent\lstinline*|    |-- formats*
\par\vskip-\parskip\noindent\lstinline*|    |-- images*

A \lstinline|./defaults| al-alkönyvtár olyan alapértelmezett szöveget tárol, mint 
a szerzői jogok.
\ao{Alapértelmezetten, ha a felhasználó nem adja meg saját dokumentum elemeit,
a \gls{MultEdu} automatikusan az alapértelmezetteket használja helyettük
(feltéve, hogy azok használata nincs megtiltva, lásd később).
}

A \lstinline|./formats| al-alkönyvtár tartalmazza a formátumokat meghatározó makrókat\ao{;
itt adhatja hozzá a felhasználó esetleges saját formátum leíró makróit.}

Az \lstinline|./images|  al-alkönyvtár képeket tartalmaz\ao{, amelyek egy része
alapértelmezett képként használatos.}
}

\MEDsubsection[\lstinline*Workstuff*]{Subdirectory \lstinline*Workstuff*}{\lstinline*Workstuff*}{A \lstinline*Workstuff* alkönyvtár}
\label{sec:subdirectories}
\MEDframe{Files in subdirectory \lstinline|Workstuff|}
{
Subdirectory \lstinline|./Workstuff| contains the files of the present demo,
and serves as an example of using the system (a kind of User's Guide). It contains a sample project \lstinline|./Workstuff/Demo|, which has three main files.
\par\vskip-\parskip\noindent\lstinline*|-- WorkStuff*
\par\vskip-\parskip\noindent\lstinline*|    |-- Demo*
\par\vskip-\parskip\noindent\lstinline*|    .    |-- CMakeLists.txt*
\par\vskip-\parskip\noindent\lstinline*|    .    |-- Demo.tex*
\par\vskip-\parskip\noindent\lstinline*|    .    |-- Main.tex*

The real main source file is \lstinline|Main.tex|, and \lstinline|Demo.tex| is a lightweight envelope to it. 
(if you want to use UseLATEX, you need to use the file with name \lstinline|Main.tex|, the envelop must be concerted with the CMakeLists.txt file)
}
{A \lstinline|Workstuff| alkönyvtár fájljai}
{
A \lstinline|./Workstuff|  al-alkönyvtár tartalmazza  (a példa programként is szolgáló)
felhasználói leírás fájljait. Egy olyan \lstinline|./Workstuff/Demo| projektet tartalmaz, amelyik (a saját főkönyvtárában)
három fájlból áll. 
\par\vskip-\parskip\noindent\lstinline*|-- WorkStuff*
\par\vskip-\parskip\noindent\lstinline*|    |-- Demo*
\par\vskip-\parskip\noindent\lstinline*|    .    |-- CMakeLists.txt*
\par\vskip-\parskip\noindent\lstinline*|    .    |-- Demo.tex*
\par\vskip-\parskip\noindent\lstinline*|    .    |-- Main.tex*

A valódi főprogram  \lstinline|Main.tex|, és ehhez készült egy \lstinline|Demo.tex| nagyon egyszerű boríték. 
Ha használja a UseLATEX csomagot, a  \lstinline|Main.tex| file
használata (ezzel a névvel) kötelező, a boríték fájl nevét pedig a CMakeLists.txt fájllal egyeztetni kell.}

\MEDframe{Subsubdirectories in \lstinline|./Workstuff|}
{
\ao{Subdirectory \lstinline|./Workstuff| has some subsubdirectories, for different goals.}
\par\vskip-\parskip\noindent\lstinline*|-- WorkStuff*
\par\vskip-\parskip\noindent\lstinline*|    |-- Demo*
\par\vskip-\parskip\noindent\lstinline*|    .    |-- build*
\par\vskip-\parskip\noindent\lstinline*|    .    .    .   |-- build*
\par\vskip-\parskip\noindent\lstinline*|    .    |-- dat*
\par\vskip-\parskip\noindent\lstinline*|    .    |-- fig*
\par\vskip-\parskip\noindent\lstinline*|    .    |-- lst*
\par\vskip-\parskip\noindent\lstinline*|    .    |-- src*

The file  \lstinline|Main.tex| inputs files in the sub-subdirectories.


Subsubdirectory 
\par\vskip-\parskip\noindent\lstinline*|   .   |-- src*  is the place for the user's source files, 
\par\vskip-\parskip\noindent\lstinline*|   .   |-- fig* for the images. 
\par\vskip-\parskip\noindent\lstinline*|   .   |-- lst* for the program source files,
\par\vskip-\parskip\noindent\lstinline*|   .   |-- dat* for the other data \ao{(like tabulated data for pgfplot or TikZ figures source code)}. 
 
\ao{You may use (and handle!, especially in CMakeLists.txt) further subsubdirectories.}
}
{A \lstinline|./Workstuff| al-alkönyvtárai}
{
\ao{A \lstinline|./Workstuff| al-alkönyvtárai különböző célokat szolgálnak.}
Célszerű a felhasználói projekt könyvtárakat is hasonlóan berendezni.
\par\vskip-\parskip\noindent\lstinline*|-- WorkStuff*
\par\vskip-\parskip\noindent\lstinline*|    |-- Demo*
\par\vskip-\parskip\noindent\lstinline*|    .    |-- build*
\par\vskip-\parskip\noindent\lstinline*|    .    .    .   |-- build*
\par\vskip-\parskip\noindent\lstinline*|    .    |-- dat*
\par\vskip-\parskip\noindent\lstinline*|    .    |-- fig*
\par\vskip-\parskip\noindent\lstinline*|    .    |-- lst*
\par\vskip-\parskip\noindent\lstinline*|    .    |-- src*

A fő  \lstinline|Main.tex| menet közben magába olvassa az alkönyvtárakban levő egyéb fájlokat.
 
\par\vskip-\parskip\noindent\lstinline*|   .   |-- src*  tartalmazza a felhasználó forráskód fájljait, 
\par\vskip-\parskip\noindent\lstinline*|   .   |-- fig* a képeit, 
\par\vskip-\parskip\noindent\lstinline*|   .   |-- lst* a programlisták forrás kódját,
\par\vskip-\parskip\noindent\lstinline*|   .   |-- dat* a többi adatot \ao{(például
táblázatok, adatok a pgfplot vagy kód a TikZ ábrák számára)}. 
 
\ao{További alkönyvtárak is készíthetők, de azokat a felhasználónak kell kezelni, és módosítania kell a CMakeLists.txt fájlt is.}
}

\MEDframe{Subsubdirectories and files for CMake}
{
It is also possible to use CMake package UseLATEX for compiling
your text to different formats and languages in batch mode;
producing the documents in different languages and formats in one single step.
File \lstinline|CMakeLists.txt| serves for that goal.

Subsubdirectories \par\vskip-\parskip\noindent\lstinline*|-- build* and \par\vskip-\parskip\noindent\lstinline*|   .  .  |-- build*
\par\vskip-\parskip\noindent are only needed
if using CMake\ao{; they contain temporary files created during processing.
The system also makes its own copy of the subdirectory \lstinline|common| in 
your project directory (corresponding to subdirectory \lstinline|Demo|). Those files can be deleted any time: when compiling,
CMake will regenerate them}.
}
{A CMake al-alkönyvtárai és fájljai}
{
A CMake rendszeren keresztül a UseLATEX csomag is használható arra,
hogy egy szerkesztés után, a kötegelt feldolgozási módot használva, egyetlen lépésben elő lehessen állítani a forrásnyelvi fájlból a különböző nyelvű és formátumú dokumentumokat; erre való a  \lstinline|CMakeLists.txt| fájl.

A \par\vskip-\parskip\noindent\lstinline*|-- build* és \par\vskip-\parskip\noindent\lstinline*|   .  .  |-- build*
\par\vskip-\parskip\noindent alkönyvtárak csak akkor kellenek
ha a CMake rendszert használjuk\ao{; ezek a feldolgozás során szükséges 
átmeneti fájlokat tartalmazzák. A rendszer készít a projekt könyvtárába (ami a \lstinline|Demo| alkönyvtárnak felel meg)
egy saját másolatot a  \lstinline|common| alkönyvtárról.
Ezek a fájlok bármikor törölhetők: amikor fordít, a CMake újra generálja azokat}.
}

\MEDsubsection[Generated]{Generated files}{Generált fájlok}{Generált fájlok}

\MEDframe{Generated files}
{
During compilation, \LaTeX{} generates a number of different working files.
These will unfortunately pollute the project base directory. As shown
in section~\ref{sec:subdirectories}, only 3 files are needed for the operation,
the rest can be deleted any time.

Compilation in batch mode also prepares some \lstinline|.tex| files,
which can be removed also any time, or even can be compiled manually. 
Do not forget to edit file \lstinline|src/Defines.tex| before compiling,
if you use them for that goal.
}
{Generált fájlok}
{
A fordítás során a \LaTeX{} számos munka fájlt állít elő.
Ezek sajnos a projekt gyökér könyvtárába kerülnek. Amint \aref{sec:subdirectories} szakaszban látható, a működéshez csak 3 fájl szükséges,
a többi bármikor törölhető.

A kötegelt feldolgozás is készít a projekt gyökér könyvtárába \lstinline|.tex| 
forrás fájlokat. Ezek is bármikor törölhetők, de akár 'kézi' fordítással kimenő fájlt is készíthetünk belőlük. Ez utóbbi esetben érdemes előtte az
 \lstinline|src/Defines.tex| fájlt átszerkeszteni.
}
%\MEDsection[Tips]{Tips for using package MultEdu}{Ötletek}{Ötletek a MultEdu csomag használatához}
%
%\MEDframe{Options for using package MultEdu}
%{
%a
%}
%{Ötletek a MultEdu csomag használatához}
%{
%a
%}

\MEDsection[Distribution]{The \lstinline|MultEdu| distribution kit}{A csomag}{A \lstinline|MultEdu| csomagról}

\MEDframe{The \lstinline|MultEdu| distribution kit}
{The \lstinline|MultEdu| package come with full source (and full faith).
The author is rather power user than \LaTeX{} expert. Many of the macros
are adapted from ideas and solution on the Internet. The source contains
references to the original publisher, but the users' guide does not waste space 
for acknowledgements. However, the author acknowledges the contribution
of all respective authors both for the code and the support on different user communities.

The package contains also some \lstinline|.pdf| files in different output
formats and languages. The file name do not contain the version number
(their title page does). The purpose of those files (in addition to
serving as users' guide) to allow the potential users to decide at a glance,
whether they like the provided features.
}
{A \lstinline|MultEdu| csomagról}
{A \lstinline|MultEdu| csomag teljes forráskódot tartalmaz (szépítgetés nélkül).
A szerző nem \LaTeX{} szakértő, csak régi felhasználó.
A makrók nagy része adaptált az Interneten megtalálható eredeti forrásokból.
A forráskód tartalmazza a hivatkozást az eredeti kódra, a felhasználói kézikönyv
nem veszteget  helyet köszönetnyilvánításra. A szerző azonban köszönetét fejezi 
ki az eredeti szerzőknek, mint az eredeti kódért, mind a különböző
felhasználói közösségekben nyújtott támogatásárt.

A csomag tartalmaz pár \lstinline|.pdf| fájlt, különböző formátumban és nyelven.
A fájl nevében nem szerepel a verzió szám (a címlapon igen).
Eme fájlok célja (amellett, hogy felhasználói kézikönyvként is szolgálnak),
hogy a leendő felhasználók gyorsan fel tudják mérni, ilyen tulajdonságokkal
rendelkező dokumentáló rendszert akarnak-e.
}

\MEDframe{The \lstinline|MultEdu| distribution kit}
{
The package \gls{MultEdu} is provided 'as is'. It is developed continuously
and in a non-uniform way. I myself can develop course materials with it.
Both macros and documention keep developing, but it requires (lot of) time.
Reports on faults in operation or errors is documentation is evaluated
as help in the development, even I might consider feature requests.
}
{A \lstinline|MultEdu| csomagról}
{
A \gls{MultEdu} makró csomagot úgy tettem közzé, ahogy van ('as is').
Folyamatosan és egyenetlenül fejlesztem, én magam már jól tudok vele
tananyagot fejleszteni. A makrókat és a dokumentációt is fejlesztem,
de az (sok) időt igényel.
Működési és dokumentációs hibák leírását,
még esetleges tulajdonságok fejlesztésének kérését is örömmel fogadom.
}