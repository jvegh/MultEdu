%%%
%%% This is the chapter about inserting figures with the MultEdu system

\MEDchapter[Supplements]{Utilizing supplements}{Kiegészítések}{Kiegészítések használata}


\MESetListingFormat[basicstyle=\ttfamily\color{black}\normalsize]{TeX}

\MEDsection[Glossaries]{Acronyms and glossary}{Rövidítések}{Rövidítések és szómagyarázat használata}

\MEDframe{What are acronyms and glossaries good for}
{
Especially in the case of technical courses, frequently occur abbreviations, mosaic words, unique interpretations of a term, etc. MultEdu can help you
with using the \lstinline|glossaries| package, to provide your students 
with a hyperlinked facility, to use those terms consequently.

Such elements should be used in the text like \lstinline|\\gls\{ref\}|.
Here \lstinline|ref| is a reference label, and in the text the short 
name of the referenced item appears. In the case of acronyms, the 
expansion also appears at the first occurrence of that acronym.
Some examples are given below; for more explanation see package \lstinline|glossaries|.

}
{Mire jók a rövidítések és a szómagyarázat}
{

Különösen technikai jellegű tárgyak esetén, gyakran szerepelnek 
rövidítések, betűszavak, illetve bizonyos fogalmak egyértelmű meghatározása.
A MultEdu a \lstinline|glossaries| csomagot használva teszi lehetővé,
hogy a dokumentumokban ilyeneket használjon, ráadásul hiper-hivatkozásként.
 
Az ilyen elemeket a szövegben  a \lstinline|\\gls\{ref\}| módon kell elhelyezni.
A szövegben ennek hatására megjelenik az elem rövid neve, és annak első
előfordulásakor annak rövid leírása is. Bővebben lásd a \lstinline|glossaries| 
csomag leírását.
}

\MEDframe{What are acronyms and glossaries good for}
{
Especially in the case of technical courses, frequently occur abbreviations, mosaic words, unique interpretations of a term, etc. MultEdu can help you
with using the \lstinline|glossaries| package, to provide your students 
with a hyperlinked facility, to use those terms consequently.

References to such elements should be used in the text as \lstinline|\\gls\{ref\}|.
Here \lstinline|ref| is a reference label, and in the text the short 
name of the referenced item appears at that place. In the case of acronyms, the 
expansion also appears at the first occurrence of that acronym.
Some examples are given below; for more explanation see package \lstinline|glossaries|.
}
{Mire jók a rövidítések és a szómagyarázat}
{

Különösen technikai jellegű tárgyak esetén, gyakran szerepelnek 
rövidítések, betűszavak, illetve bizonyos fogalmak egyértelmű meghatározása.
A MultEdu a \lstinline|glossaries| csomagot használva teszi lehetővé,
hogy a dokumentumokban ilyeneket használjon, ráadásul hiper-hivatkozásként.
 
Az ilyen elemeket a szövegben  a \lstinline|\\gls\{ref\}| hivatkozásként kell elhelyezni,
és a nyomtatott szövegben ennek hatására azon a helyen az elem rövid neve jelenik meg, és a rövidítések feloldására, a hivatkozás első
előfordulásakor annak rövid leírása is. Bővebben lásd a \lstinline|glossaries| 
csomag leírását.
}

\MEDsubsection[Utilization]{How to use acronyms and glossary}{Használatuk}{Rövidítések és szómagyarázat használata}

\MEDframe{An example}
{

When as a \gls{sampleone} you use the term \gls{computer},
where \gls{CPU} és \gls{DMA} also happens;
in the text \lstinline|When as a \\gls\{sampleone\} you use the term, \\gls\{computer\} where \\gls\{CPU\} and \\gls\{DMA\} also happens| should appear. Multedu then appends chapters \lstinline|Acronyms| and \lstinline|Glossary| to the end of the document, and clicking on those hyperlinked words, you are taken
to the explanation of the terms. When there, you migh click on the page number after the term, to go back.


}
{Egy példa}
{

Ha \gls{minta}ként használja a \gls{szamitogep} fogalmát,
ahol \gls{CPU} valamint \gls{DMA} is előfordul
akkor a szövegben a \lstinline|Ha \\gls\{minta\}k\\'ent haszn\\'alja a \\gls\{szamitogep\} fogalm\\'at,
ahol \\gls\{CPU\} valamint \\gls\{DMA\} is el\\H\{o\}fordul|.
%\lstinline|Ha \\gls\{minta\}ként használja a \\gls\{számítógép\} fogalmát, ahol \\gls\{CPU\} valamint \gls\{DMA\} is előfordul| mintázatnak kell megjelennie.
Ilyenkor a MultEdu hozzáfűzi a dokumentumhoz a \lstinline|Acronyms| and \lstinline|Glossary| fejezeteket, ahol a megjelölt hivatkozások kifejtése 
található. A dokumentum olvasásakor a hivatkozásra kattintva, az olvasó 
program a kifejtésre ugrik, ahonnét az oldalszámra kattintva, folytathatja
az olvasást.

}


\MEDsubsection[Definition]{How to define acronyms and glossary}{Meghatározásuk}{Rövidítések és szómagyarázat meghatározása}


\MEDframe{Definitions}
{
MultEdu expects that (if you want to use this facility) your project
contains a file \lstinline|src/Glossary.tex|, where the expansion
of the referred to items can be found. The entries corresponding 
to the items used in the sample can be coded like

\lstinline|\\ifthenelse\{\\equal\{\\LectureLanguage\}\{english\}\}
  \{
	\\newglossaryentry\{computer\}
	\{
		name=\{computer\},
		description=\{is a programmable machine that receives input,
			stores and manipulates data, and provides
			output in a useful format\}
	\}	
	\\newglossaryentry\{sampleone\}\{name=\{sample\},description=\{a little example\}\}
    \\newacronym\{CPU\}\{CPU\}\{Central Processing Unit\}
    \\newacronym\{DMA\}\{DMA\}\{Direct Memory Access\}
 \}	
 \{\}|

}
{Egy példa}
{
A MultEdu azt várja, hogy (ha használni akar ilyen lehetőséget) a projekt
tartalmaz egy \lstinline|src/Glossary.tex| fájlt,
ahol a hivatkozások részletes kifejtése megtalálható.

}


\MEDsubsection[Utilization]{How to utilize acronyms and glossary}{Használatuk}{Rövidítések és szómagyarázat használata}

\MEDframe{Utilization}
{
These facilities can of course be only reasonably used in printable formats.
Formats based on \lstinline|beamer| do not generate such a list of terms,
but the \lstinline|\\gls\{ref\}|  are of course usable.
}
{Használatuk}
{
Ezeknek a lehetőségeknek csak a nyomtatható változatok esetén van szerepe.
A \lstinline|beamer| alapú formátumok nem generálnak ilyen jegyzékeket,
de a \lstinline|\\gls\{ref\}| természetesen ott is használható.
}

\MEDsection[Indices]{Indices}{Indexek}{Indexek használata}

\MEDsection[Bibliography]{Using bibliography}{Irodalom jegyzék}{Irodalom jegyzék}
