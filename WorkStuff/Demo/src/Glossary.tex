
\ifthenelse{\equal{\LectureLanguage}{english}}
  {
	\newglossaryentry{computer}
	{
		name={computer},
		description={is a programmable machine that receives input,
			stores and manipulates data, and provides
			output in a useful format}
	}
	
	\newglossaryentry{MultEdu}
	{
		name={MultEdu},
		description={MultEdu is a \LaTeX{}-based macro package, which produces .pdf output files in different formats and languages, from the same source file. Primarily for educational purposes.}
	}
	\newglossaryentry{sampleone}{name={sample},description={a little example}}

    \newacronym{CPU}{CPU}{Central Processing Unit}
    \newacronym{DMA}{DMA}{Direct Memory Access}
 }	
 {}
\ifthenelse{\equal{\LectureLanguage}{magyar}}
  {%% implement Hungarian documenting
	\newglossaryentry{szamitogep}
	{
		name={számítógép},
		description={olyan programozható gép, amelyik adatokat
		fogad, tárol és feldolgoz, valamint értelmes formátumú eredményt szolgáltat}
	}
	\newglossaryentry{MultEdu}
	{
		name={MultEdu},
		description={A MultEdu  \LaTeX{}-alapú makró csomag, ami különböző formátumú és nyelvű .pdf eredményfájlokat készít, ugyanabból a forrás fájlból. Elsősorban tananyag készítés céljára.}
	}
	\newglossaryentry{minta}{name={minta},description={egy minta}}
	\newglossaryentry{sampleone}{name={sample},description={a little example}}
    \newacronym{CPU}{CPU}{Central Processing Unit, központi egység}
    \newacronym{DMA}{DMA}{Direct Memory Access, közvetlen memória elérés}
  }
{
}