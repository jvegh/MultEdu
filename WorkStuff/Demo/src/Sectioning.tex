%%%
%%% This is the chapter about sectioning documents for the MultEdu system

\MEDchapter[Sectioning]{Sectioning document}{Tagolás}{A dokumentum tagolása}


\MESetListingFormat{TeX}
\lstset{ basicstyle=\ttfamily\color{black}\normalsize}
\MEDsection[Units]{Document units}{Egységek}{Dokumentum egységek}

\MEDframe{Document units}
{
Basically, the document must be organized as 'beamer' needs it,
but to print it in a book-like form, the sectioning must be changed,
and also the package 'beamerarticle' must be used.
In order to provide a uniform wrapper around sectioning, \gls{MultEdu} introduces
its own sectioning units. The source text itself comprises 'frames'.
}
{Dokumentum egységek}
{
A dokumentumot a 'beamer' csomag követelményeinek megfelelően kell szervezni.
A nyomtatható formában való megjelenítéshez a \gls{MultEdu}  a 'beamerarticle' csomagot használja,
és a tagolást is megfelelően változtatni kell.
Ennek érdekében a \gls{MultEdu} saját tagolási egységeket vezet be, amelyek valójában
a 'book' formátumnak felelnek meg és amelyeket dia készítéshez megfelelően átalakít. A szöveg viszont 'diakeret' egységekből áll össze.
}
\index{package!beamer}
\index{package!beamerarticle}

\MEDsubsection{Frames}{}{Dia keretek}
\MEDframe{Frames}
{
These units actually correspond to the ones used in format 'book',
and \gls{MultEdu} transforms them properly when preparing slides.

Usage:\par
\noindent\lstinline|\\MEframe[keys]\{subtitle\}\{content\}|

Legal keys are 

\noindent\lstinline!shrink=true|false! and \lstinline!plain=true|false!

By default, both are false.
}
{Dia keretek}
{

Használata:\par
\noindent\lstinline|\\MEframe[keys]\{subtitle\}\{content\}|

Értelmezett kulcsok 

\noindent\lstinline!shrink=true|false! and \lstinline!plain=true|false!

Alapértelmezetten mindkettő false.
}

\index{MEframe@\bs MEframe}

\MEDsubsection{Chapter}{}{Fejezet}
\MEDframe{Chapter}
{
Correspondingly, the biggest unit is the 'chapter'.
(As mentioned, for slides it is transformed to 'section'.)
Usage:\par
\noindent\lstinline|\\MEchapter[short title]\{long title\}|

When preparing slides, it is transformed to \lstinline|\\section|
}
{Fejezet}
{

A dokumentum legnagyobb egysége a fejezet.

Használata:\par
\noindent\lstinline|\\MEchapter[short title]\{long title\}|

Ha diákat készítünk,  \lstinline|\\section| lesz belőle.

}
\index{MEchapter@\bs MEchapter}

\MEDsubsection{Section and below}{}{Szakasz és az alatt}
\MEDframe{Chapter}
{
The next, smaller unit is the 'section'.
(As mentioned, for slides it is transformed to 'subsection'.)
Usage:\par
\noindent\lstinline|\\MEsection[short title]\{long title\}|

In a similar way, there exists

\lstinline|\\MEsubsection [short title] \{long title\}|
and

\lstinline|\\MEsubsubsection [short title] \{long title\}|

 the latter one is
transformed for slides to \lstinline|\\paragraph|.
}
{Fejezet}
{
A következő, kisebb egység a szakasz
Használata:\par
\noindent\lstinline|\\MEsection[r\"ovid cím]\{hosszú cím\}|

Hasonló módon létezik

\lstinline|\\MEsubsection [rövid cím] \{hosszú cím\}|


és

\lstinline|\\MEsubsubsection [rövid cím] \{hosszú cím\}|; 

ez utóbbi dia készítés esetén \lstinline|\\paragraph| alakot ölt.

}
\index{MEsection@\bs MEsection}

\MEDsection{Dual language sources}{}{Kétnyelvű forráskódok}

\MEDframe{Dual language sources}
{
It happens, that I teach the same course in my mother tongue for my domestic students,
and in English, for foreign students.
The course material is the same, and it must be developed in parallel.
Obviously it is advantageous, if they are located in the same source file, side by side;
so they can be developed in the same action.
The \lstinline|\\UseSecondLanguage| macro supports this method.

The macros introduced above have a version with prefix 'MED' rather than 'ME' only,
which takes double argument sets (arguments for both languages). Depending on whether \lstinline|\\UseSecondLanguage| is defined,
the first or the second argument set is used.
}
{Kétnyelvű forráskódok}
{
Előfordul, hogy ugyanazt az anyagot saját nyelvemen oktatom hazai hallgatóknak,
és angolul, külföldi hallgatóknak. A tananyag megegyezik, és együtt kell fejleszteni.
Nyilván előnyös, ha a két anyag ugyanabban a forrásnyelvi fájlban, egymás mellett fejleszthető.

Erre szolgál a \lstinline|\\UseSecondLanguage|. A fent bevezetett makróknak van egy 'D' (Dual) taggal
kibővített változata, amelyikben mind az elsődleges, mind a másodlagos nyelven 
megadjuk a szükséges tartalmakat.
}
\index{UseSecondLanguage@\bs UseSecondLanguage}
\MEDsubsection{Switching between languages}{}{Átváltás a nyelvek között}

\MEDframe{Switching between languages}
{

Usage:

\noindent\lstinline|\\UseSecondLanguage\{YES\}|

\noindent where the argument in \lstinline|\{\}| is not relevant, only if this macro is defined or not.

The two kinds of macros can be mixed, but only the 'D' macros 
are sensitive to changing the language.
}
{Átváltás a nyelvek között}
{

Használata:

\noindent\lstinline|\\UseSecondLanguage\{YES\}|

\noindent ahol az \lstinline|\{\}|-ben megjelenő argumentum nem számít, csak az, hogy definiálva van-e ez a makro.

A kétféle makrókészlet keverhető, de csak a 'D' makrók reagálnak 
a nyelv változtatásra.
}

\MEDsubsection{Frames}{}{Dia keretek}

\MEDframe{Frames}
{
In dual language documents, usually 
\par\noindent\lstinline|\\MEDframe[keys]\{subtitle, first language\} \{content, first language \} \{subtitle, second language\} \{content, second language\}|
\par\noindent is used. I.e. the user provides titles and contents in both languages,
and for preparing the output, selects one of them with \lstinline|\\UseSecondLanguage|.
}
{Dia keretek}
{
Kétnyelvű dokumentumokban általában a
\par\noindent\lstinline|\\MEDframe[keys]\{subtitle, first language\} \{content, first language \} \{subtitle, second language\} \{content, second language\}|
\par\noindent  keretet használjuk. Azaz a felhasználó megadja mindkét nyelven
a címet és a tartalmat, majd fordítás előtt \lstinline|\\UseSecondLanguage|
használatával kiválasztja az egyik nyelvet.
}

\index{MEDframe@\bs MEDframe}

\MEDframe{Timing for presenting slide series}
{
Although that point of view has less importance, when presenting a conference talk,
it is very important to properly utilize the available time.
\gls{MultEdu} can support this through displaying a chrono time on the slides. An example:

\lstinline|\\MEframe\{Frame title \\ifx\\EnableTimer\\undefined \\else \\initclock\\fi\}|

The \gls{MultEdu} also warns the with changing the color of the time value,
if we are approaching the end of the lecture.
The maximum time can be set using instruction \lstinline|\\def\\LectureTime\{minutes\}|,
the default value is \lstinline|15|.
\ao{The timer starts when the second slide is projected, and is reset, when
the lecturer returns to that slide.}
\index{time on slides}
}
{Időmérés diasorozat vetítésekor}
{
Bár tananyag készítésekor kevésbé fontos szempont, egy konferencia előadás bemutatásakor
nagyon fontos az előadásra szánt idő megfelelő felhasználása.
A \gls{MultEdu} a kivetített diákon a felhasznált idő  kivetítésével tudja ezt támogatni.
Ez a lehetőség alapállapotban tiltott, külön engedélyezni kell \lstinline|\\def\\EnableTimer\{YES\}| utasítással,
célszerűen a \lstinline|src/Defines.tex| fájlban.
Ezt az utasítást célszerű az első "valódi" keret címében elhelyezni, különben üres keretet eredményezhet.
Példa:

\lstinline|\\MEframe\{Keret cim \\ifx\\EnableTimer\\undefined \\else \\initclock\\fi\}|

A \gls{MultEdu} a kijelzett idő színének megváltoztásával figyelmezteti az előadót, ha az előadás végéhez közel kerül. A maximális értéket

\lstinline|\\def\\LectureTime\{perc\}|

utasítással 
lehet beállítani, az alapértelmezett érték \lstinline|15|.
\ao{Az időmérés a második dia megjelenytésekor indul, az idő újra indul,
ha oda visszamegyünk.}
\index{idő kijelzése dián}
}
\index{src/Defines.tex}
\index{EnableTimer@\bs EnableTimer}
\index{LectureTime@\bs LectureTime}


\MEDsubsection{Chapter}{}{Fejezet}

\MEDframe{Chapter}
{
Correspondingly, the biggest unit in a dual language document is the 'Dchapter'.
(As mentioned, for slides it is transformed to 'Dsection'.)
Usage:\par
\noindent\lstinline|\\MEDchapter[short title1]\{long title1\}\{short title2\}\{long title2\}|
\par\noindent which is transformed to
\par\noindent\lstinline|\\MEchapter[short title1]\{long title1\}| or
\par\noindent\lstinline|\\MEchapter[short title2]\{long title2\}| calls,
\par\noindent depending on whether \lstinline|\\UseSecondLanguage| is or is not defined.

The usage of the lower units is absolutely analogous.

}
{Fejezet}
{

Hasonlóképpen, a kétnyelvű dokumentum legnagyobb egysége a 'Dchapter'.
(Amint említettük, dia készítéskor ez átalakul 'Dsection' egységgé.)
Használata:\par
\noindent\lstinline|\\MEDchapter[rövid cím1]\{hosszú cím1\}\{rövid cím2\}\{hosszú cím2\}|
\par\noindent ami aztán átalakul 
\par\noindent\lstinline|\\MEchapter[rövid cím1]\{hosszú cím1\}| vagy
\par\noindent\lstinline|\\MEchapter[rövid cím2]\{hosszú cím2\}| 
\par\noindent attól függően, hogy  \lstinline|\\UseSecondLanguage| definiált vagy sem.

Teljesen hasonló a kisebb formázási egységek használata is.

}
\index{MEDchapter@\bs MEDchapter}
\index{MEDsection@\bs MEDsection}

\MEDsection[]{Chapter illustration}{}{Fejezet illusztráció}

\MEDframe{Chapter illustration}
{
Some book styles also allow presenting some illustration at the beginning of the chapters.
\par Usage:\par
\noindent\lstinline|\\MEchapterillustration\{file\}|

\noindent For slides, the illustration appears in a 'plain' style style.
For books, the picture is placed at the beginning of the chapter.
If the file name is empty, a \lstinline|fig/DefaultIllustration.png| file is searched.
If the file not found, no illustration generated.

If macro \lstinline|\\DisableChapterIllustration| is defined, no picture generated.

}
{Fejezet illusztráció}
{
Néhány könyv stílus lehetővé teszi, hogy a fejezetek elején egy illusztrációt helyezzünk el.
\par Használata:\par
\noindent\lstinline|\\MEchapterillustration\{file\}|

\noindent Dia készítéskor, a kép egy 'plain' dián jelenik meg.
Nyomtatható változatban a fejezet elején jelenik meg a kép.

Ha a fájl név üres, a csomag a \lstinline|fig/DefaultIllustration.png| képet keresi.
Ha a fájl nem található, nem készül illusztráció.

Ha definiáljuk a  \lstinline|\\DisableChapterIllustration|  makrót,
a csomag nem generál képet. 
}
\index{MEchapterillustration@\bs MEchapterillustration}

\MEDsection[Slides and printed]{Concerting text on slides and printed output}{Nyomtatott és vetített szöveg}{Nyomtatott és vetített szöveg összehangolása}

\MEDframe{Concerting text on slides and printed outputs}
{
The printed outputs usually contain much more text, than the slides.
This extra text can be placed in the source file
inside an \lstinline|\\ao\{text\}| (article only) macro,
where the extra text appears inside the macro.
That text appears only in the printed output, and is not visible on the slides.
Take case, the text must be reasonable in both version;
especially if used within a sentence.
}
{Nyomtatott és vetített szöveg}
{A nyomtatott anyag jelentősen több szöveget szokott tartalmazni, mint
a diák. Ezt az extra szöveget úgy lehet a forrás fájlban elhelyezni,
hogy az \lstinline|\\ao\{text\}| (article only) makró belsejében adjuk meg az extra szöveget.
Az így megadott szöveg csak a nyomtatott változatban látható, 
a diákon nem jelenik meg. 
Vigyázzunk rá, hogy a szöveg mindkét változatban értelmes legyen,
különösen, ha mondat belsejében használjuk.
}


\MEDsection[]{Floating objects}{}{Lebegő objektumok}

\MEDframe{Floating objects}
{
\LaTeX{} might handle objects like figures, tables, program listings, etc.
as "floating objects, i.e. they might appear at a place,
where \LaTeX{} thinks to be optimal.
This place is not necessarily  the place in the printed materials,
what you expect based on the referece point in the source
but they do on the slides.
Because of this, do not refer to the listings like
'In the following listing'.
Instead, using something like  '\lstinline|In listing\ \\ref\{lst:hello.cpp\}|' is suggested.

In contrast, on the slides the lobject appears in the right place,
but has no number. Because of this the really good method of referencing
is something like '\lstinline|In listing\ \\ao\{\\ref\{lst:hello.cpp\}\}|' 
is the really good one.
Take care of the meaning in the sentence, both on slides and printed output.

}
{Lebegő objektumok}
{
A \LaTeX{} bizonyos objektumokat, úgymint ábrákat, táblázatokat, programlistákat, stb. ún
lebegő objektumként kezelhet, tehát nem feltétlenül a forrásnyelvi helynek 
megfelelő helyen jelennek meg a nyomtatott változatban, viszont a dia képeken igen.
Ezért a nyomtatott változatban nem érdemes 'A következő programlistán' módon hivatkozni. Helyette
az '\lstinline|\\Aref\{lst:hello.cpp\}\ programlista|' mód javasolt.
\ao{Az \lstinline|\\Aref| formájó makró csak magyarul használatos
és az objektum számának megfelelő névelőt használ.}

A dia képeken viszont  a megfelelő helyen van a lista, de nincs száma.
Ezért az '\lstinline|A\ \\ao\{\\ref\{lst:hello.cpp\}\}\ programlista|' mód az igazi.
}