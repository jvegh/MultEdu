%%%
%%% This is the chapter about sectioning documents for the MultEdu system

\MEDchapter{Sectioning document}{}{A dokumentum tagolása}


\MESetListingFormat{TeX}
\lstset{ basicstyle=\ttfamily\color{black}\normalsize}
\MEDsection[]{Document units}{}{Dokumentum egységek}

\MEDframe{Document units}
{
Basically, the document is organized as 'beamer' needs it,
but to print it in a book-like form, the sectioning must be changed,
and also the package 'beamerarticle' must be used.
In order to provide a uniform wrapper around sectioning, MultEdu introduces
its own sectioning units.
}
{Dokumentum egységek}
{
A dokumentumok a 'beamer' csomag követelményeinek megfelelően van szervezve.
A nyomtatható formában való megjelenítéshez a MultEdu  a 'beamerarticle' csomagot használja,
és a tagolást is megfelelően változtatni kell.
Ennek érdekében a MultEdu saját tagolási egységeket vezet be, amelyek valójában
a 'book' formátumnak felelnek meg, és amelyeket dia készítéshez megfelelően átalakít.
}
\index{package!beamer}
\index{package!beamerarticle}

\MEDsubsection{Frames}{}{Dia keretek}
\MEDframe{Frames}
{
These units actually correspond to the ones used in format 'book',
and MultEdu transforms them properly when preparing slides.

Usage:\par
\noindent\lstinline|\\MEframe[keys]\{subtitle\}\{content\}|

Legal keys are 

\noindent\lstinline!shrink=true|false! and \lstinline!plain=true|false!

By default, both are false.
}
{Dia keretek}
{

Használata:\par
\noindent\lstinline|\\MEframe[keys]\{subtitle\}\{content\}|

Értelmezett kulcsok 

\noindent\lstinline!shrink=true|false! and \lstinline!plain=true|false!

Alapértelmezetten mindkettő false.
}

\index{MEframe@\bs MEframe}

\MEDsubsection{Chapter}{}{Fejezet}
\MEDframe{Chapter}
{
Correspondingly, the biggest unit is the 'chapter'.
(As mentioned, for slides it is transformed to 'section'.)
Usage:\par
\noindent\lstinline|\\MEchapter[short title]\{long title\}|

}
{Fejezet}
{

A dokumentum legnagyobb egysége a fejezet.

Használata:\par
\noindent\lstinline|\\MEchapter[short title]\{long title\}|

}
\index{MEchapter@\bs MEchapter}

\MEDsubsection{Section and below}{}{Szakasz és az alatt}
\MEDframe{Chapter}
{
The next, smaller unit is the 'section'.
(As mentioned, for slides it is transformed to 'subsection'.)
Usage:\par
\noindent\lstinline|\\MEsection[short title]\{long title\}|

In a similar way, there exists \lstinline|\\MEsubsection[short title]\{long title\}|
and \lstinline|\\MEsubsubsection[short title]\{long title\}|; the latter one is
transformed for slides to \lstinline|\\paragraph|
}
{Fejezet}
{
A következő, kisebb egység a szakasz
Használata:\par
\noindent\lstinline|\\MEsection[r\"ovid cím]\{hosszú cím\}|

Hasonló módon létezik \lstinline|\\MEsubsection[rövid cím]\{hosszú cím\}|
és \lstinline|\\MEsubsubsection[rövid cím]\{hosszú cím\}|; 
ez utóbbi dia készítés esetén \lstinline|\\paragraph| alakot ölt.

}
\index{MEsection@\bs MEsection}

\MEDsection{Dual language sources}{}{Kétnyelvű forráskódok}

\MEDframe{Dual language sources}
{
It happens, that I teach the same course in my mother tongue for my domestic students,
and in English, for foreign students.
The course material is the same, and it must be developed in parallel.
Obviously it is advantageous, if they are located in the same source file, side by side;
so they can be developed in the same action.
The \lstinline|\\UseSecondLanguage| macro supports this method.

The macros introduced above have a version with prefix 'MED' rather than 'ME only,
which takes double argument sets (arguments for both languages). Depending on whether \lstinline|\\UseSecondLanguage| is defined,
the first or the second argument set is used. In this way, just the 'src/Heading.tex'
source file shall be edited, and the other language output will be prepared.
}
{Kétnyelvű forráskódok}
{
Előfordul, hogy ugyanazt az anyagot saját nyelvemen oktatom hazai hallgatóknak,
és angolul, külföldi hallgatóknak. A tananyag megegyezik, és együtt kell fejleszteni.
Nyilván előnyös, ha a két anyag ugyanabban a forrásnyelvi fájlban, egymás mellett fejleszthető.

Erre szolgál a \lstinline|\\UseSecondLanguage|. A fent bevezetett makróknak van egy 'D' (Dual) taggal
kibővített változata, amelyikben mind az elsődleges, mind a másodlagos nyelven 
megadjuk a szükséges tartalmakat. A 'Heading.tex' file megváltoztatásával készíthetünk 
más nyelvű dokumentumot.
}
\index{UseSecondLanguage@\bs UseSecondLanguage}
\MEDsubsection{Switching between languages}{}{Átváltás a nyelvek között}

\MEDframe{Switching between languages}
{

Usage:

\noindent\lstinline|\\UseSecondLanguage\{YES\}|

\noindent where the argument \lstinline|\{\}| is not relevant, only if this macro is defined or not.
Traditionally, it is defined in 'src/Heading.tex'.

The two kinds of macros can be mixed, but only the 'D' macros 
are sensitive to changing the language.
}
{Átváltás a nyelvek között}
{

Használata:

\noindent\lstinline|\\UseSecondLanguage\{YES\}|

\noindent ahol az \lstinline|\{\}| nem számít, csak az, hogy definiálva van-e ez a makro.
Hagyományosan az 'src/Heading.tex' fájlban definiáljuk.

A kétféle makrókészlet keverhető, de csak a 'D' makrók reagálnak 
a nyelv változtatásra.
}

\MEDsubsection{Frames}{}{Dia keretek}

\MEDframe{Frames}
{
In dual language documents, usually 
\par\noindent\lstinline|\\MEDframe[keys]\{subtitle, first language\} \{content, first language \} \{subtitle, second language\} \{content, second language\}|
\par\noindent is used. I.e. the user provides titles and contents in both languages,
and for preparing the output, selects one of them with \lstinline|\\UseSecondLanguage|.
}
{Dia keretek}
{
Kétnyelvű dokumentumokban általában a
\par\noindent\lstinline|\\MEDframe[keys]\{subtitle, first language\} \{content, first language \} \{subtitle, second language\} \{content, second language\}|
\par\noindent  keretet használjuk. Azaz a felhasználó megadja mindkét nyelven
a címet és a tartalmat, majd fordítás előtt \lstinline|\\UseSecondLanguage|
használatával kiválasztja az egyik nyelvet.
}

\index{MEDframe@\bs MEDframe}

\MEDsubsection{Chapter}{}{Fejezet}

\MEDframe{Chapter}
{
Correspondingly, the biggest unit in a dual language document is the 'Dchapter'.
(As mentioned, for slides it is transformed to 'Dsection'.)
Usage:\par
\noindent\lstinline|\\MEDchapter[short title1]\{long title1\}\{short title2\}\{long title2\}|
\par\noindent which is transformed to
\par\noindent\lstinline|\\MEchapter[short title1]\{long title1\}| or
\par\noindent\lstinline|\\MEchapter[short title2]\{long title2\}| calls,
\par\noindent depending on whether \lstinline|\\UseSecondLanguage| is or is not defined.
}
{Fejezet}
{

Hasonlóképpen, a kétnyelvű dokumentum legnagyobb egysége a 'Dchapter'.
(Amint említettük, dia készítéskor ez átalakul 'Dsection' egységgé.)
Használata:\par
\noindent\lstinline|\\MEDchapter[rövid cím1]\{hosszú cím1\}\{rövid cím2\}\{hosszú cím2\}|
\par\noindent ami aztán átalakul 
\par\noindent\lstinline|\\MEchapter[rövid cím1]\{hosszú cím1\}| vagy
\par\noindent\lstinline|\\MEchapter[rövid cím2]\{hosszú cím2\}| 
\par\noindent attól függően, hogy  \lstinline|\\UseSecondLanguage| definiált vagy sem.
}
\index{MEDchapter@\bs MEDchapter}


\MEDsubsection{Section and below}{}{Szakasz és az alatt}

\MEDframe{Section and below}
{
The usage of the lower units is absolutely analogous.
}
{Szakasz és az alatt}
{
Teljesen hasonló a kisebb formázási egységek használata is.
}
\index{MEDsection@\bs MEDsection}

\MEDsection[]{Chapter illustration}{}{Fejezet illusztráció}

\MEDframe{Chapter illustration}
{
Some book styles also allow presenting some illustration at the beginning of the chapters.
\par Usage:\par
\noindent\lstinline|\\MEchapterillustration\{file\}|

\noindent For slides, the illustration appears in a 'plain' style style.
For books, the picture is placed at the beginning of the chapter.
If the file name is empty, a 'fig/DefaultIllustration.png' file is searched.
If the file not found, no illustration generated.

If macro \lstinline|\\DisableChapterIllustration| is defined, no picture generated.

}
{Fejezet illusztráció}
{
Néhány könyv stílus lehetővé teszi, hogy a fejezetek elején egy illusztrációt helyezzünk el.
\par Használata:\par
\noindent\lstinline|\\MEchapterillustration\{file\}|

\noindent Dia készítéskor, a kép egy 'plain' dián jelenik meg.
Nyomtatható változatban a fejezet elején jelenik meg a kép.

Ha a fájl név üres, a csomag a 'fig/DefaultIllustration.png' képet keresi.
Ha a fájl nem található, nem készül illusztráció.

Ha definiáljuk a  \lstinline|\\DisableChapterIllustration|  makrót,
a csomag nem generál képet. 
}
\index{MEchapterillustration@\bs MEchapterillustration}

