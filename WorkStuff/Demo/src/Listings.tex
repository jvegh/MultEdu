%%%
%%% This is the chapter about showing program listings for the MultEdu system

\MEDchapter[Listings]{Preparing program listings}{Program listák}{Program listák készítése}


\MESetListingFormat[basicstyle=\ttfamily\color{black}\normalsize]{TeX}


\MEDframe{Preparing program listings}
{
When teaching programming, it is a frequent need to display program listings.
Through using package 'listings', MultEdu can implement this in very good quality.
For details not described here see documentation of package 'listings'.

Notice that here the ratio of the listings within the text is unusually high,
so it is very hard for the compiler to find good positioning. In the case of 
real texts, the page is much more aesthetic.
}
{Program listák készítése}
{
Programozás tanításakor alapvető követelmény programlisták megjelenítése.
A 'listings' csomag felhasználásával a MultEdu ezt jó minőségben tudja biztosítani.
Az itt nem ismertetett részletekért lásd a 'listings' csomag leírását.

Ebben a szakaszban szokatlanul sok elhelyezendő programlista van,
ami nagyon megnehezíti a fordítóprogram dolgát. Valódi szövegek esetén
az készített oldal sokkal esztétikusabb.
}
\index{package!listings}

\MEDsection[Appearance]{Setting appearance}{Megjelenítés}{A megjelenítés beállítása}

\MEDframe{Setting appearance}
{
Package 'listings' allows to set up the style of displaying program listings
according to our taste (and the requirements).
MultEdu pre-sets some style and allows to modify it as much as you like.

Macro \par\noindent\lstinline|\\MESetStandardListingFormat| sets up a default appearance,
and no programming language. Macro \par\noindent\lstinline|\\MESetListingFormat[options]\{language\}|
\par\noindent sets the language, the same appearance as macro \par\noindent\lstinline|\\MESetStandardListingFormat|
\par\noindent and also allows to overwrite parameters of 'listings' through 'options'.
}
{A megjelenítés beállítása}
{
A 'listings' csomag számos lehetőséget biztosít arra, hogy a programlista megjelenítés stílusát ízlésünknek
(és a követelményeknek) megfelelően állítsuk be.

\noindent\lstinline|\\MESetStandardListingFormat| 
\par\noindent beállít egy alap-megjelenítést,
de nem állít be programnyelvet.
\par\noindent \lstinline|\\MESetListingFormat[options]\{language\}|
\par\noindent beállítja a nyelvet, és
\par\noindent\lstinline|\\MESetStandardListingFormat| 
\par\noindent szerinti alap-megjelenítést
és 'options' használatával lehetővé teszi a 'listings' alapértelmezett argumentumainak felülírását.
}
\index{MESetStandardListingFormat@\bs MESetStandardListingFormat}
\index{MESetListingFormat@\bs MESetListingFormat}


\MEDsection[Code fragments]{Displaying inline fragments}{Töredék kód}{Sorközi töredék megjelenítése}

\MEDframe{Displaying inline fragments}
{
A tipical task is to display a shorter fragment, like a line or a keyword.
It is possible using \lstinline!\\lstinline|code|!.

The LaTeX commands appearing in this documentation are produced in such a way that
at the beginning of the chapter commands


\noindent\lstinline|\\MESetListingFormat\{TeX\}|\par
\noindent\lstinline|\\lstset\{basicstyle= \\ttfamily\\color\{black\}\\normalsize\}|

\noindent or

\noindent\lstinline|\\MESetListingFormat[basicstyle= \\ttfamily\\color\{black\}\\normalsize]\{TeX\}|

\noindent are issued (otherwise the character size of the program text will be too small).
}
{Sorközi töredék megjelenítése}
{
Gyakori feladat egy rövidebb töredék, mint egyetlen sor vagy akár kulcsszó/változó
megjelenítése. Ezt a  \lstinline!\\lstinline|code|! módon tehetjük meg.

Az ebben a leírásban is kiterjedten használt LaTeX parancsok megjelenítéséhez
a fejezet elején használok egy 


\noindent\lstinline|\\MESetListingFormat\{TeX\}|\par
\noindent\lstinline|\\lstset\{basicstyle= \\ttfamily\\color\{black\}\\normalsize\}|

\noindent vagy

\noindent\lstinline|\\MESetListingFormat[basicstyle= \\ttfamily\\color\{black\}\\normalsize]\{TeX\}|

\noindent parancsot. (különben túl kicsi lesz a megjelenített program kód karaktereinek mérete)
}

\MEDsection[Full code]{Displaying program listings}{Teljes program}{Teljes programlista megjelenítése}

\MEDframe{Displaying program listings}
{

Program listings can be displayed using macro
\par\noindent\lstinline|\\MESourceFile[keys] \{filename\} \{caption\} \{label\}\{scale\}|.
Possible keys: \lstinline|wide,decorations|.

\MESourceFile[language={[ISO]C++}]{lst/HelloWorld.cpp}{"Hello World" -- a C++ way}{lst:hello.cpp}{}

\par\noindent The command used to display Listing\ao{~\ref{lst:hello.cpp}} was
\par\noindent\lstinline|\\MESourceFile[language=\{[ISO]C++\}] \{lst/HelloWorld.cpp\} \{A "Hello World"- C++ program\} \{lst:hello.cpp\}\{\}|


}
{Teljes programlista megjelenítése}
{
A
\par\noindent\lstinline|\\MESourceFile[keys] \{filename\} \{caption\} \{label\}\{scale\}|
\par\noindent makróval jeleníthetők meg programlisták.
Lehetséges kulcs: \lstinline|wide,decorations|.
A \ao{\ref{lst:hello.cpp}} programlista megjelenítéshez használt programsor:
\par\noindent\lstinline|\\MESourceFile[language=\{[ISO]C++\}] \{lst/HelloWorld.cpp\} \{A "Hello World"- C++ program\} \{lst:hello.cpp\}\{\}|

\MESourceFile[language={[ISO]C++}]{lst/HelloWorld.cpp}{A "Hello World"- C++ program}{lst:hello.cpp}{}


}
\index{MESourceFile@\bs MESourceFile}

\MEDframe{Wide program listings}
{
Many times one needs wider program listings.
In the case of the two-column printing, 
the listing shall fill the width of the two columns.
In the case of one-column printing, the narrow list extend to 70\% of the text width,
while the wide lists 

The wide listings can be placed even hardly on the printed page
(the first proper place, relative to the appearance of the macro
is the top of the next page), and in addition, the orders of 
normal and wide listings cannot be changed.
Because of this, the place where the listing appears,
might be relatively far from the place of referencing it.
}
{Széles programlista megjelenítése}
{
Sokszor van szükség szélesebb programlista megjelenítésére. Ennek hatására a két
oszlopos nyomtatás teljes szélességében jelenik meg a lista.
Egyoszlopos megjelenítés esetén a keskeny lista az oldalszélesség 70\%-ára terjed ki,
a széles pedig a teljes oldal szélességet igénybe veszi.
A széles programlistákat még nehezebb elhelyezni az oldalon (a megjelenítő utasítás helye
utáni oldal tetejére kerülhet legelőször), ráadásul nem is szabad
felcserélni a normál és széles programlisták sorrendjét.
Emiatt a megjelenési hely eléggé messze is kerülhet
a hivatkozás helyétől.
}

\MEDframe{Wide program listings}
{

\MESourceFile[language={[ISO]C++},wide]{lst/HelloWorld.cpp}{A "Hello World"- C++ program, wide}{lst:Whello.cpp}{}
\index{MESourceFile@\bs MESourceFile}

The command used to display Listing \ao{\ref{lst:Whello.cpp}} :
\par\noindent\lstinline|\\MESourceFile[language=\{[ISO]C++\},wide] \{lst/HelloWorld.cpp\} \{A "Hello World"- C++ program, wide\} \{lst:Whello.cpp\}\{\}|
}
{Széles programlista megjelenítése}
{

\MESourceFile[language={[ISO]C++},wide]{lst/HelloWorld.cpp}{A "Hello World"- C++ program, wide}{lst:Whello.cpp}{}

A \ao{\ref{lst:Whello.cpp}} programlista megjelenítéséhez használt programsor:
\par\noindent\lstinline|\\MESourceFile[language=\{[ISO]C++\},wide] \{lst/HelloWorld.cpp\} \{A "Hello World"- C++ program, wide\} \{lst:Whello.cpp\}\{\}|
}

\MEDsection[Decorations]{Decorations on listings}{Díszítések}{Programlista díszítései}

\MEDframe{Decorations on listings}
{
Different decorations can be placed on top of listings.
To do so, one has to use the keyword  \lstinline|decorations|,
and to insert as arguments the macros presented in this section.

The general form:
\par\noindent\lstinline|\\MESourceFile[options,
decorations=\{
list of decorations
\}
] \{source file\} \{caption\} \{label\}\{\}|
\par\noindent where the list of decorations may contain any of the 
decoration macros presented in the section.\ao{\footnote{
The compiler prepares in the first pass the program listing,
the decorations follow in the next pass.}}
In \lstinline|options| any option, used by package 'listings' applies.
\index{MESourceFile!decorations@\bs MESourceFile decorations}
}
{Programlista díszítései}
{
A programlistán különféle díszítményeket helyezhetünk el.
Ehhez a programlista készítésekor használnunk kell a \lstinline|decorations| kulcsszót
és annak argumentumaként az e szakaszban bemutatott makrókat kell megadni.

Az általános forma:
\par\noindent\lstinline|\\MESourceFile[options,
decorations=\{
list of decorations
\}
] \{source file\} \{caption\} \{label\}\{\}|
\par\noindent ahol a dekorációk listája a szakaszban felsorolt 
bármelyik fajta dekorációt tartalmazhatja.\ao{\footnote{
A program az első menetben elhelyezi a programlistát és egy újabb fordítás során tudja 
a díszítéseket is felrakni.}}
Az \lstinline|options| argumentumaként a 'listings' csomagban használt bármely opció használható.
%\index{MESourceFile!díszítés@\bs MESourceFile díszítés}
}

\MEDsubsection[Highlighting]{Highlighting lines}{Kijelölés}{Programsorok kijelölése}

%\MEDframe{Highlighting lines}
%{
%In a program listing, a range of source lines can be highlighted using macro 
%\par\noindent\lstinline|\\\MESourcelinesHighlight \{BallonName\} \{SourceName\} \{FirstLine\} \{LastLine\}|.
%\par\noindent Here  \lstinline|BallonName| is a new label, which denotes the 
%newly created source line region highlighting box, \lstinline|SourceName| 
%is the label of the source file. 
%}
%{Programsorok kijelölése}
%{
%Egy programlistán a
%\par\noindent\lstinline|\\MESourcelinesHighlight \{BallonName\} \{SourceName\} \{FirstLine\} \{LastLine\}|
%\par\noindent makró használatával jelölhetünk ki programsorokat. Itt \lstinline|BallonName|
%az az új címke, amit az így létrehozott sortartományra húzott burkoló kap, \lstinline|SourceName| pedig a
%forrásfájl címkéje. 
%}

\MEDframe{Highlighting lines}
{
To highlight a program body in listing
 \ao{\ref{lst:HLhello.cpp}} the macro

\par\noindent\lstinline|\\MESourceFile[language=\{[ISO]C++\},
decorations=\{
\\MESourcelinesHighlight \{HelloBalloon\} \{lst:HLhello.cpp\} \{6\}\{8\}
\}
] \{lst/HelloWorld.cpp\} \{"Hello World" -- a C++ way, kijelölt\} \{lst:HLhello.cpp\}\{\}|

\par\noindent shall be used.

\MESourceFile[language={[ISO]C++},
decorations={
\MESourcelinesHighlight{HelloBalloon}{lst:HLhello.cpp}{6}{8}
}
]{lst/HelloWorld.cpp}{"Hello World" -- a C++ way, highlighted}{lst:HLhello.cpp}{}
\index{MESourceFile!highlighting@\bs MESourceFile highlighting}
}
{Programsorok kijelölése}
{
A \ao{\ref{lst:HLhello.cpp}} programlistán a programtörzs utasításainak
kijelöléséhez a


\par\noindent\lstinline|\\MESourceFile[language=\{[ISO]C++\},
decorations=\{
\\MESourcelinesHighligh \{HelloBalloon\} \{lst:HLhello.cpp\} \{6\}\{8\}
\}
] \{lst/HelloWorld.cpp\} \{"Hello World" -- a C++ way, kijelölt\} \{lst:HLhello.cpp\}\{\}|

\par\noindent parancsot kell kiadni.

\MESourceFile[language={[ISO]C++},
decorations={
\MESourcelinesHighlight{HelloBalloon}{lst:HLhello.cpp}{6}{8}
}
]{lst/HelloWorld.cpp}{"Hello World" -- a C++ way, kijelölt}{lst:HLhello.cpp}{}
%\index{MESourceFile!sorkijelölés@\bs MESourceFile sorkijelölés}
}
\index{MESourcelinesHighlight@\bs MESourcelinesHighlight}

\MEDsubsection[Commenting]{Commenting highlighted lines}{Megjegyzések}{Megjegyzés kijelölt programsorokhoz}

\MEDframe{Commenting highlighted lines}
{
The higlighting box can also be commented. Using macro
\par\noindent\lstinline|\\MESourceBalloonComment[keys]\{BallonName\} \{ShiftPosition\} \{Comment\} \{CommentShape\}|
\par\noindent allows to comment the balloon created previously.
Here \lstinline|BallonName| is the first argument of  \lstinline|\\MESourcelinesHighlight|,
\lstinline|ShiftPosition| is the shift of the comment box, \lstinline|Comment| is the comment text.
Possible keys, with defaults are:
\par\noindent \lstinline|width[=3cm]| and \lstinline|color[=deeppeach]|.

}
{Megjegyzés a kijelöléshez}
{
Az előbbi programlistán a kijelöléshez megjegyzést is fűzhetünk.
Ennek formája 
\par\noindent\lstinline|\\MEBalloonComment[keys]\{BallonName\} \{ShiftPosition\} \{Comment\} \{CommentShape\}|
amivel az előzőleg felrajzolt ballonhoz fűzhetünk megjegyzést.
Itt \lstinline|BallonName| az \lstinline|\\MEHighlightLines| első argumentuma,
\lstinline|ShiftPosition| a megjegyzésdoboz eltolása, \lstinline|Comment| pedig maga 
a megjegyzés.
A lehetséges opciók: \lstinline|width[=3cm]| és \lstinline|color[=deeppeach]|.

}

\MEDframe{Commenting highlighted lines}
{
Listing \ao{\ref{lst:HLChello.cpp}} is produced using macro
\par\noindent\lstinline|\\MESourceFile[language=\{[ISO]C++\},wide,
decorations=\{
\\MESourcelinesHighlight \{HelloBalloon\} \{lst:HLChello.cpp\} \{6\}\{8\}
\\MESourceBalloonComment\{HelloCBalloon\} \{0cm,0cm\} \{This is the body\} \{CommentShape\}
\} 
] \{lst/HelloWorld.cpp\} \{"Hello World" -- a C++ way, commenting highlighted\} \{lst:HLhello.cpp\}\{\}|

\MESourceFile[language={[ISO]C++},wide,
decorations={
\MESourcelinesHighlight{HelloCBalloon}{lst:HLChello.cpp}{6}{8}
\MESourceBalloonComment{HelloCBalloon}{0cm,0cm}{This is the body}{CommentShape}
}
]{lst/HelloWorld.cpp}{"Hello World" -- a C++ way, remark to the highlighing}{lst:HLChello.cpp}{}
\index{MESourceFile!highlighting@\bs MESourceFile highlighting}
}
{Megjegyzés a kijelöléshez}
{
A \ao{\ref{lst:HLChello.cpp}} programlista készítéséhez a

\par\noindent\lstinline|\\MESourceFile[language=\{[ISO]C++\},wide,
decorations=\{
\\MESourcelinesHighlight \{HelloBalloon\} \{lst:HLChello.cpp\} \{6\}\{8\}
\\MESourceBalloonComment \{HelloCBalloon\} \{0cm,0cm\} \{This is the body\} \{CommentShape\}
\} 
] \{lst/HelloWorld.cpp\} \{"Hello World" -- a C++ way
%, megjegyzés a kijelöléshez
\} \{lst:HLhello.cpp\}\{\}|


parancsot kell kiadni.


\MESourceFile[language={[ISO]C++},wide,
decorations={
\MESourcelinesHighlight{HelloBalloonC}{lst:HLChello.cpp}{6}{8}
\MESourceBalloonComment{HelloBalloonC}{0cm,0cm}{This is the body}{CommentShape}
}
]{lst/HelloWorld.cpp}{"Hello World" -- a C++ way %, megjegyzés a kijelöléshez
}{lst:HLChello.cpp}{}
%\index{MESourceFile!tartomány kijelölés@\bs MESourceFile tartomány kijelölés}
}
\index{MESourcelinesHighlight@\bs MESourcelinesHighlight}

\MEDsubsection[Commenting]{Commenting source lines}{Megjegyzés}{Megjegyzés forráskód programsorhoz}

\MEDframe{Commenting source lines}
{
The individual source lines can also be commented, see Listing\ao{ \ref{lst:Chello.cpp}}.
To produce it, the command was:
\par\noindent\lstinline|
\\MESourceFile[language=\{[ISO]C++\},wide,
decorations=\{
\\MESourcelineComment\{lst:Chello.cpp\} \{6\} \{0cm,0cm\} \{This is a comment\} \{CommentShape\}
\}
]\{lst/HelloWorld.cpp\} \{"Hello World" -- a C++ way, commenting source lines\} \{lst:Chello.cpp\}\{\}
|

\MESourceFile[language={[ISO]C++},wide,
decorations={
\MESourcelineComment{lst:Chello.cpp}{6}{0cm,0cm}{This is a comment}{CommentShape}
}
]{lst/HelloWorld.cpp}{"Hello World" -- a C++ way, commenting source lines}{lst:Chello.cpp}{}
\index{MESourceFile!commenting@\bs MESourceFile commenting}
}
{Megjegyzés a forrásprogramhoz}
{
Az egyes forráskód sorokhoz is fűzhetünk megjegyzéseket, lásd\ao{  \ref{lst:Chello.cpp}} programlista.
Ehhez a
\par\noindent\lstinline|
\\MESourceFile[language=\{[ISO]C++\},wide,
decorations=\{
\\MESourcelineComment\{lst:Chello.cpp\} \{6\} \{0cm,0cm\} \{This is a comment\} \{CommentShape\}
\}
]\{lst/HelloWorld.cpp\} \{"Hello World" -- a C++ way, commenting source lines\} \{lst:Chello.cpp\}\{\}
|
utasítást kellett kiadni.

\MESourceFile[language={[ISO]C++},wide,
decorations={
\MESourcelineComment{lst:Chello.cpp}{6}{0cm,0cm}{This is a comment}{CommentShape}
}
]{lst/HelloWorld.cpp}{"Hello World" -- a C++ way, megjegyzés a forráskód sorához}{lst:Chello.cpp}{}
%\index{MESourceFile!megjegyzés@\bs MESourceFile megjegyzés}
}
\index{MESourcelineComment@\bs MESourcelineComment}

\MEDsubsection[Balls]{Numbered balls to listing}{Golyók}{Hivatkozási pontok elhelyezése a programlistán}

\MEDframe{Numbered balls to listing}
{
On the program listing numbered balls can also be located, for referencing the 
lines from the text. This can be done using macro
\par\noindent\lstinline|\\MESourcelineListBalls[keys]\{ListingLabel\}\{List of lines\}|
\par\noindent which puts a numbered ball at the end of the listed lines.
Here \lstinline|ListingLabel| is the label of the listing, \lstinline|List of lines|
is the list of sequence numbers of the lines to be marked.
Possible key, with defaults:
\par\noindent\lstinline|color[=orange]| and \lstinline|number[=1]|. 


Notes:
\begin{itemize}
\item When making slides, the balls will be put to separated slides.
\item The positioning using geometrical positions, does not consider 'firstline'.
\end{itemize}

The marked lines can then be referenced through the balls like '(\MEBall{Listing~\ref{lst:LBhello.cpp}}{2})
is the return instruction'. It can be produced using 
\par\noindent\lstinline|\\MEBall\{Listing~\\ref\{lst:LBhello.cpp\}\}\{2\}|
}
{Hivatkozási pontok elhelyezése a programlistán}
{
Az előbbi programlistán különböző programsorokat is megjelölhetünk.
Ennek formája 
\par\noindent\lstinline|\\MESourceListBalls[keys]\{ListingLabel\}\{List of lines\}|
\par\noindent amivel a megjelölt programsorok végére kerül egy-egy számozott golyó.
Itt \lstinline|ListingLabel| a programlista címkéje, \lstinline|List of lines|
pedig azon sorszámok listája, ahová golyót szeretnénk elhelyezni.
Lehetséges kulcsok, az alapértelmezett értékkel:
\par\noindent\lstinline|color[=orange]| and \lstinline|number[=1]|. 


Megjegyzések:
\begin{itemize}
\item Dia készítéskor az egyes golyók a egy dia sorozatra kerülnek
\item A golyók elhelyezése csak geometria pozíció alapján történik, nem veszi figyelembe
a 'firstline' paramétert.
\item a golyók számozása a \lstinline|number[=1]| értékt ől indul.
\end{itemize}


Az így megjelölt sorokra később így hivatkozhatunk: "(\MEBall{Listing~\ref{lst:LBhello.cpp}}{2})
a programtörzset lezáró visszatérési utasítás".
Ehhez a 
\par\noindent\lstinline|\\MEBall\{lst:LBhello.cpp\}\{2\}| makrót kell használnunk.

}

\MEDframe{Numbered balls to listing}
{
To produce Listing\ao{ \ref{lst:LBhello.cpp}}, the macro

\par\noindent\lstinline|\\MESourceFile[language=\{[ISO]C++\},
decorations=\{
\\MESourcelineListBalls\{lst:LBhello.cpp\}\{3,8,5\}
\}
] \{lst/HelloWorld.cpp\} \{"Hello World" -- a C++ way, with balls\} \{lst:LBhello.cpp\}\{\}|
\par\noindent has been used

\MESourceFile[language={[ISO]C++},
decorations={
\MESourcelineListBalls{lst:LBhello.cpp}{3,8,5}
}
]{lst/HelloWorld.cpp}{"Hello World" -- a C++ way,  with balls}{lst:LBhello.cpp}{}
}
{Hivatkozási pontok elhelyezése a programlistán}
{
A \ao{\ref{lst:LBhello.cpp}} lista készítéséhez a
\par\noindent\lstinline|\\MESourceFile[language=\{[ISO]C++\},
decorations=\{
\\MESourcelineListBalls\{lst:LBhello.cpp\}\{3,8,5\}
\}
] \{lst/HelloWorld.cpp\} \{"Hello World" -- a C++ way, golyokkal\} \{lst:LBhello.cpp\}\{\}|
\par\noindent parancsot kell kiadni.

\MESourceFile[language={[ISO]C++},
decorations={
\MESourcelineListBalls{lst:LBhello.cpp}{3,8,5}
}
]{lst/HelloWorld.cpp}{"Hello World" -- a C++ way, golyókkal}{lst:LBhello.cpp}{}
}

\index{MESourceListBalls@\bs MESourceListBalls}

\MEDsubsection[Figures]{Figure to listing}{Ábrák}{Ábra elhelyezése a programlistán}

\MEDframe{Figure to listing}
{
Sometimes one might need to insert figures into the listing.
The macro is
\par\noindent\lstinline|\\MESourcelineFigure[keys] \{SourceLabel\} \{LineNo\} \{ShiftPosition\} \{GraphicsFile\}|.
\par\noindent Possible key is \lstinline|width[=3cm]|

}
{Ábra elhelyezése a programlistán}
{
Néha ábrát is akarhatunk elhelyezni a programlistán.
Az ezt a célt szolgáló makró
\par\noindent\lstinline|\\MESourcelineFigure[keys] \{SourceLabel\} \{LineNo\} \{ShiftPosition\} \{GraphicsFile\}|.
\par\noindent Lehetséges kulcs: \lstinline|width[=3cm]|
}

\MEDframe{Figure to listing}
{
To produce Listing\ao{ \ref{lst:forloops.v}},
 macro
\par\noindent\lstinline|
\\MESourceFile[language=\{Verilog\},wide,
decorations=\{
\\MESourcelineFigure[width=5.2cm] \{lst:forloops.v\}\{8\} \{3.0,-.3\} \{fig/forloops\}
\}
] \{lst/forloops.v\} \{Implementing \\ctext\{for\} loop with repeating HW\} \{lst:forloops.v\}\{\}|
\par\noindent was used.

\MESourceFile[language={Verilog},wide,
decorations={
\MESourcelineFigure[width=5.2cm]{lst:forloops.v}{8}{3.0,-.3}{fig/forloops}
}
]{lst/forloops.v}{Implementing \ctext{for} loop with repeating HW}{lst:forloops.v}{}
}
{Ábra elhelyezése a programlistán}
{
A \ao{\ref{lst:forloops.v}} programlista előállításához használt makró:
\par\noindent\lstinline|
\\MESourceFile[language=\{Verilog\},wide,
decorations=\{
\\MESourcelineFigure[width=5.2cm] \{lst:forloops.v\}\{8\} \{3.0,-.3\} \{fig/forloops\}
\}
] \{lst/forloops.v\} \{Implementing \\lstinline\|for\| loop with repeating HW\} \{lst:forloops.v\}\{\}|

\MESourceFile[language={Verilog},wide,
decorations={
\MESourcelineFigure[width=5.2cm]{lst:forloops.v}{8}{3.0,-.3}{fig/forloops}
}
]{lst/forloops.v}{'for'
ciklus megvalósítása HW ismétléssel}{lst:forloops.v}{}
}
\index{MESourcelineFigure@\bs MESourcelineFigure}


\MEDsection[Other]{Other related macros}{Egyéb}{Kapcsolódó egyéb makrók}

\MEDsubsection{Comparing source files}{}{Forrás fájlok összehasonlítása}


\MEDframe{Comparing source files}
{
Sometimes it is worth to compare source files, side by side.
The macro for this is
\par\noindent\lstinline|\\MESourceFileCompare[keys]\{source file1\} \{source file2\} \{caption\} \{label\}|
}
{Kapcsolódó egyéb makrók}
{
Néha érdemes forrás kód fájlokat egymás mellé helyezve összehasonlítani.
Az erre szolgáló makró
\par\noindent\lstinline|\\MESourceFileCompare[keys]\{source file1\} \{source file2\} \{caption\} \{label\}|
}

\MEDframe{Comparing source files}
{
\par\noindent The command to produce Listing \ao{\ref{lst:lower12.c}} is
\par\noindent\lstinline|\\MESourceFileCompare[language=\{[ANSI]C\}] \{lst/lower1.c\} \{lst/lower2.c\} \{Comparing two routines for converting string to lower case\} \{lst:lower12.c\}|

\MESourceFileCompare[language={[ANSI]C}]{lst/lower1.c}{lst/lower2.c}{Comparing two routines for converting string to lower case}{lst:lower12.c}

The macro does not touch the source files. In the figure, the empty lines,
allowing to compare the source files with easy, were inserted manually.
}
{Kapcsolódó egyéb makrók}
{
\par\noindent A \ao{\ref{lst:lower12.c}} programlista előállításához használt
utasítás
\par\noindent\lstinline|\\MESourceFileCompare[language=\{[ANSI]C\}] \{lst/lower1.c\} \{lst/lower2.c\} \{Comparing two routines for converting string to lower case\} \{lst:lower12.c\}|

\MESourceFileCompare[language={[ANSI]C}]{lst/lower1.c}{lst/lower2.c}{A két kisbetűssé alakító rutin összehasonlí­tása}{lst:lower12.c}

A makró a forrásfájlt nem kezeli; az ábrán a jobb összehasonlítás kedvéért
beiktatott üres sorokat kézzel kellett beírni.
}
\index{\MESourceFileCompare@\bs MESourceFileCompare}

\MEDsubsection[Output]{Source with output}{Eredménnyel}{Forrás eredménnyel}

\MEDframe{Source with output}
{
It is also useful sometimes to show the source file with its output.
The macro 
\par\noindent\lstinline|\\MESourceFileWithResult\} [keys]\{source file\} \{result file\} \{caption\} \{label\}|
\par\noindent allows to do that. 

 For producing Listing \ao{\ref{lst:calculatorwithresult}}
the command 
\par\noindent\lstinline|\\MESourceFileWithResult [language=C++,wide,decorations=\{
	\\MESourcelineListBalls \{lst:calculatorwithresult\} \{13,14,16,18,19\}
\}] \{lst/expensive_calculator.cpp\}
\{lst/calculatorresult.txt\} \{The calculator program with its result\} \{lst:calculatorwithresult\}|
\par\noindent was used.
}
{Forrás fájl eredménnyel}
{
A \ao{\ref{lst:calculatorwithresult}} program lista
a 
\par\noindent\lstinline|\\MESourceFileWithResult[language=C++,wide,decorations=\{
	\\MESourcelineListBalls\{lst:calculatorwithresult\} \{13,14,16,18,19\}
\}] \{lst/expensive_calculator.cpp\}
\{lst/calculatorresult.txt\} \{The calculator program with its result\} \{lst:calculatorwithresult\}|
\par\noindent utasítás eredménye.

Néha hasznos egy forrásfájlt a futtatás eredményével együtt megmutatni.
A
\par\noindent\lstinline|\\MESourceFileWithResult[keys]\{source file\} \{result file\} \{caption\} \{label\}|
\par\noindent makró ezt teszi lehetővé.
A forráskódban it is megjelölhetünk 'nevezetes pontokat', az eredményfájlban ez nem lehetséges.
}

\MEDframe{Source with output}
{

\MESourceFileWithResult[language=C++,wide=true,decorations={
	\MESourcelineListBalls{lst:calculatorwithresult}{13,14,16,18,19}
}]{lst/expensive_calculator.cpp}
{lst/calculatorresult.txt}{The calculator program with its result}{lst:calculatorwithresult}

}
{Forrás fájl eredménnyel}
{

\MESourceFileWithResult[language=C++,wide=true,decorations={
	\MESourcelineListBalls{lst:calculatorwithresult}{13,14,16,18,19}
}]{lst/expensive_calculator.cpp}
{lst/calculatorresult.txt}{A kalkulator program és eredménye}{lst:calculatorwithresult}
}
\index{\MESourceFileWithResult@\bs MESourceFileWithResult}

\MEDsection[Program languages]{Extra program languages}{Program nyelvek}{További program nyelvek}

\MEDframe{Extra program languages}
{ For my own goals, in addition to the programming languages
defined in package 'listings', some further languages
have been defined:
\begin{itemize}
\item diff
\item {[DIY]Assembler}
\item {[ARM]Assembler}
\item {[x64]Assembler}
\item {[y86]Assembler}
\end{itemize}
}
{További program nyelvek}
{
Saját céljaimra a 'listings' csomagban definiáltakon felül,
további program nyelveket definiáltam:
\begin{itemize}
\item diff
\item {[DIY]Assembler}
\item {[ARM]Assembler}
\item {[x64]Assembler}
\item {[y86]Assembler}
\end{itemize}
}