
%%% This is the chapter about compiling documents for the MultEdu system

\MEDchapter[Compiling]{Compiling document}{Fordítás}{A dokumentum fordítása}
\label{sec:compiling}


\MESetListingFormat{TeX}
\lstset{ basicstyle=\ttfamily\color{black}\normalsize}

\MEDsection[Manual]{Manual mode compiling}{Kézi}{Kézi fordítás}


\MEDframe{The file Main.tex}
{
File \ctext{Main.tex} is the common part of the dual compilation system. 
This contains the real source code. Any setting in this file (as well as in the included files)
overwrites the settings, in both the manual and the batch mode, so it is better
not to use any settings here. The best policy is to collect all the settings
in a separate file, which is then included in the envelope file.
}
{A Main.tex fájl}
{
A \ctext{Main.tex} fájl a közös és a két fordítási módban egyformán használt rész:
ez tartalmazza a tényleges forráskódot. Az ebben a fájlban (továbbá az ide beolvasott fájlokban)
szereplő bármely beállítás változtatás megváltoztatja a rendszer beállításait,
azaz itt nem tanácsos bármiféle beállítást használni. Érdemes az összes beállítást 
egyetlen fájlba gyűjteni, amit aztán a fő fájl magába olvas.
}


\MEDframe{Manual mode}
{
Developing course materials is best to do using an editor, integrated into an IDE.
You need to read the envelope file (corresponding to \lstinline|Demo.tex|) into the
editor and mark it as your main document. 
In the file \lstinline|Main.tex| you should insert references to the chapters of your course material.
Those chapter files should be placed in subdirectory \lstinline|src|, following the structure of 
the demonstrational material.
}
{Kézi üzemmód}
{
A tananyag fejlesztést általában valamilyen szerkesztőbe integrált fejlesztő rendszerrel
érdemes végezni. A szerkesztőbe be kell olvasni a boríték fájlt (a \lstinline|Demo.tex| megfelelőjét)
és azt gyökér dokumentumként megjelölni.
A \lstinline|Main.tex| fájlban érdemes hozzáadni a hivatkozásokat a tananyag fejezeteire, ami anyagokat
természetesen a \lstinline|src| alkönyvtárban célszerű elhelyezni, követve a demonstrációs anyag elrendezését.
}

\MEDframe{The file for settings}
{
The settings file should be placed in subdirectory \lstinline|src|, its reasonable name can be
\lstinline|Defines.tex|. The task of the wrapper file \lstinline|Demo.tex| is only to input
the setting file and the main file.

The batch compilation generates a file \lstinline|Defines.tex|, which goes into subdirectory
\lstinline|build/build/src|. (You may use it to 'cheat', what settings and how should be utilized.)
The batch compilation also generates a template file \lstinline|Defines.tex.in| in subdirectory \lstinline|src|.
The content of this file corresponds to the last pass of the batch compilation.
}
{A beállítások tárolására szolgáló fájl}
{
A beállítások tárolására szolgáló fájlt is a \lstinline|src| alkönyvtárban érdemes elhelyezni,
célszerűen \lstinline|Defines.tex| néven. A burkolóként szolgáló \lstinline|Demo.tex| feladata, hogy
ezt és a fő fájlt beolvassa.

A kötegelt mód a konfigurálás során készít egy \lstinline|Defines.tex| fájlt, de az a \lstinline|build/build/src|
alkönyvtárba kerül. (Onnét lehet puskázni, hogy mit és hogyan érdemes beállítani; miután egyszer már futott
a kötegelt fordítás.) A kötegelt fordítás egy "minta" fájlt is készít \lstinline|Defines.tex.in| néven a
\lstinline|src| alkönyvtárba. Ennek a két fájlnak a tartalma a kötegelt fordítás utolsó menetének felel meg.
}

\MEDsection[Batch]{Batch mode compiling}{Kötegelt}{Kötegelt fordítás}

\MEDframe{Batch mode compiling}
{
Batch processing serves (mainly) the goal to generate the output
from the common source in the different formats and languages.

From technical reasons, \gls{MultEdu} prepares a private copy from
the \gls{MultEdu} files, in the subdirectory \lstinline|common|  of the project.
You may safely experiment with this copy or also delete it; the next
batch compile will recreate it. (I.e. one should save the valuable developments; possibly in subdirectory \lstinline|../../common|
if you want to use it also by the other project groups.)
}
{Kötegelt fordítás üzemmód}
{
A kötegelt fordítás (főként) arra szolgál, hogy a közös forráskódból kényelmesen tudjuk
előállítani a különféle formátumokban és nyelveken anyagainkat.

Technikai okokból a tényleges fordítás előtt a rendszer saját másolatot
készít a \gls{MultEdu} szükséges fájljairól a projekt \lstinline|common| alkönyvtárába.
Ezzel a saját kópiával lehet kísérletezni, vagy akár törölni;
a következő kötegelt fordítás majd helyreállítja. (azaz a következő fordítás előtt az értékes fejlesztést el kell menteni, akár a \lstinline|../../common| alkönyvtárba, ha azt másutt is használni akarjuk.)
}

\MEDsection[Settings]{Changing default settings}{Beállítások}{Az alapbeállítások megváltoztatása}

\MEDframe{Changing default settings}
{
Settings of \gls{MultEdu} can be defined using \lstinline|\\def\{OptionName\}| macros.
If the compiler does not find the corresponding macro, the default setting will be used.
The settings differ in the cases of manual and batch compiling.
During batch processing the compiler uses settings from file
\lstinline|build/build/src/Defines.tex|, which is newly created based on
the settings in \lstinline|CMakeFiles.txt|.
During manual compilation, the settings from fájl \lstinline|src/Defines.tex|
are used. These two setting files should have the same (or at least similar) content,
but the latter one is only handled by the user.
}
{Az alapbeállítások megváltoztatása}
{
A \gls{MultEdu} alap-beállításait \lstinline|\\def\{OptionName\}| utasításokkal
lehet meghatározni. Amennyiben a fordítás előtt a fordítóprogram nem talál
ilyen meghatározást, az alapbeállítást használja.
A kézi és a kötegelt fordítás beállításai különböznek. A kötegelt feldolgozás
esetén a fordítóprogram a \lstinline|CMakeFiles.txt| fájlban megadott
beállításokkal újonnan létrehozott \lstinline|build/build/src/Defines.tex|
meghatározásokat használja, a kézi fordítás pedig a \lstinline|src/Defines.tex|
meghatározásokat. Ezek célszerűen megegyeznek, de az utóbbi beállításokat 
a felhasználónak kell megadni.

}

\MEDsubsection[Versioning]{Versioning}{Verziók}{A verziók kezelése}

\MEDframe{Versioning}
{
Multedu uses three-level version numbering (major, minor and patch).
The course materials prepared with \gls{MultEdu} have two kinds of version numbers: the user maintains his/her own version numbers, and the 
developer maintains version of \gls{MultEdu}. 

Version number of \gls{MultEdu} is located in file  \lstinline|../../common/MEMacros.tex|; better not to change it.
The own course material version number is held in file \lstinline|CMakeFiles.txt|, and that setting will be refreshed 
in the generated source files (through file \lstinline|Defines.tex|) when batch compiling.
The version number of the course material appears also in the name
of the generated file, so it is worth to use it in a consequent way.

Usage:
\par\noindent\lstinline|\\def\\Version\{major.minor.patch\}|
}
{A verziók kezelése}
{
A \gls{MultEdu} a standard háromszintű verzió számozást használja (fő és alszám, valamint folt). A \gls{MultEdu}val készült anyagoknak kétféle 
verziója van: a saját tananyagának verzióját a felhasználó tartja karban, a \gls{MultEdu} változatát pedig a fejlesztő.

A \gls{MultEdu} verziószáma a \lstinline|../../common/MEMacros.tex| fájlban található; célszerű változatlanul hagyni. 
A saját kurzus anyag verzióját a \lstinline|CMakeFiles.txt| file tartalmazza, az minden kötegelt fordítás alkalmával frissül a
\lstinline|Defines.tex| fájlban. A kézi fordításnak saját beállításai vannak, de célszerű azt átvenni a generált fájlból.

A saját verzió száma a generált kimeneti fájl nevében is szerepel,
tehát érdemes következetesen használni azt. Használata:
\lstinline|\\def\\Version\{nagy.kis.folt\}|
}

\MEDsubsection[Languages]{Languages}{Nyelvek}{Nyelvek kezelése}

\MEDframe{Languages}
{
\gls{MultEdu} can handle single- and dual-language documents.
Different spelling, section name, captions belong to the different languages. In the settings file the language must be specified,
like using setting \lstinline|\\LectureLanguage\{english\}| (this is the default).
The name of the selected language appears also in the name of the result file.
}
{Nyelvek}
{
A \gls{MultEdu} egy- és két-nyelvű dokumentumokat tud kezelni.
A különböző nyelvekhez különböző helyesírás, fejezetcímek, feliratok tartoznak. A beállításoknál kell megadni a nyelvet: ezt pl. a \lstinline|\\LectureLanguage\{magyar\}| beállítással lehet megtenni
(enélkül az alapbeállítás \lstinline|\\LectureLanguage\{english\}|).

A kiválasztott nyelv neve az eredmény file nevében is megjelenik.
}

\MEDframe{Dual-language documents}
{

In the dual-language documents, a first and second language co-exist,
meaning in which order the texts in the different languages appear in the document. This allows to develop course material in both languages simultanously, one below the other. Selecting the proper language
one can generate output in either language.
If \lstinline|\\UseSecondLanguage\{\}| is defined, then the text appearing in the second position will be processed, using the 
language features defined by \lstinline|\\LectureLanguage\{\}|.

When using batch compilation, the options  \lstinline|FirstLanguage| and
\lstinline|SecondLanguage| must be provided (that defines the language
 found in the dual-language macros in the first and second position, respectively). If option \lstinline|NEED_BOTH_LANGUAGES| is on,
 the output file will be produced in both languages.
 If it is switched off, option \lstinline|USE_SECOND_LANGUAGE| decides
 which language to use.
}
{Két-nyelvű dokumentumok}
{
A kétnyelvű dokumentumokban van egy első és egy második nyelv,
amilyen sorrendben szerepelnek a nyelvi szövegek a dokumentumban.
Ez lehetővé teszi, hogy az egymás alatt levő kétféle nyelvű kurzus anyagot összhangban tudjuk fejleszteni. A nyelv kiválasztásával a két anyag bármelyikéből tudjunk eredmény fájlt generálni.
Ha a \lstinline|\\UseSecondLanguage\{\}| definiálva van, a sorrendben
második nyelvet fogja a csomag feldolgozni, és arra a \lstinline|\\LectureLanguage\{\}| által megadott szabályokat használja.

Kötegelt fordítás esetén meg kell adnunk a \lstinline|FirstLanguage| és
\lstinline|SecondLanguage| értékét (azaz, hogy az elsőként és másodikként megtalált szöveg milyen nyelvű).
Ha bekapcsoljuk a \lstinline|NEED_BOTH_LANGUAGES| kapcsolót,
a kötegelt feldolgozás során mindkét nyelvű kimenő fájt előállítja a rendszer. Ha ez ki van kapcsolva, akkor a \lstinline|USE_SECOND_LANGUAGE| kapcsoló dönti el, melyik nyelvet fogja a rendszer használni. 

}
%\section{Document frame}
%
%\subsection{Frontmatter}
%
%\subsection{Mainmatter}
%
%\subsection{Backmatter}
%
%\section{Document sectioning}
%
%chapter, section, 
