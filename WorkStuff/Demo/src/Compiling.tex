
%%% This is the chapter about sectioning documents for the MultEdu system

\MEDchapter[Compiling]{Compiling document}{Fordítás}{A dokumentum fordítása}


\MESetListingFormat{TeX}
\lstset{ basicstyle=\ttfamily\color{black}\normalsize}

\MEDsection[Manual]{Manual mode compiling}{Kézi}{Kézi fordítás}

\MEDframe{Settings}
{
The MultEdu system works perfectly with its default settings, but it cannot read your mind.
The settings can be changed using definitions of form \lstinline|\\def\{\\xxx\}|.
The place where the settings can be changed, depends on the compilation mode.
The next two sections shows the utilization of the compilation mode,
while the third one describes the setting in details.
}
{Beállítások}
{
A MultEdu rendszer tökéletesen működik alapértelmezett beállításokkal is,
de nem gondolatolvasó. A beállításokat \lstinline|\\def\{\\xxx\}| formájú definíciókkal
lehet megváltoztatni. A beállítások helye üzemmódtól függ. A következő két szakasz 
az üzemmódok használatát mutatja be, a harmadik pedig a beállításokat ismerteti részletesen.
}

\MEDframe{The file Main.tex}
{
File \ctext{Main.tex} is the common part of the dual compilation system. 
This contains the real source code. Any setting in this file (or in the included files)
overwrites the settings, in both the manual and the batch mode, so it is better
not to use any settings here. The best policy is to collect all the settings
in a separate file, which is then included in the main file.
}
{A Main.tex fájl}
{
A \ctext{Main.tex} fájl a közös és a két fordítási módban egyformán használt rész:
ez tartalmazza a tényleges forráskódot. Az ebben a fájlban (vagy az ide beolvasott fájlokban)
szereplő bármely beállítás változtatás megváltoztatja a rendszer beállításait,
azaz itt nem tanácsos bármiféle beállítást használni. Érdemes az összes beállítást 
egyetlen fájlba gyűjteni, amit aztán a a fő fájl magába olvas.
}


\MEDframe{Manual mode}
{
Developing course materials is best to do using an editor, integrated into an IDE.
You need to read the real main file (corresponding to \lstinline|Demo.tex|) into the
editor and mark it as your main document. 
In the file \lstinline|Main.tex| you should insert references to the chapters of your course material.
Those chapter files should be placed in subdirectory \lstinline|src|, following the structure of 
the demonstrational material.
}
{Kézi üzemmód}
{
A tananyag fejlesztést általában valamilyen szerkesztőbe integrált fejlesztő rendszerrel
érdemes végezni. A szerkesztőbe be kell olvasni a fő fájlt (a \lstinline|Demo.tex| megfelelőjét)
és azt gyökér dokumentumként megjelölni.
A \lstinline|Main.tex| fájlban érdemes hozzáadni a hivatkozásokat a tananyag fejezeteire, ami anyagokat
természetesen a \lstinline|src| alkönyvtárban célszerű elhelyezni, követve a demonstrációs anyag elrendezését.
}

\MEDframe{The file for settings}
{
The settings file should be placed in subdirectory \lstinline|src|, its reasonable name can be
\lstinline|Defines.tex|. The task of the wrapper file \lstinline|Demo.tex| is only to input
the setting file and the main file.

The batch compilation generates a file \lstinline|Defines.tex|, which goes into subdirectory
\lstinline|build/build/src|. (You may use it to 'cheat', what settings and how should be utilized.)
The batch compilation also generates a template file \lstinline|Defines.tex.in| in subdirectory \lstinline|src|.
The content of this file corresponds to the lass pass of the batch compilation.
}
{A beállítások tárolására szolgáló fájl}
{
A beállítások tárolására szolgáló fájlt is a \lstinline|src| alkönyvtárban érdemes elhelyezni,
célszerűen \lstinline|Defines.tex| néven. A burkolóként szolgáló \lstinline|Demo.tex| feladata, hogy
ezt és a fő fájlt beolvassa.

A kötegelt mód a konfigurálás során készít egy \lstinline|Defines.tex| fájlt, de az a \lstinline|build/build/src|
alkönyvtárba kerül. (Onnét lehet puskázni, hogy mit és hogyan érdemes beállítani; miután egyszer már futott
a kötegelt fordítás.) A kötegelt fordítás egy "minta" fájlt is készít \lstinline|Defines.tex.in| néven a
\lstinline|src| alkönyvtárba. Ennek a két fájlnak a tartalma a kötegelt fordítás utolsó menetének fele meg.
}

\MEDsection[Batch]{Batch mode compiling}{Kötegelt}{Kötegelt fordítás}

\MEDframe{Batch mode compiling}
{
a
}
{Kötegelt fordítás üzemmód}
{
a
}

\MEDsection[Settings]{Changing default settings}{Beállítások}{Az alapbeállítások megváltoztatása}

\MEDframe{Changing default settings}
{
}
{Az alapbeállítások megváltoztatása}
{
}
%\section{Document frame}
%
%\subsection{Frontmatter}
%
%\subsection{Mainmatter}
%
%\subsection{Backmatter}
%
%\section{Document sectioning}
%
%chapter, section, 
