\MEDtext
{
For teaching my own courses, I developed a set of macros, since to display  the course material
under different circumstances different forms of teaching materials are needed. On the lectures
I present the theoretical material in form of slides, and the explanation referring to the slides
(of course in somewhat compressed form) I offer for my students, in a booklet-like form. My students
studies that material either from printed hard copies, or from screen, using a browser or sometimes
on mobile devices. The field is in continuous development, so I need to develop my teaching material
also continuously. Because of this, it is a must to develop those different forms of the material
synchronously. The simplest way to do so, is to use the same source, with proper formatting instructions.
It is a serious challange, to develop course material for the today's students, who are used to
lessons with high resolution, attractive graphics, good computer background.

As a common base, I used LaTeX, from which I produce the slides from the lectures using package beamer,
and the reading material using package 'memoir'. This latter one can even reach the "on demand printing"
quality. The printed material attempts to catch the attention with attractive graphical appearance,
using above the average amount of figures (of course on can make also 'book-like' book, too). The
booklet-like version contains all figures from the lectures, and some comprehensive version of
the text of the lecture. The same text appears in screen-oriented form in the WEB-book format, and
in the eBook compatible (native PDF) format.In those two forms (mainly targeting the small-screen
mobile devices) bigger fonts are used and one figure/screen is displayed.

To satisfy those, somewhat contradictional  requirements, one must make bargains,and more time and care
must be invented in the formatting. The macros allow to support also foreign languages, and even
to prepare course in English and your own language side by side. Using the possibilities of LaTeX,
animations, movies, web-pages, sound files, etc. can also be embedded, but one has to think about
the equivalent appearance on hard copies.
}
{
Kurzusaim oktatásához saját makrókészletet fejlesztettem, mivel az oktatandó anyag megjelenítéséhez
különböző körülmények között különböző formákra van szükség. Az elméleti anyagot az előadásokon
diasorozat alapján mutatom be, és a diákhoz fűzött magyarázatokat (természetesen tömörítve) 
jegyzet-szerű formában a hallgatóság számára is rendelkezésre bocsátom. A hallgatóság ezt az 
anyagot részben kinyomtatva, részben képernyőn olvasva (akár mobil eszközökön is) tanulmányozza.
A terület folyamatos fejlődése miatt a tananyag is állandó fejlesztésre szorul, ezért feltétlenül
szükséges, hogy az említett megjelenési formákat egymással szinkronban lehessen fejleszteni.
Ennek legegyszerűbb megvalósítási formája, hogy egyazon forrásból, megfelelő formattálási utasításokkal
készítem a tananyagokat.
Számítógéppel alaposan megtámogatott, nagy felbontáshoz és vonzó grafikához
szokott hallgatóság számára a fenti feltételeknek megfelelő tananyagot készíteni komoly kihívás.

Közös alapként a LaTeX nyelvet használtam, amely nyelven készült forrásból az elterjedten használt
Beamer prezentáció készítő makró csomaggal állítom elő az előadáson bemutatandó diákat, és a memoir
könyv készítő makró csomaggal a hallgatóság számára rendelkezésre bocsátandó "tananyagot". Ez utóbbi
akár az "on demand printing" minőséget is elérheti. Vonzó grafikus megjelenéssel, a szokásoshoz képest 
sokkal több ábrával igyekszik felkelteni az anyag a hallgatóság figyelmét (de lehet belőle kevésbé "fancy",
inkább "plain" stílusú, de még mindig könyv minőségű változatot is készíteni). A jegyzet-szerű 
változat az előadáson bemutatott ábrákat és szöveget teljes egészében tartalmazza, az előadás
szövegének egy tömörített változatával kiegészítve. Ugyanez a könyv-szerű anyag jelenik meg a képernyős
olvasásra szánt WEB-es formátumban, illetve az eBook kompatibilis (natív PDF) változatban. Ebben a változatban az anyag egy-képernyőnyi darabokra van tördelve,
és (főként kisebb képernyőjű mobil eszközökre gondolva) nagyobb betűkkel, egy ábra/képernyő módon
jelenik meg.

A többféle, egymásnak ellentmondó megjelenítési igény természetesen csak kompromisszumokkal
oldható meg, így a tananyag megírása során a szöveg megjelenés formázására több gondot és időt kell fordítani.
A makrócsomaggal akár egyidejűleg idegen nyelvre is lehet ugyanazon tananyagot fejleszteni.
A LaTeX lehetőségeivel akár animáció, mozgófilm, WEB-lap, hang, stb. is beépíthető, természetesen gondolni
kell a nyomtatott anyag ekvivalens megjelenítésére.
}

